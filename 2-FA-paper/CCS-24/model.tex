%!TEX root = main.tex

%\subsection{The Model} \label{sec::model}


\section{Threat Model and System Design}\label{sec::model}

A 2FA scheme involves two players, (1) {User ($U$)}: an honest party that tries to prove its identity to a server by using a combination of a PIN and a token and (2) {Server ($S$)}:  a semi-honest adversary which follows the protocol's instructions and tries to learn $U$'s PIN. It also tries to authenticate itself to $U$.  

We enable server-to-hardware token communication through the user's computer (the client), assuming the token has a camera for scanning 2-D barcodes containing messages sent by the server via the client, similar to previous works such the ones in \cite{JareckiJKSS21,Digipass-website,Gemalto}. 
%
%We let the server communicate with the hardware token through the user's computer, \ie, the client. Specifically, similar to previous works (\eg, those in \cite{JareckiJKSS21,Digipass-website,Gemalto}), we assume the token has a camera that lets the token scan a 2-D barcode containing messages the server sends to it via the client. Each of the above parties may have several instances running concurrently. 
%
In this work, we denote instances of user and server by  $U^{i}$ and  $S^{j}$ respectively. Each instance is called an oracle.  Multiple instances of these parties may run concurrently. %\todo{Assume high bandwidth in one direction, low in another} 

Figure \ref{fig:setup.} outlines the message flow of our 2FA scheme during the authentication phase.  At a high level, the authentication phase works as follows. Any time the user wants to authenticate itself to the server, the user sends a message to the server via the client.  The server replies to the client with a challenge and transaction details (Step 1). The user scans with the token the message that the client received  (Step 2). 

The token shows the transaction details on the screen (Step 3). The user types the PIN into the token (Step 4). The token generates a response (Step 5). The user manually types the response into the client (Step 6).  The client sends the response to the server for authentication (Step 7). 




%\vspace{-2mm}
\begin{figure}
\begin{centering}
\includegraphics[height=3.4cm, width=8.5cm]{setup}
\end{centering}
%\vspace{-2mm}
\caption{\label{fig:setup.}Protocol participants and message flows}
%\vspace{-2mm}
\end{figure}
%\vspace{-5mm}
 %
%\subsection{Formal Model} \label{sec::model}
%




To formally capture the capabilities of an adversary, $\mathcal{A}$, in hardware token-based 2FA,  we use the model introduced in \cite{BellarePR00}. 


%%%%%%%%%
%In this model, the adversary’s capabilities are cast via queries that it sends to different oracles, \ie, instances of the honest parties; the user and server interact with each other for some fixed number of flows, until both instances have terminated. By that time, each instance should have accepted holding a particular session key (\seckey), session id (\SID), and partner id (\PID). At any point in time, an oracle may \emph{``accept''}. When an oracle accepts, it holds \seckey, \SID, and \PID. A user instance and a server instance can accept at most once. The above model was initially proposed for password-based key exchange schemes in which the adversary does not corrupt either player.  Later, Wang \textit{et al.} \cite{WangW18}  added more queries to the model of  Bellare \textit{et al.} to make it suitable for two-factor authentication schemes. The added queries allow an adversary to learn either of the user's factors (\ie, either PIN or secret parameters stored in the hardware token) or the server's secret parameters.  Below, we restate the related queries. 
% 
%
%
%\begin{itemize}[leftmargin=4.6mm]
%%
%\item [$\bullet$] \execute($U^{i}, S^{j}$): this query captures \textbf{passive} attacks in which the adversary, $\mathcal{A}$, has access to the messages exchanged between $U^{i}$ and $S^{j}$ during the correct executions of a 2FA protocol, $\pi$. 
%
%
%
%\item [$\bullet$] \reveal($I$): this query models the misuse of the session key $sk$ by instance $I$.  Adversary $\mathcal{A}$ can use this query if $I$ holds a session key; in this case, upon receiving this query, $sk$ is given to $\A$. 
%%
%
%
%
%\item [$\bullet$] \test($I$): this query models the semantic security of the session key. It is sent at most once by $\A$ if the attacked instance $I$  is ``fresh'' (\ie, in the current protocol execution $I$ has accepted and neither it nor the other instance with the same $\SID$ was asked for a Reveal query). This query is answered as follows. Upon receiving the query, a coin $b$ is flipped. If $b=1$, then session key $sk$ is given to $\A$; otherwise (if $b=0$), a random value is given to $\A$. 
%
%
%
%\item [$\bullet$] \send($I, m$):  this query models \textbf{active} attacks where $\mathcal{A}$ sends a message, $m$, to instance $I$ which follows  $\pi$'s instruction, generates a response, and sends the response back to $\mathcal{A}$.  Query  \send($U^{\sss i}, \text{start}$) initialises $\pi$; when it is sent, $\mathcal{A}$ would receive the message that the user would send to the server. 
%
%
%
%%
%\item [$\bullet$]  \corrupt($I, a$): this query models the adversary's capability to corrupt the involved parties. 
%
%
%
%\begin{itemize}
%
%\item if $I=U$:  it can learn (only) one of the factors of $U$. Specifically, 
%%
%
%
%
%\begin{itemize}
%\item if $a=1$, it outputs $U$'s PIN. 
%
%
%
%\item if  $a=2$, it outputs all parameters stored in the  token. 
%  \end{itemize}
%  
%  
%  
%\item if $I=S$, it outputs all parameters stored in $S$. 
%\end{itemize}
%%%
%\end{itemize}
%%%%%%%%%%%

 \vspace{-1mm}
\subsubsection{Authenticated Key Exchange (AKE)  Security.} Security notions (\ie, session key's semantic security and authentication) are defined with regard to the executing of protocol $\pi$, in the presence of   $\mathcal{A}$. 


To this end, a game $Game^{\sss ake} (\mathcal{A},\pi)$ is initialized by drawing a PIN  from the PIN's universe,
providing coin tosses to $\A$ as well as to a set of oracles, and then running the adversary by letting it ask a polynomial
number of queries, including \test($I$) query, which is answered as follows. Upon receiving the query, a coin $b$ is flipped. If $b=1$, then session key $sk$ is given to $\A$; otherwise (if $b=0$), a random value is given to $\A$. At the end of game $Game^{\sss ake}$, $\A$  outputs its guess $b'$ for bit $b$ involved in the Test-query. 



\noindent\underline{\textit{Semantic security.}} Informally, it requires that the privacy of a session key be preserved in the presence of $\A$. We say that $\A$  wins if it manages to correctly guess (bit $b$ associated with) the session key. We denote its advantage as the probability that $\A$  can correctly guess $b$, such an advantage is denoted as  $Adv_{\sss\pi}^{\sss ss}(\A)$.  Protocol $\pi$ is said to be semantically secure if $\A$'s advantage is negligible in the security parameter, \ie, $Adv_{\sss\pi}^{\sss ss}(\A)\leq \mu(\lambda)$.



%The authentication property is another  fundamental feature that a two-factor authentication scheme must offer; 


\noindent\underline{\textit{Authentication.}} It requires that $\A$ must not be able: (a)  to impersonate $U$, even if it has access to the traffic between the two parties as well as having access to either $U$'s PIN, or its hardware token, or (b) to impersonate  $S$, even if it has access to the traffic between the two parties.   Protocol $\pi$ is said to achieve mutual authentication if for any adversary $\A$  interacting with the parties, there exists a negligible function $\mu(.)$ such that for any security parameter $\lambda$ the advantage of $\A$  is negligible in the security parameter, \ie, 
%
$Adv_{\sss\pi}^{\sss aut}(\mathcal{A})\leq \mu(\lambda)$.

In our scheme, the user must verify a message (e.g., a bank transaction) during the key agreement and authentication phase. To facilitate deterministic verification, especially for the scheme's proof, we define a predicate $y\leftarrow \phi(m, \gamma)$, where $y\in \{0,1\}$. This predicate,  taking a message $m$ (\eg., bank's transactions) and  policy $\gamma$ (\eg, a user's policy specifying a payment amount and destination account number), validates whether the message conforms to the policy, yielding $1$ for a match, and $0$ otherwise. 


Appendix \ref{sec::model-long} provides a more in-depth formal discussion of the threat model. 



%However, we do not use the original BPR2000
%model directly, but adopt the reified version proposed by
%Bresson et al. [50] with a few key modifications so that we
%can define the special security goals (e.g., security against
%smart card loss attack) for two-factor authentication. We
%refer the reader unfamiliar with the BPR2000 model to [35],
%[50] for more details.







