%!TEX root = main.tex


\section{Related Work}\label{Related-Work}





%In order for clients to prove their identity to a (local/remote) computer resource, they provide a piece of evidence, called an ``authentication factor''. The authentication factors can be broadly categorised based on (a) what clients know (a.k.a. knowledge factors), such as PIN, password, (b)  what a client possesses (a.k.a. possession factors), such as an access card or physical hardware token, or (c) what clients are (a.k.a. inherent factors), such as a fingerprint, or iris. 
%
%
%The knowledge factors are still the most predominant factors used for authentication \cite{bonneau2010password,JacommeK21}, mainly because their implementations impose low costs on the verifiers and require clients to invest minimal efforts to authenticate themselves. However, the knowledge factors themselves are not strong enough to adequately prevent impersonation, for various reasons, e.g., clients may pick weak passwords that can be easily guessed, or reuse them to authenticate themselves to many servers which would increase their leakage if one of the servers is penetrated by adversaries \cite{SinigagliaCCZ20,JacommeK21}. 
%
%In general, authentication methods that depend on more than one factor are more difficult to compromise
%than single-factor methods. Although the authentication mechanisms that rely on more than two factors may guarantee stronger security, they would impose higher costs and negatively affect clients' experience; therefore, two-factor authentications have gained special attention; including those that rely on a combination of password/PIN and device possession that allows clients to generate one-time passwords (OTPs) when a client types its secret (e.g., password or PIN) into the device. An OTP can be sent to and checked by a remote verifier to authenticate a client. The combination of a password or PIN and a hardware token has gained popularity among various e-commerce and banks.  


In this section, we outline the related work. Appendix \ref{Survey-of-Related-Work} presents a survey of related work. 


\subsection{Common Approaches for Generating OTP}

In authentication systems using both knowledge and possession factors, the hardware token combines the user's entered secret with  (i) random challenge, (ii) an internal counter, or (iii) current accurate time to generate a unique OTP, sometimes after validating the secret. There are 2FA solutions, including our proposal in this paper, that combine these approaches.


%
%In authentication mechanisms that rely on a combination of knowledge and possession factors, once the user enters the secret into the hardware token, the token (in some cases after validating the secret) combines this secret with the output of one of the following methods to generate a unique OTP: 
%
%\begin{enumerate}[leftmargin=5mm]
%
%\item \textit{a random challenge}: This approach mandates the server to send a random challenge to the token through the client, requiring protocols to maintain the confidentiality of these challenges in the presence of potential eavesdropping adversaries.
%
%
%
%
%
%% this approach requires the server to send a random challenge to the token (through the client). Those protocols that use this approach need to ensure the random challenges themselves remain confidential in the presence of an eavesdropping adversary.
%
%\item  \textit{an internal counter}:  Solutions using this approach must consider token-side counter desynchronisation.. 
%
%
%\item \textit{the current accurate time}: This approach relies on synchronised clocks between the authentication server and token, but they may become unsynchronised over time. 
%
%\end{enumerate}



%There exist 2FA solutions (including the one we propose in this paper) that employ a combination of the above approaches. 
\subsection{Variants of OTP Hardware Tokens}

\subsubsection{Connected Token.}


%One approach to hardware tokens is to plug them into the computer's USB port, but for such devices to work there must be corresponding software installed on the computer, and this might not be possible on shared devices or computers implementing a corporate IT policy.
%Also, if any special software is needed it would have to be implemented for every supported operating system and processor architecture, and periodically updated, so imposing higher development costs.
%In the longer term, support for hardware authentication devices may become a common operating system provision, but current proposed standards to do so, such as FIDO 2.0 do not support transaction authentication and are not designed to mitigate compromise of the user's computer.
%For this reason, connected authentication hardware does not meet the requirements we have set out.
%Similarly using a smartphone as the hardware authentication device does not fulfil the usability requirement because not everyone has a smartphone or is willing to install special software on it, and smartphones have a large attack surface that can be exploited to extract the authentication secrets.







This token needs a physical connection to a computer (e.g., laptop or card reader) for user authentication. Once connected, it sends authentication information to the computer, either automatically or with a button press. USB tokens and smart cards are popular in this category. Google, Dropbox, and the FIDO Alliance have developed USB hardware token specifications. YubiKey is a well-known one implementing FIDO specifications.





However, researchers have discovered various vulnerabilities within FIDO specifications, e.g., see  \cite{PanosMNPX17,ChangMSS17,LoutfiJ15,ndss/FengLP021}.  Also, these devices require specific software on the computer, which may not work for shared devices or adhere to a corporate IT policy. If specialised software is needed, it must be developed for each supported OS and processor type, requiring ongoing updates and higher development and upkeep expenses.


%If any specialised software is required, it must be developed and implemented for each supported operating system and processor architecture. This entails periodic updates, leading to increased development and maintenance costs.


Smart card technology is another widely employed authentication method. Since its introduction in \cite{chang1991remote}, there have been numerous protocols for smart card-based 2FA (\eg, in \cite{gupta2021machine,WangW18,radhakrishnan2022dependable}). % along with a few works that identify vulnerabilities of existing solutions, \eg, in \cite{TianLHL20,WangGCW16,ChaturvediDMM16}. 
%
However, the existing smart card-based solutions (\eg, in \cite{gupta2021machine,WangW18,radhakrishnan2022dependable}) are often based on public-key cryptography which imposes a high computation and energy cost. Also some solutions (e.g., in \cite{kim2009more}) rely on tamper-proof secure chipsets embedded in the card which would ultimately increase its cost. 


%There are a number of different types, including USB tokens, smart cards and wireless tags.[7] Increasingly, FIDO2 capable tokens, supported by the FIDO Alliance and the World Wide Web Consortium (W3C), have become popular with mainstream browser support beginning in 2015.





\subsubsection{Disconnected Tokens.}

This type of token does not have a physical connection to a user's computer making them more convenient than connected tokens. 
%
%A disconnected token is often equipped with a built-in screen and a keypad allowing a user to type in the knowledge factor and view the OTP on the screen.  
%
Two main categories of disconnected tokens are (i) dedicated hardware-based token and (ii) mobile phone-based token. 
%
RSA SecureID \cite{secureID}, OneSpan Digipass 770 \cite{Digipass-website}, and Thales Gemalto SWYS QR Token  \cite{Gemalto} fall in the former category. 



In RSA SecureID, the OTP is generated using the current time and a secret key (allocated to the user and) stored in the token \cite{biryukov2003cryptanalysis}. Thus, not only the token has to have a synchronised clock with the server, also the token's OTP can be generated by an adversary having physical access to the token.%, as it can extract the token's key.  

The main advantage of  Digipass 770 and Thales Gemalto SWYS QR Token to RSA SecureID is that they allow users to see and verify the transaction details through the token. Therefore, the user is given more understandable information about the transaction it is approving,
so phishing (by man-in-the-browser attacks or social engineering attacks) becomes harder. Our investigation suggests that Digipass 770 and Thales Gemalto SWYS QR Token also \emph{locally stores and verify} users' PINs. 




%
% later, when the client enters a PIN, the token locally verifies the PIN and if it accepts the PIN, then it generates an OTP. 
%Specifically, once a user receives the token, it also receives an activation code from the verifier, \eg, the user's bank.  Then, the user (i) registers the activation code in the token and (ii) registers the activation code to the verifier, so the verifier knows that this specific user has a token with the provided activation code. Then, the user registers its PIN in the token which stores it locally. Every time a user uses the verifier's online system  (e.g., online banking) and makes a transaction, the system generates and displays an encrypted visual image. The user uses the token (camera) to scan the image and then enters its PIN into the token. 
%
%Next, the token checks the PIN; if the PIN matches the previously registered PIN, then it decrypts the image and displays the transaction's content on the token's screen which allows the user to check whether the transaction is the one it has made. If the user accepts the transaction and presses a certain button, then the token generates and displays an OTP that the user can type into the verifier's online system~\cite{Digipass-website}.  Thales Gemalto SWYS QR Token Eco also uses a mechanism similar to the one we described above. 

%Thus, in the above solutions, an attacker who has physical access to the token can extract the PIN and impersonate the client. 

%Jules et al. \cite{juels2016configurable} discussed that the adversary who can intercept the user and server's communication and also has physical access to the user's token or the server's storage can extract the user's PIN and impersonate the user. To address the issue they also suggested a solution that can address the above issue by using (i) a forward-secure pseudorandom number generation, (ii) multiple servers, etc. Yet, the proposed scheme lacks formal proof and neglects situations where users must verify transaction details on the token.

Matsuo \textit{et al.} \cite{MatsuoMY11} proposed an efficient symmetric-key cryptography scheme, but it relies on a secure chipset to store a secret key, which differs from most schemes, including ours. Also, their scheme does not address offline dictionary attacks and assumes the server is never breached. 
%
Jarecki \textit{et al.} \cite{JareckiJKSS21} introduced a protocol for single-server scenarios, safeguarding against PIN extraction and user impersonation even if the server or token is compromised. It relies on symmetric and asymmetric encryption and  hash functions.


%proposed a (single server) protocol to ensure that even if the server or the token is corrupted a user's PIN cannot be extracted and the adversary cannot impersonate an honest user. It is mainly based on a hash function, both symmetric and asymmetric-key encryptions, and (Diffie–Hellman) key exchange. 

%This scheme suffers from several shortcomings; namely, (1) it imposes a high computation and communication cost due to its complexity,  the use of public-key cryptography, and numerous rounds of communication, even between the user's computer and token, (2) it requires the token to perform asymmetric-key operations and invoke symmetric-key primitives many times, (3) it mandates users to input their passwords/PINs into their own (personal) computers (referred to as client $C$ in the paper) instead of entering them into the designated hardware token. 



This scheme has multiple drawbacks: (1) it incurs high computation and communication costs due to its complexity, and numerous communication rounds, (2) it demands the token to perform multiple asymmetric-key operations and numerous symmetric-key operations, (3) it obliges users to enter passwords/PINs on their personal computers instead of using the designated hardware token. %This poses a concern since the PINs become more vulnerable to exposure by attackers. This is due to the fact that users' computers are nearly always connected to the Internet, serve multiple functions, and are more susceptible to unauthorised access, (4) to maintain the security of the protocol, it is necessary for users' personal computers to be fully trusted, especially in situations where the token or the server is compromised. However, this introduces an additional assumption that may not always be desirable.


%for the protocol's security to hold, it is required that users personal computers to be fully trusted, in the case where the token or the server is corrupted; this is an additional assumption that may not be always desirable. %In contrast, our protocol does not suffer from the above shortcomings. 
%
%This protocol requires the user (in addition to remembering its PIN) to remember/store a cryptographic secret key locally (but not on the token), as a result of invoking a subroutine called asymmetric  ``password-authenticated key exchange'' (PAKE). 


%Moreover, an alternative authentication protocol, presented in  \cite{zhang2020strong}, operates independently of a trusted chipset. However, it has been specifically tailored for ``federated identity systems'' and thus is not well-suited for two/multi-factor authentication scenarios.

%Furthermore, there is another authentication protocol, that does not rely on a trusted chipset, presented in \cite{zhang2020strong}. Nevertheless, it has been designed for ``federated identity systems'' and is not suitable for two/multi-factor authentication settings. 



%Thus, unlike the above solutions, offer all the following features simultaneously; it (a) efficient, (b) is provably secure, (c) is resilient against the adversary which may corrupt the server or have access to the client's token, (d) relies on a single server, and (e) allows the client to also check the transaction detail on its token. 




The solutions introduced in \cite{SARA22,KoganMB17,KonothFFARB20}
fall in the later category, i.e., mobile phone-based token. In these solutions, a mobile phone serves as a hardware token to generate a OTP. These solutions frequently leverage the additional capabilities inherent in mobile phones, including features like a ``trusted execution environment'', direct server communication, and a rechargeable battery. 

Mobile phone-based OTP tokens have limitations: they require network coverage, may not be suitable for all users due to privacy concerns when sharing phone numbers, and not everyone owns a smartphone or is willing to install special software on it. Moreover, smartphones have a large attack surface that can be exploited to extract authentication secrets. 
%
For major banks, it is often not commercially viable to refuse service to customers unable or unwilling to use smartphone-based authentication, and in countries that impose a universal service obligation on certain banks (e.g., the United Kingdom) it would not be legal to do so.
Therefore, it is established practice for banks to offer dedicated hardware-based authentication tokens, at least to customers who request these.

%
%\end{enumerate}




%
%. Disconnected tokens are the most common type of security token used (usually in combination with a password) in two-factor authentication for online identification



% We refer readers to \cite{SinigagliaCCZ20} for a survey of different methods of authentications and their adoptions by banks.  




%Therefore, instead of using only a (clientname and) password/PIN, clients need to prove the possession of an additional device and/or an inherent factor. 