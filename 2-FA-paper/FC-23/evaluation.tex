%!TEX root = main.tex

\vspace{-4mm}
\section{Evaluation}\label{sec::eval}
\vspace{-2mm}

In this section, we briefly analyse and compare our 2FA protocol with the smart-card-based protocol proposed in  \cite{WangW18} and the hardware token-based protocol in \cite{JareckiJKSS21} as the latter two protocols are relatively efficient, do not use secure chipsets, and they consider the same security threats as we do. We summarise the analysis result in Table \ref{comparisonTable}. We refer readers to Appendix \ref{sec::long-eval} for a more detailed evaluation. 

%Before we compare the two protocols' costs, we highlight a vital difference between the two. 






%!TEX root = main.tex


%\vspace{-2mm}
\begin{table} 

\begin{center}
\caption{ \small Comparison of efficient two-factor authentication protocols.}  \label{comparisonTable} 
%\renewcommand{\arraystretch}{1}
\begin{tabular}{|c|c|c|c|c|c|c|c|c|c|} 
\hline

{\cellcolor{gray!40}\scriptsize {Features}} &\cellcolor{gray!40}{\scriptsize {Operation}}&\cellcolor{gray!40}{ \scriptsize {Our Protocol}}&\cellcolor{gray!40}{\scriptsize{\cite{WangW18}}}&\cellcolor{gray!40} {\scriptsize{\cite{JareckiJKSS21}}}  \\
\hline

\multirow{2}{*}{\scriptsize Computation cost}


&\scriptsize{Sym-key}&\cellcolor{gray!20}\scriptsize$18$&\cellcolor{gray!20}\scriptsize$19$&\cellcolor{gray!20}\scriptsize $7$\\


\cline{2-5}

&\scriptsize {Modular expo.}&\cellcolor{gray!20}\scriptsize {$0$}&\cellcolor{gray!20}\scriptsize$5$&\cellcolor{gray!20}\scriptsize$12$\\

\hline

\scriptsize  Communication cost&$-$&\cellcolor{gray!20}\scriptsize$2804$-bit&\cellcolor{gray!20}\scriptsize$3136$-bit&\cellcolor{gray!20}\scriptsize$3900$-bit\\

\hline 
\scriptsize Not requiring multiple pass/PIN&$-$&\cellcolor{gray!20}\scriptsize{\textcolor{blue}\checkmark}&\cellcolor{gray!20}\scriptsize\textcolor{red}{$\times$}&\cellcolor{gray!20}\scriptsize\textcolor{red}{$\times$}\\ 
\hline
\scriptsize Not requiring modular expo.  &$-$&\cellcolor{gray!20}\scriptsize{\textcolor{blue}\checkmark}&\cellcolor{gray!20}\scriptsize\textcolor{red}{$\times$}&\cellcolor{gray!20}\scriptsize\textcolor{red}{$\times$}\\ 

\hline


\scriptsize Security assumption  &$-$&\cellcolor{gray!20}\scriptsize{Standard}&\cellcolor{gray!20}\scriptsize{Random oracle}&\cellcolor{gray!20}\scriptsize{Random oracle}\\ 

\hline
\end{tabular}
%}
%\renewcommand{\arraystretch}{1}
%\end{footnotesize}
\end{center}
%}
\end{table}







\vspace{-5mm}
\subsection{Computation Cost}
%\vspace{-2mm}

In our protocol, each party (client or server) invokes the authenticated encryption scheme $4$ times and the pseudorandom function $5$ times. In the protocol proposed in \cite{WangW18}, the client invokes a hash function $11$ times and performs $3$ modular exponentiations while the server invokes the hash function $8$ times and performs $2$ modular exponentiations. Moreover, in the protocol presented in \cite{JareckiJKSS21}, the client invokes a hash function $3$ times and runs symmetric key encryption once. It also performs $10$ modular exponentiations. While the server invokes a pseudorandom function once and performs at least $2$ modular exponentiations and $2$ symmetric-key encryptions. Thus,  our protocol and the ones in \cite{WangW18,JareckiJKSS21} involve a constant number of symmetric key primitive invocations; however, our protocol does not involve any modular exponentiations, whereas those in  \cite{WangW18,JareckiJKSS21} involve a constant number of modular exponentiations which leads to a higher cost. 

\vspace{-4mm}
\subsection{Communication Cost}
\vspace{-2mm}

In our protocol, the communication cost of the client is $1268$ bits and the server is $1536$ bits. However, in the protocol proposed in \cite{WangW18},  the communication cost of the client is $1952$ bits and the server is  $1184$ bits; while in the protocol developed in \cite{JareckiJKSS21} the client's and server's communication costs are at least $2856$ and $1044$ bits respectively. Hence, our protocol imposes $10\%$ and $40\%$ lower communication costs than the protocols in \cite{WangW18} and \cite{JareckiJKSS21} do respectively.


\vspace{-4mm}
\subsection{Other Features}
\vspace{-2mm}
 In our protocol, a client needs to know only a single secret, i.e., a  PIN. Nevertheless, in the protocol in \cite{WangW18}, a client has to know an additional secret, i.e., a random ID. As shown in \cite{Scott12a}, the scheme in \cite{WangW18} will not remain secure, even if only the client's ID is revealed. The protocol in \cite{JareckiJKSS21} requires the client (in addition to remembering its PIN) to locally store a cryptographic secret key of sufficient length, e.g., $128$ bits; this secret key must not be kept on the device. 
 % 
 Furthermore, our protocol is secure in the standard model while the protocols in \cite{WangW18,JareckiJKSS21} are in the non-standard random oracle model. 

