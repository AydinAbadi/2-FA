%!TEX root = main.tex

\section{Formal Security Analysis}\label{sec::Formal-Security-Analysis}

In this section, we present the security proof of the protocol, presented in Section \ref{sec::the-protocol}.  First, we prove the semantic security of the scheme and then prove its authentication. 

\subsection{Semantic Security}\label{sec::semSec-proof}

In this section, we assert that under standard assumptions protocol $\psi$, presented in Section \ref{sec::the-protocol}, securely distributes
session keys. To do so, we incrementally define a sequence of games starting at the real
game $G_{\sss 0}$ and ending up at $G_{\sss  7}$. We first define various events in every game and then explain each game. 

\begin{itemize}
%
\item  $S_{\sss  i}$: it takes place if $b=b'$, where $b$ is the bit involved in the test query and $b'$ is the output of $\A$ which wants to guess $b$. 
%
\item $Auth_{\sss  i}$: it occurs if $\A$ generates and sends to the server an authenticator message that is accepted by the server.
%
\item $Enc_{\sss  i}$: occurs if $\A$ submits data it has encrypted by itself using the correct key that an honest party would use to encrypt. 
%
\end{itemize}

\begin{itemize}
%
\item[$\bullet$] \textit{\textbf{Game}} $G_{\sss  0}$: This is the real attack game. Several oracles are  available to the adversary; namely, the pseudorandom function ($\prf$),  the encryption/decryption oracles ($\mathtt{Enc}$ and $\mathtt{Dec}$) and all instances $U^{\sss  i}$ and $S^{\sss  j}$.
%
According to the definition we presented in Section \ref{sec::model}, the advantage of the adversary in this protocol is: 
%
\begin{equation}\label{eq::adv-ake-}
Adv_{\sss  \psi}^{\sss  ss}(\A)=2Pr[S_{\sss  0}]-1
\end{equation}

Similar to the security proof in \cite{BressonCP03}, we assume that if any of the games halts and $\A$ does not output $b'$, then $b'$  is chosen at random. Also, if $\A$ has not finished playing the game after sending $q_{\sss  s}$  \send$(.)$ queries or if it plays the game more than a predefined time $t$, the game is stopped and a random value is assigned to $b'$. 
%



\item[$\bullet$] \textit{\textbf{Game}}  $G_{\sss  1}$: This game is similar to  $G_{\sss  0}$, except that the output of the $\prf$ is replaced by an output of  a uniformly random function $f$, i.e., when the simulator in Figure \ref{fig::PRF-SIM} is used. Since the output of $f$ (in the simulator) and $\prf$ are indistinguishable, except with a negligible probability, we will have: 
%
\begin{equation}\label{eq::game_1}
|Pr[S_{\sss  1}]-Pr[S_{\sss  0}]|\leq (q_{\sss s}+q_{\sss p})Adv^{\sss\prf}(\adv)
\end{equation}

that captures both send and execute queries. We highlight that as we use a standard $\prf$, the probability of finding a collision is $0$. 



\item[$\bullet$] \textit{\textbf{Game}}  $G_{\sss  2}$: This game is the same as $G_{\sss  1}$, with the difference that we simulate the authenticated encryption scheme (i.e., $Enc$ and $Dec$ algorithms). We replace the output of $Enc$ with a uniformly random value picked from the encryption scheme's range. The adversary has access to the encryption and decryption oracles. Since we treat the encryption scheme as a black box, the two games are distinguishable except with a negligible probability; this we will have: 
 %
\begin{equation}\label{eq::game_2}
|Pr[S_{\sss  2}]-Pr[S_{\sss  1}]| \leq (q_{\sss s}+q_{\sss p}) Adv^{\sss Enc}(\adv)
\end{equation}

The above also captures both the send and execute queries. Since we have used a standard encryption scheme, the probability of finding a collision (e.g., two ciphertexts result in the same plaintext or two plaintexts result in the same ciphertext) is $0$, as the scheme is bijective. 



\item[$\bullet$] \textit{\textbf{Game}}  $G_{\sss  3}$: This game is the same as $G_{\sss  2}$, with the difference that we simulate the verification of a transaction, i.e., via predicate $\phi$ defined in Section \ref{sec::model}. Moreover, we simulate all parties' instances via defining simulators for  \send, \execute, \reveal, and \test\ queries. We present the simulators for user's and server's \send\ queries in Figures \ref{fig::Send-sim-to-client} and \ref{fig::Send-sim-to-server} respectively. Also, we present the simulators for the rest of the queries in Figure \ref{fig::sim-for-exe-rev-test}. By definition,  $\phi$ is a deterministic function, given the transaction $t^{\sss  U}$ and policy $\pi$, it always returns the same output as the user's device does when verifying $t^{\sss  U}$ in the previous game.  Therefore, both $\phi$ and the user's device would output identical values, given pair $(t^{\sss  U}, \pi)$, meaning that their outputs are indistinguishable in both games.  Given the above argument, we conclude that:  
%
\begin{equation}\label{eq::game_3}
Pr[S_{\sss  3}]-Pr[S_{\sss  2}]=0
\end{equation}
%
\item[$\bullet$] \textit{\textbf{Game}}  $G_{\sss  4}$: This game is the same as $G_{\sss  3}$, with the difference that when the adversary manages to use the correct encryption key and encrypts (or decrypts) a message itself, then the simulation aborts. Therefore, we have: 
%
\begin{equation*}
|Pr[S_{\sss  4}]-Pr[S_{\sss  3}]|\leq Pr[Enc_{\sss  4}]
\end{equation*}


We know that the key has been picked uniformly at random and is of length $\lambda$ bits (recall that the outputs of $\prf$ have been replaced with truly random values in $G_{\sss  1}$). Therefore:
%
 $$Pr[Enc_{\sss  4}]=\frac{4(q_{\sss  s}+q_{\sss  p})}{2^{\sss \lambda}}$$ and 
 %
 \begin{equation}\label{eq::game_4}
 |Pr[S_{\sss  4}]-Pr[S_{\sss  3}]|\leq \frac{4(q_{\sss  s}+q_{\sss  p})}{2^{\sss \lambda}}
 \end{equation}
%

\item[$\bullet$] \textit{\textbf{Game}}  $G_{\sss  5}$: In this game, we modify the simulator such that it would abort if
the adversary correctly guesses the authenticator. Therefore, we modify the way the server responds to  query {\send($S^{\sss  j},   \bar{\VC{\mathit{response}}}$)} as follows: 

\begin{enumerate}
%
\item computes $\VS{\mathit{expected}} \gets \prf_{\sss\bar{\VS{\keyt_{\sss3}}}}(\ddot{\VS{\nonce}}  \| \VS{\trans} \| \VS{\verifier}|| 1)$.
%
\item checks if $\bar{\VC{\mathit{response}}}=\VS{\mathit{expected}}$. It proceeds to the next step if the equation holds. 
%
\item\label{Game::check-sim} checks if $\Big((U_{\sss  ID},  \text{enrolment}), (U_{\sss  ID},  \text{authentication}), (\bar M, \bar t),$ $ (\bar M',$ $ \bar t'),$ $ (\ddot M', $ $\ddot t'), $ $(\hat M', \hat t'), \bar{\VC{\mathit{response}}}\Big)\in \vv L$.

\item\label{Game::prf} checks if $ \bar{\VC{\mathit{response}}}\in L_{\sss \adv}$. 
%
\item if both checks in steps \ref{Game::check-sim} and \ref{Game::prf} fail, then it rejects authenticator $\bar{\VC{\mathit{response}}}$ and terminates without accepting the key. Otherwise, it accepts the key. 
%
\end{enumerate}
This game ensures that if the message (\ie, the authenticator) does not come from the simulator or the adversary (which decrypted $\ddot M',\ddot t', \hat M'$, and  $\hat t'$, then correctly computed a valid authenticator by querying $\update$ and $\prf_{\sss  k'}$) then it aborts. So, games $G_{\sss  4}$ and $G_{\sss  5}$ are indistinguishable unless the server rejects a valid authenticator. However, this means the adversary has correctly guessed the output of $\prf$. Thus, 
%
\begin{equation}\label{eq::game_5}
|Pr[S_{\sss  5}]-Pr[S_{\sss  4}]|\leq \frac{q_{\sss  s}}{2^{\sss \lambda}}
\end{equation}



\item[$\bullet$] \textit{\textbf{Game}}  $G_{\sss  6}$: 
 In this game, we modify the simulator in a way that it would abort if $\adv$ decrypts $ (\ddot M', \ddot t', \hat M', \hat t')$ and uses the result to generate and send a valid authenticator to the server. To do so, we modify the way the server responds to  query {\send($S^{\sss  j},   \bar{\VC{\mathit{response}}}$)}, as follows: 

\begin{enumerate}
%
\item computes $\VS{\mathit{expected}} \gets \prf_{\sss\bar{\VS{\keyt_{\sss3}}}}(\ddot{\VS{\nonce}}  \| \VS{\trans} \| \VS{\verifier}||1)$.
%
\item checks if $\bar{\VC{\mathit{response}}}=\VS{\mathit{expected}}$. It proceeds to the next step if the equation holds. 
%
\item\label{Game::check-sim-} checks if $\Big((U_{\sss  ID},  \text{enrolment}), (U_{\sss  ID},  \text{authentication}), (\bar M, \bar t),$ $ (\bar M',$ $ \bar t'),$ $ (\ddot M', $ $\ddot t'), $ $(\hat M', \hat t'), \bar{\VC{\mathit{response}}}\Big)\in \vv L$. If this check fails, then it rejects authenticator  $\bar{\VC{\mathit{response}}}$ and terminates, without accepting any key.

\item\label{Game::prf-} checks if $(\ddot{\VS{\nonce}}  \| \VS{\trans} \| *,\bar{\VC{\mathit{response}}})\in L_{\sss \adv}$.  
%
\item aborts, if the above check (in step \ref{Game::prf-}) passes.
%
\end{enumerate}

The above modification ensures that all valid authenticators are sent by the simulator. Let $\hat {Auth}_{\sss  6}$ be the event that the check in step \ref{Game::prf-} passes. Games $G_{\sss  5}$ and $G_{\sss  6}$ are indistinguishable unless $\hat {Auth}_{\sss  6}$ occurs. Hence,
%
$$|Pr[S_{\sss  6}]-Pr[S_{\sss  5}]|\leq Pr[\hat {Auth}_{\sss  6}]$$


We know that $\hat {Auth}_{\sss  6}$ occurs with probability $ \frac{q_{\sss  s}}{2^{\sss \lambda}}$ when the query $q=\ddot{\VS{\nonce}}  \| \VS{\trans} \| *$ to $\prf$ results in $\bar{\VC{\mathit{response}}}$. Thus, 
%
\begin{equation}\label{eq::game_6}
|Pr[S_{\sss  6}]-Pr[S_{\sss  5}]|\leq \frac{q_{\sss  s}}{2^{\sss \lambda}}
\end{equation}

\item[$\bullet$] \textit{\textbf{Game}}  $G_{\sss  7}$: In this game, we modify the simulator such that it would abort if the adversary comes up with the authenticator and session key without decrypting  $ (\ddot M', \ddot t', \hat M', \hat t')$.  Therefore, we modify  the way the user's device processes query \send($U^{\sss  i}, (\ddot M', \ddot t'), (\hat M', \hat t')$) as follows.
 %
 \begin{enumerate}
 \item compute $\bar{\VC{\mathit{response}}} \gets \prf_{\sss  \ddot t'}( \ddot M'|| 1)$.
 \item compute $\bar {sk}^{\sss U}\gets \prf_{\sss  \ddot t'}( \ddot M'|| 2)$.
 \end{enumerate}

We also amend the way the server compiles query \send($S^{\sss  j},   \bar{\VC{\mathit{response}}}$) as follows. 
%
\begin{enumerate}[label=\alph*]
%
\item\label{Game::prf--}  checks if  $(\ddot M'||1, \bar{\VC{\mathit{response}}})\in L_{\sss \adv}$ or  $(\ddot M'||2, \bar {sk}^{\sss U})\in L_{\sss \adv}$. 
%
\item aborts, if either of the above checks (in step \ref{Game::prf--}) passes.
%
\end{enumerate}



 Let $\hat {Auth}_{\sss  7}$ be the event that the check in step \ref{Game::prf--} passes. Games $G_{\sss  6}$ and $G_{\sss 7}$ are indistinguishable unless $\hat {Auth}_{\sss  7}$ occurs. Therefore,
%
$$|Pr[S_{\sss  7}]-Pr[S_{\sss  6}]|\leq Pr[\hat {Auth}_{\sss  7}]$$


Event $\hat {Auth}_{\sss  7}$ occurs with probability $ \frac{2q_{\sss  s}}{2^{\sss \lambda}}$ when the query $q=\ddot M'||1$ to $\prf$ results in $\bar{\VC{\mathit{response}}}$ or $q=\ddot M'||2$ to $\prf$ results in $\bar {sk}^{\sss U}$. Thus, 
%
\begin{equation}\label{eq::game_7}
|Pr[S_{\sss  7}]-Pr[S_{\sss  6}]|\leq \frac{2q_{\sss  s}}{2^{\sss \lambda}}
\end{equation}

Moreover, the session key and authenticator are random values, as they are the outputs of $\prf$ whose secrete key is not known. Therefore, $Pr[S_{\sss  7}]=\frac{1}{2}$. 



By summing up all the above relations \ref{eq::game_1}-\ref{eq::game_7}, we would have 
%
\begin{equation}\label{equ::sum}
|Pr[S_{\sss  7}]-Pr[S_{\sss  0}]|\leq (q_{\sss s}+q_{\sss p})\Big(Adv^{\sss\prf}(\adv)+Adv^{\sss Enc}(\adv)\Big)+\frac{4(q_{\sss  s}+q_{\sss  p})}{2^{\sss \lambda}}+\frac{4q_{\sss  s}}{2^{\sss \lambda}}
\end{equation}

By combining Equations \ref{eq::adv-ake-} and \ref{equ::sum}, we would have: 
%
\begin{equation*} 
Adv_{\sss  \psi}^{\sss  ss}(\A) \leq 2(q_{\sss s}+q_{\sss p})\Big(Adv^{\sss\prf}(\adv)+Adv^{\sss Enc}(\adv)\Big)+\frac{8(2q_{\sss  s}+q_{\sss  p})}{2^{\sss \lambda}}
\end{equation*}

This completes the proof. 


%\
%
%\item[$\bullet$] \textit{\textbf{Game}}  $G_2$: In this game, we simulate the $\prf$ and all the instances as the real players would do for the send, execute, reveal, and test queries. 

 
%Since the pseudorandom function is deterministic and outputs the same value if queried twice on the same inputs, without loss of generality we assume a distinguisher never queries the pseudorandom oracle twice on the same inputs.

%%%%
\end{itemize}

%!TEX root = main.tex


%%%%%%%%%%%%%%%%%%%%%%%%%%%%%%%%%%%%%%
\begin{figure}[!hpt]
%\setlength{\fboxsep}{1pt}
\begin{center}
    \begin{tcolorbox}[enhanced,width=3.3in, left=.1cm,
    drop fuzzy shadow southwest,
    colframe=black,colback=white]
{\small{
\doubleunderline{Pseudorandom function}\\
%The simulator maintains a history of already answered queries and acts as follows. 

The simulator upon receiving query $(\prf,q)$ acts as follows. 

\begin{itemize}
%\item  if record $(q,r)$ is in list $L_{\prf}$, then output $r$. 
 \item picks a function $f$, i.e., $f\stackrel{\sss \$}\leftarrow Func$, where $Func$ is the set of all functions mapping $|q|$-bit strings to $|q|$-bit strings. 
 \item adds record $(q,f(r))$ to list $L_{\sss\adv}$ and then outputs $f(r)$.  
 \end{itemize}
 
 
}}
\end{tcolorbox}
\end{center}
%\vspace{-4mm}
\caption{Pseudorandom function's simulator.} 
\label{fig::PRF-SIM}
\end{figure}





%%%%%%%%%%%%%%%%%%%%%%%%%%%%%%%%%%%%%%%%
\begin{figure}[H]
\setlength{\fboxsep}{1pt}
\begin{center}
    \begin{tcolorbox}[enhanced,width=3.3in,left=0.1cm, 
    drop fuzzy shadow southwest,
    colframe=black,colback=white]
{\small{
%\underline{Pseudorandom function}\\
%%The simulator maintains a history of already answered queries and acts as follows. 
%The simulator upon receiving query $(\prf,q)$ acts as follows. 
%
%\begin{itemize}
%%\item  if record $(q,r)$ is in list $L_{\prf}$, then output $r$. 
% \item picks a function $f$, i.e., $f\stackrel{\sss \$}\leftarrow Func$, where $Func$ is the set of all functions mapping $|q|$-bit strings to $|q|$-bit strings. 
% \item  output $f(r)$. %Next, add record $(q,r)$ to $L_{\prf}$. 
% \end{itemize}
% 
% 
% \noindent\rule{5.1in}{1pt}
 
  \doubleunderline{\send($U^{\sss i}, .$)}\\
 
 This query is dealt with as below: 
 
 \begin{itemize}[leftmargin=.4cm]
 %
 \item if the user's instance is not in the ``expecting'' state and it receives query  \send($U^{\sss i}, \text{start}, \text{phase}$), where $\text{phase}\in \{\text {enrolment, authentication}\}$ then it: 
 %
  \begin{enumerate}
  %
  \item generates pair $(U_{\sss ID},  \text{phase})$. 
  %
  \item responds to the query with $(U_{\sss ID},  \text{phase})$. 
  %
  \item sets the user's instance state to expecting.
  %
  \end{enumerate}
 % 
 \item if the user's instance state is in expecting, then: 
 %
 \begin{itemize}[leftmargin=.4cm]
 %
 \item upon receiving \underline{\send($U^{\sss i}, (\bar M, \bar t)$)}, it:
 % 
  \begin{enumerate}
  %
  \item authenticates and decrypts the ciphertext $\bar M$ as $(p, b) \gets \mathsf{Dec}_{\sss\VC{k}}(\bar M, \bar t)$. If the authentication fails (\ie, $b\neq 1$), it halts.  
  %
  \item extracts $(\VM{\nonce}, \VM{\counter})$ from plaintext $p$ and checks if $\VM{\counter} > \VC{\counter} $. If the check fails, it halts. 
  %
  \item generates $\VC{\verifier}$ using $\VC{\salt}$ and $\VC{\pin}$ as follows $\VC{\verifier} \gets \prf_{\VC{\salt}}(\VC{\pin})$. Then, it updates its state as follows: $\forall i, 1\leq i\leq \VM{\counter}- \VC{\counter}: \text{\ (a)\ } \VC{\counter} \gets \VC{\counter} + 1\text{\ and (b) \ } (k,  \VC{\state}) \gets \update(\VC{\state}, \VC{\counter})$. 
  %
  \item encrypts $p'=\VM{\nonce}||\VC{\verifier}$ using key $k$  as follows: $(\bar M', \bar t') \gets \enc_{\sss k}(p')$, which results in a ciphertext $\bar M'$ and tag $\bar t'$. 
  %
  \item responds to the query with $(\bar M', \bar t')$. It sets the user's instance state to ``not expecting''.  
  %
  \end{enumerate}
 %
 \item  upon receiving \underline{\send($U^{\sss i}, (\ddot M', \ddot t'), (\hat M', \hat t')$)}, it: 
 %
 \begin{enumerate}
 %
 \item   authenticates and decrypts the ciphertext $\hat M'$ as follows: $(p', b') \gets \mathsf{Dec}_{\sss\VC{k}}(\hat M', \hat t')$. If the authentication fails (i.e., $b'\neq 1$), it halts.  
 %
  \item extracts $(\tmp_{\sss\VM{\counter}}, \VM{\counter})$ from $p'$ and checks if $\tmp_{\sss\VM{\counter}} > \VC{\counter} $. If the check fails, it halts. 
 %
 \item updates its state as follows: $\forall i, 1\leq i\leq \tmp_{\sss\VM{\counter}}- \VC{\counter}: \text{\ (a)\ } \VC{\counter} \gets \VC{\counter} + 1\text{\ and (b) \ } (k,  \VC{\state}) \gets \update(\VC{\state}, \VC{\counter})$. 
 %
 \item  authenticates and decrypts $\ddot M'$ as follows: $(p, b) \gets \mathsf{Dec}_{\sss k}(\ddot M', \ddot t')$. If the authentication fails (\ie, $b\neq 1$), it halts.
 %
 \item    extracts $(\VM{\nonce}, \VM{\trans})$ from plaintext $p$.  
 %
 \item runs the predicate, $y \leftarrow\phi(\VM{\trans}, \gamma)$. If $y=0$, it halts. 
 %
\item generates $\VC{\verifier}$ using $\VC{\salt}$ and $\VC{\pin}$ as follows, $\VC{\verifier} \gets \prf_{\sss\VC{\salt}}(\VC{\pin})$.
 %
 \item updates its state one more time as follows, $(k',  \VC{\state}) \gets \update(\VC{\state}, \VM{\counter})$.
 %
 \item computes  the authenticator: $\bar{\VC{\mathit{response}}} \gets \prf_{\sss k'}( \VM{\nonce} || \VM{\trans} || \VC{\verifier}|| 1)$ and session key:  $\bar {sk}^{\sss U} \gets \prf_{\sss k'}( \VM{\nonce} || \VM{\trans} || \VC{\verifier}|| 2)$.
 %
 \item responds the send query with  $\bar{\VC{\mathit{response}}}$. It makes the user's instance accept the key and then terminates the instance. 
 %
 \end{enumerate}
 %
 \end{itemize}
 %
 \end{itemize}
 %
}}
To keep track of all the exchanged messages, it stores the above incoming and going messages in vector $\vv L$. So, we have $\Big((U_{\sss ID},  \text{enrolment}),$ $ (U_{\sss ID},  \text{authentication}), $ $(\bar M, \bar t), $ $(\bar M', \bar t'),$ $(\ddot M', \ddot t'), $ $(\hat M', \hat t'), $ $\bar{\VC{\mathit{response}}}\Big)\in \vv L$.
\end{tcolorbox}
\end{center}
\vspace{-3mm}
\caption{Simulators for \send\ query to  a user's instance.} 
\label{fig::Send-sim-to-client}
\end{figure}
%%%%%%%%%%%%%%%%%%%%%%%%%%%%%%%%%



%%%%%%%%%%%%%%%%%%%%%%%%%%%%%%%%%%%%%%%%
\begin{figure}[H]
\setlength{\fboxsep}{1.2pt}
\begin{center}
    \begin{tcolorbox}[enhanced,width=3.3in, left=0.1cm,
    drop fuzzy shadow southwest,
    colframe=black,colback=white]
{\small{

 \doubleunderline{\send($S^{\sss j}, .$)}\\
 
 This query is dealt with as below: 
 
 \begin{itemize}[leftmargin=.4cm]
 %
 \item upon receiving  \underline{\send($S^{\sss j}, (U_{\sss ID},  \text{enrolment})$)}, it:
 %
 \begin{enumerate}
 %
 \item  increments its counter as $\VS{\counter} \gets \VS{\counter} + 1$, updates its state as $\bar{\VS{\keyt}_{\sss 1}}, \VS{\state} \gets \update(\VS{\state}, \VS{\counter})$, picks a random value $\bar{\VS{\nonce}} \stackrel{\sss\$}\leftarrow \bin^{\sss\secpar}$, and generates ciphertext and tag   $(\bar M,\bar t) \gets \enc_{\sss\VS{k}}(\bar{\VS{\nonce}}|| \VS{\counter})$. 
 
 \item responds to the query with $(\bar M,\bar t)$. The state of the server instance is set to ``expecting''.
 %
 %\item upon receiving query  \send($S^{j}, $), 
 %
 \end{enumerate}
 %
 \item upon receiving  \underline{\send($S^{\sss j}, (\bar M',  \bar t'$))}, it:
 %
 \begin{enumerate}
 %
 \item authenticates and decrypts the ciphertext $\bar M'$ as $(p', b') \gets \mathsf{Dec}_{\sss\bar{\VS{\keyt}_{\sss 1}}}(\bar M', \bar t')$. If the authentication fails (i.e., $b'\neq 1$), it halts. 
 %
\item    extracts $(\VM{\nonce}, \VM{\verifier})$ from plaintext $p'$. 

It sets $\VS{\verifier}\gets\VM{\verifier}$ and also  checks if $\VM{\nonce}=\bar{\VS{\nonce}}$. If the equation does not hold, it halts. The state of the server instance is set to expecting.
 %
 \end{enumerate}
 %
 \item upon receiving  \underline{\send($S^{\sss j}, (U_{\sss ID},  \text{authentication})$)}, it:
 %
 \begin{enumerate}
 %
  \item  increments its counter  $\VS{\counter} \gets \VS{\counter} + 1$, updates its state  $\bar{\VS{\keyt}_{\sss 2}}, \VS{\state} \gets \update(\VS{\state}, \VS{\counter})$,  temporarily stores this counter $\tmp_{\sss\VS{\counter}} \gets \VS{\counter}$, increments the counter again $\VS{\counter} \gets \VS{\counter} + 1$, updates its state again $\bar{\VS{\keyt}_{\sss 3}}, \VS{\state} \gets \update(\VS{\state}, \VS{\counter})$, and picks a random value $\ddot{\VS{\nonce}} \stackrel{\sss\$}\leftarrow \bin^{\sss\secpar}$.
  %
  \item generates two pairs of ciphertext and tag as follows, $(\ddot M', \ddot t')\leftarrow \enc_{\sss\bar{\VS{\keyt_{\sss 2}}}}(\ddot{\VS{\nonce}} || \VS{\trans})$ and  $(\hat M', \hat t')\leftarrow \enc_{\sss\VS{k}}(\tmp_{\sss\VS{\counter}} || \VS{\counter})$. 
  %
  \item responds to the query with $(\ddot M', \ddot t')$, $(\hat M', \hat t')$. The state of the server instance is set to expecting.
 %
 \end{enumerate}
 %
 \item upon receiving  \underline{\send($S^{\sss j},   \bar{\VC{\mathit{response}}}$)}, it:
 %
 \begin{enumerate}
 %
 \item computes $\VS{\mathit{expected}} \gets \prf_{\sss\bar{\VS{\keyt_{\sss3}}}}(\ddot{\VS{\nonce}}  \| \VS{\trans} \| \VS{\verifier}|| 1)$. It checks whether  $ \bar{\VC{\mathit{response}}}=\VS{\mathit{expected}}$. If the equality does not hold, the server instance terminates without accepting any session key. 
 %
  \item generates the session key $\bar {sk}^{\sss S} \gets \prf_{\sss\bar{\VS{\keyt_{\sss3}}}}(\ddot{\VS{\nonce}}  \| \VS{\trans} \| \VS{\verifier}|| 2)$. 
 %
It accepts the key and terminates.
%
  \end{enumerate}
  %
 \end{itemize} 
 %
}}
\end{tcolorbox}
\end{center}
\caption{Simulators for \send\ query to  a server's instance.} 
\label{fig::Send-sim-to-server}
\end{figure}
%%%%%%%%%%%%%%%%%%%%%%%%%%%%%%%%

%%%%%%%%%%%%%%%%%%%%%%%%%%%%%%%%%%%%%%%%
\begin{figure}[H]
\setlength{\fboxsep}{1pt}
 \vspace{-3mm}
\begin{center}
    \begin{tcolorbox}[enhanced,width=3.3in, height=105mm, left=0.1cm, 
    drop fuzzy shadow southwest,
    colframe=black,colback=white]
     \vspace{-3mm}
{\small{

 \doubleunderline{\execute($U^{\sss i}, S^{\sss j}$)}\\
 
 This query is dealt with as below: 
 %
 \begin{enumerate} 
 %
 \item $(U_{\sss ID},  \text{enrolment})\gets$\send($U^{\sss i}, \text{start}, \text{enrolment}$).%, where $\text{phase}\in \{\text {enrollment, authentication}\}$. 
 %
\item  $(\bar M,\bar t)\gets$\send($S^{\sss j}, (U_{\sss ID},  \text{enrolment})$).
  %
\item $(\bar M', \bar t')\gets$\send($U^{\sss i}, (\bar M, \bar t)$).
  %
\item $(U_{\sss ID},  \text{authentication})\gets$\send($U^{\sss i}, \text{start}, \text{authentication}$).
  %
\item $(\ddot M', \ddot t', \hat M', \hat t')\gets$\send($S^{\sss j}, (U_{\sss ID},  \text{authentication})$).
  %
\item $\bar{\VC{\mathit{response}}}\gets$\send($U^{\sss i}, (\ddot M', \ddot t'), (\hat M', \hat t')$).
  %
  \item outputs the following transcript: $[(U_{\sss ID},  \text{enrolment}), (\bar M, \bar t), (\bar M', \bar t'),(U_{\sss ID},  $ $\text{authentication}), $ $(\ddot M', \ddot t'), (\hat M', \hat t'), \bar{\VC{\mathit{response}}}]$.
  %
 \end{enumerate} 
 %
  \noindent\rule{3.2in}{1pt}
  %
  \doubleunderline{\reveal($I$)}\\
  
This query is processed as follows.
  
\ $\bullet$ returns session key $\bar {sk}^{\sss I}$ (computed by $I\in\{U,S \}$), if $I$ has already accepted the key.
  
   %
  \noindent\rule{3.2in}{1pt}
  %
    \doubleunderline{\test($I$)}\\
    
  This query is processed as below.  
  
 \begin{enumerate} 
 %
 \item $sk\gets$\reveal($I$).
 %
 \item $b\stackrel{\sss\$}\gets\{0,1\}$.
 %
 \item sets $v$ as follows:  \begin{equation*}
   v= 
\begin{cases}
    sk,              &\text{if } b= 1\\
   r \stackrel{\sss\$}\gets\{0,1\}^{\sss \iota}, & \text{otherwise }\\
\end{cases}
\end{equation*}
 %
 \item returns $v$.
 \end{enumerate}
}}
\end{tcolorbox}
\end{center}
 %\vspace{-3mm}
\caption{Simulators for \execute, \reveal, and \test\ queries. } 
\label{fig::sim-for-exe-rev-test}
% \vspace{-3mm}
\end{figure}



\subsection{Authentication}

In this section, we prove the protocol's authentication. We begin with the case where the adversary $\adv$ has access to the traffic between the two parties and wants to impersonate the user, $U$; we denote such a case with $\bar {aut}$. The Authentication proof relies on the semantic security proof (and games) we presented in Section \ref{sec::semSec-proof}.  Now, we outline the proof.  By definition, it holds that:
%
 \begin{equation}\label{eq::adv-ake}
Adv_{\sss \psi}^{\sss \bar {aut}}(\A)=Pr[Auth_{\sss  0}]
\end{equation}


Also, we can extend Equation \ref{eq::game_1} to:
%
\begin{equation*}
|Pr[Auth_{\sss  1}]-Pr[Auth_{\sss  0}]| \leq (q_{\sss s}+q_{\sss p})Adv^{\sss\prf}(\adv),
\end{equation*}
%
 because the only difference between the two games (i.e., $G_{\sss  0}$ and $G_{\sss  1}$) is that the output of the $\prf$ is replaced with an output of a uniformly random function $f$. 
 
 
 Furthermore, we can extend Equations \ref{eq::game_1}-\ref{eq::game_7} as follows: 
 
 
 \begin{equation*}
 \begin{split}
  &  |Pr[Auth_{\sss  1}]-Pr[Auth_{\sss  0}]| \leq (q_{\sss s}+q_{\sss p}) Adv^{\sss \prf}(\adv) \\ 
 &  |Pr[Auth_{\sss  2}]-Pr[Auth_{\sss 1}]| \leq (q_{\sss s}+q_{\sss p}) Adv^{\sss Enc}(\adv) \\ 
 & Pr[Auth_{\sss  3}]-Pr[Auth_{\sss  2}]=0\\
 & |Pr[Auth_{\sss  4}]-Pr[Auth_{\sss  3}]|\leq \frac{4(q_{\sss  s}+q_{\sss  p})}{2^{\sss \lambda}}\\
 & |Pr[Auth_{\sss  5}]-Pr[Auth_{\sss  4}]|\leq \frac{q_{\sss  s}}{2^{\sss \lambda}}\\
 & |Pr[Auth_{\sss  6}]-Pr[Auth_{\sss  5}]|\leq \frac{q_{\sss  s}}{2^{\sss \lambda}}\\
 & |Pr[Auth_{\sss  7}]-Pr[Auth_{\sss  6}]|\leq \frac{2q_{\sss  s}}{2^{\sss \lambda}}
 %
\end{split}
\end{equation*}

 Moreover, since the authenticator is a random value in $G_{\sss 7}$, it holds that $Pr[Auth_{\sss 7}]=\frac{q_{\sss  s}}{2^{\sss \lambda}}$. We conclude the proof, by summing up the above relations and combining with Equation \ref{eq::adv-ake}: 
 %
  \begin{equation}
 Adv_{\sss \psi}^{\sss \bar {aut}}(\A) = Pr[Auth_{\sss  0}] \leq (q_{\sss s} + q_{\sss p})\Big(Adv^{\sss\prf}(\adv)+Adv^{\sss Enc}(\adv)\Big)+\frac{9q_{\sss  s}+4q_{\sss  p}}{2^{\sss \lambda}}%+ \frac{4q_p}{2^{\lambda}}
 \end{equation}
 
 
Next, we proceed to the case where the adversary is given further access to the PIN, i.e., $\adv$ can also send query  \corrupt($C, 1$). We argue that given such an extra capability does not affect the adversary's advantage and the above analysis (as the protocol and its analysis have relied on the security of the CCA-secure symmetric encryption and $\prf$). Now move on to the case where $\adv$ (a) is given all the parameters stored in the hardware token, and (b) has access to all the traffic between the two parties, i.e., $\adv$ can also send query  $\bar{\text{Cpt}_{\sss  2}} =$ \corrupt$(C, 2)$. We argue that in this case, the upper bound of $\adv$'s advantage will be changed as follows:  $Adv_{\sss\psi, \bar{\text{Cpt}_{\sss  2}}}^{\sss  aut}(\A)\leq  \cfrac{q_{\sss  s}}{N}$. The reason for such a big change is that in this case, $\adv$ has all secret parameters, except the PIN and verifier $\VC{\verifier}$. \footnote{The case where $\adv$ has the additional capability to send query \corrupt($U, 2$) was never discussed and analysed in \cite{BressonCP03}. However, we noticed that $\adv$ in that scheme would have the same upper bound advantage as $\adv$ in our scheme does.} Thus, when we take the forward security into account, the advantage of the adversary (due to the union bound)  is as follows: 
%
  \begin{equation*}
 Adv_{\sss \psi}^{\sss  aut}(\A)  \leq (q_{\sss s} + q_{\sss p})\Big(Adv^{\sss\prf}(\adv)+Adv^{\sss Enc}(\adv)\Big)+\frac{9q_{\sss s}+4q_{\sss  p}}{2^{\sss \lambda}}+  \cfrac{q_{\sss  s}}{N}
 \end{equation*}


\subsection{PIN's Privacy Against A Corrupt Server} 

In the case where the adversary (i) has access to the parties' traffic and (ii) can make query \corrupt($S, 1$), to extract all parameters of the server, then the probability that the adversary can find the valid PIN depends on the probability of finding the correct PIN and finding $U$'s correct key of $\prf$; therefore, the probability is at most $\cfrac{q_{\sss  p}}{2^{\sss \lambda}N}$. 


