% This is samplepaper.tex, a sample chapter demonstrating the
% LLNCS macro package for Springer Computer Science proceedings;
% Version 2.20 of 2017/10/04
%
\documentclass[runningheads]{llncs}

%
\usepackage[T1]{fontenc}
\usepackage{multicol}
\usepackage{multirow}
\usepackage[table]{xcolor}
\usepackage{graphicx}
\usepackage{longtable}
\usepackage{color}
\usepackage{prettyref}
\newrefformat{fig}{Figure~\ref{#1}}
\newrefformat{tab}{Table~\ref{#1}}
\newrefformat{app}{Appendix~\ref{#1}}
\usepackage{booktabs}
\usepackage{xspace}
%\usepackage[disable]{todonotes}
\usepackage[disable]{todonotes}
\raggedbottom% to remove huge space
\usepackage{url}
\def\doubleunderline#1{\underline{\underline{#1}}}

	

%%%%%%%%
\usepackage{framed}
\usepackage{esvect}
\usepackage{tikz}
\usepackage{blkarray}% http://ctan.org/pkg/blkarra
\usepackage{mathtools}
\usepackage{amsmath}
\usepackage{bm}
\usepackage{adjustbox}
\usepackage{boxedminipage}
\usepackage{blindtext}
\usepackage{multicol}
%%%%%%%%%
\usepackage{pgfplots, pgfplotstable}

\usepackage{hyperref}
\usepackage{float}
\usepackage{boxedminipage}
\usepackage{enumitem}
%\usepackage{makeidx}  % allows for indexgeneration
%\usepackage{amsfonts,amsmath,amssymb,graphicx,setspace,tipx}
\usepackage{amsmath}
\usepackage{float}
\usepackage[ruled,linesnumbered]{algorithm2e}
\usepackage{subfig}
\usepackage{graphicx}
\usepackage{framed}
\usepackage{esvect}
\usepackage{tikz}
\usepackage{latexsym}

%\usepackage{pgfplots}
\usepackage{pgfplots, pgfplotstable}
\usepackage{blkarray}% http://ctan.org/pkg/blkarra
\usepackage{mathtools}
\usepackage{amsmath}
\usepackage{bm}
\usepackage{adjustbox}
\usepackage{blindtext}
\usepackage{multicol}
\usepackage{prettyref}
\usepackage{multirow}
\usepackage{adjustbox}
\usepackage{blindtext}
\usepackage{multicol}
\usepackage{tablefootnote}
\usepackage{colortbl}
\usepackage[skins]{tcolorbox}
%%%%%%%%%%%
\newtcolorbox{mybox}[2][]{%
  attach boxed title to top center
               = {yshift=-11pt},
               %width=85mm,%
                  %height=52mm,
  %colback      = black,
  colframe     =black,
  %fonttitle    = \bfseries,
  colbacktitle = black,
  title        = #2,#1,
  enhanced,
}




% Add a period to the end of an abbreviation unless there's one
% already, then \xspace.
\makeatletter
\DeclareRobustCommand\onedot{\futurelet\@let@token\@onedot}
\def\@onedot{\ifx\@let@token.\else.\null\fi\xspace}

\def\eg{\emph{e.g}\onedot} \def\Eg{\emph{E.g}\onedot}
\def\ie{\emph{i.e}\onedot} \def\Ie{\emph{I.e}\onedot}
\def\cf{\emph{c.f}\onedot} \def\Cf{\emph{C.f}\onedot}
\def\etc{\emph{etc}\onedot} \def\vs{\emph{vs}\onedot}
\def\wrt{w.r.t\onedot} \def\dof{d.o.f\onedot}
\def\etal{\emph{et al}\onedot}
\makeatother

%\usepackage{amsmath}
% Used for displaying a sample figure. If possible, figure files should
% be included in EPS format.
%
% If you use the hyperref package, please uncomment the following line
% to display URLs in blue roman font according to Springer's eBook style:
\usepackage{hyperref}
\newcommand{\sss}{\scriptscriptstyle}
\renewcommand\UrlFont{\color{blue}\rmfamily}
\usepackage[primitives,operators,sets,keys,ff,lambda,adversary]{cryptocode}
\newcommand{\nonce}{\ensuremath{{N}}}
\newcommand{\keyt}{\ensuremath{{kt}}}
\newcommand{\counter}{\ensuremath{{ct}}}
\newcommand{\pin}{\ensuremath{\mathsf{PIN}}}
\newcommand{\salt}{\ensuremath{{sa}}}
\newcommand{\trans}{\ensuremath{{t}}}
\newcommand{\extract}{\ensuremath{\mathsf{Extract}}}
\renewcommand{\verifier}{\ensuremath{{v}}}
\renewcommand{\state}{\ensuremath{{st}}}
\newcommand{\update}{\ensuremath{\mathsf{Update}}}
\newcommand{\VC}[1]{\ensuremath{#1^{\sss U}}}
\newcommand{\VS}[1]{\ensuremath{#1^{\sss S}}}
\newcommand{\VM}[1]{\ensuremath{#1^{\sss M}}}
\newcommand{\parse}{\ensuremath{\mathsf{Parse}}}
\newcommand{\tmp}{\ensuremath{{tmp}}}
\newcommand{\iif}{\ensuremath{\text{If\ }}}
\newcommand{\ssample}{\stackrel{\sss\$}\leftarrow}
\newcommand{\forEach}{\ensuremath{\text{For each\ }}}
\newcommand{\tto}{\ensuremath{\text{to\ }}}

\newcommand{\execute}{$\mathtt{Execute}$}
\newcommand{\send}{$\mathtt{Send}$}
\newcommand{\test}{$\mathtt{Test}$}
\newcommand{\reveal}{$\mathtt{Reveal}$}
\newcommand{\corrupt}{$\mathtt{Corrupt}$}
\newcommand{\A}{\mathcal{A}}
\newcommand{\seckey}{\ensuremath{\mathit{sk}}}
\newcommand{\SID}{\ensuremath{\mathit{SID}}}
\newcommand{\PID}{\ensuremath{\mathit{PID}}}

%% Workaround for incompatibility between LLNCS and titlesec
%% Save the class definition of \subparagraph
\let\llncssubparagraph\subparagraph
%% Provide a definition to \subparagraph to keep titlesec happy
\let\subparagraph\paragraph
%% Load titlesec
\usepackage[compact]{titlesec}
%\titlespacing*{\section}{0pt}{1.1\baselineskip}{\baselineskip}
%% Revert \subparagraph to the llncs definition
\let\subparagraph\llncssubparagraph

\begin{document}

\title{A Forward-secure Efficient\\ Two-factor Authentication Protocol}

\author{}
\institute{}

\maketitle  

\begin{abstract}
%Two-factor authentication (2FA) schemes that rely on a combination of knowledge factors (\eg, PIN) and device possession have gained popularity. Some of these schemes remain secure even against strong adversaries that (a) observe the traffic between a user and server, and (b) have physical access to the user’s device, or its PIN, or breach the server.  However, these solutions have several shortcomings; namely, they (i) require a user to remember \emph{multiple} secret values to prove its identity, (ii) involve \emph{several modular exponentiations}, and (iii) are in the \emph{non-standard} random oracle model. In this work, we present a 2FA protocol that resists such a strong adversary while addressing the above shortcomings. Our protocol requires a user to remember only a single secret value/PIN, does not involve any modular exponentiations, and is in a standard model. It is the first one that offers these features without using trusted chipsets. This protocol also imposes up to $40\%$ lower communication overhead than the state-of-the-art solutions do. \todo{Note application to financial cryptography}
Two-factor authentication (2FA) is commonly used for controlling access to high-value online accounts, particularly where financial transactions are involved.
In such schemes, knowledge factors (\eg, PIN) and hardware device possession are required to complete authentication.
However, existing 2FA protocols leave the PIN vulnerable should the server, user's computer, or hardware device be compromised.
In this paper, we propose a 2FA protocol proved to be secure against adversaries who can (a) observe the traffic between a user and server and (b) have physical access to the user’s device, or its PIN, or breach the server.
Compared to previous work, our protocol reduces the cost of authentication devices by not requiring tamper-resistant hardware or that the authentication device be connected to the computer, and through only using efficient symmetric cryptographic primitives.
Furthermore, our protocol is highly usable, requiring the user only to remember a short PIN and type short authentication codes and imposes up to 40\% lower communication overhead compared to state-of-the-art.
The protocol achieves these goals through a novel combination of splitting secrets between the server and authentication device and key-evolving symmetric-key encryption.
\end{abstract}



%!TEX root = main.tex



\section{Introduction}

The adoption of online services, such as online banking and e-commerce, has been swiftly increasing, and so has the effort of adversaries to gain unauthorised access to such services.  For clients to prove their identity to a remote service provider, they provide a piece of evidence, called an ``authentication factor''. Authentication factors can be based on (i) knowledge factors, e.g., PIN or password, (ii)  possession factors, e.g., access card or physical hardware token, or (iii) inherent factors, e.g., fingerprint.  
%
Knowledge factors are still the most predominant factors used for authentication \cite{bonneau2010password,JacommeK21}. 
%
%, mainly because their implementations impose low costs on the verifiers and require clients to invest minimal effort to authenticate themselves.
%
 The knowledge factors themselves are not strong enough to adequately prevent impersonation 
 %
% ; for instance, because clients may pick weak passwords that can be easily guessed 
 %
 \cite{SinigagliaCCZ20,JacommeK21}.  Multi-factor authentication methods that depend on more than one factor are more difficult to compromise than single-factor methods. Recently (on January 26, 2022), the ``Executive Office of the US President'' released a memorandum requiring
the Federal Government's agencies to meet specific cybersecurity standards, including the use of multi-factor authentication, to reinforce the Government’s defences against increasingly sophisticated threat campaigns \cite{Zero-Trust-Cybersecurity}. 
% 
Among multi-factor authentication schemes, two-factor authentication (2FA) methods, including those that rely on a combination of PIN and device possession, have attracted special attention, from banks and e-commerce, due to their low cost and good usability. % than those methods that rely on more than two factors. 
%


Researchers and companies have proposed various 2FA solutions based on a combination of PIN and device possession. Some of these solutions offer a strong security guarantee against an adversary which may (a) observe the communication between a client and server, and (b) have physical access to the client's device, or its PIN, or breaches the server. These solutions do not rely on trusted chipsets and still ensure that even such a strong adversary cannot succeed during the authentication. Nevertheless, these solutions (i) require a client to remember multiple secret values (instead of a single PIN)  to prove its identity which ultimately harms these solutions' usability, (ii) involve several modular exponentiations that make the device battery power run out fast, and (iii) are in the non-standard random oracle model. 

\vspace{-4mm}
\paragraph{\textbf{\textit{Our Contributions.}}}  In this work, we present a 2FA protocol that resists the strong adversary above while addressing the aforementioned shortcomings and imposing a lower communication cost. Specifically, our protocol:

\vspace{-2mm}
\begin{itemize}
\item[$\bullet$] requires a client to remember only a single PIN.

\item[$\bullet$] {allows the device to generate a short authentication message.} 

\item[$\bullet$] does not involve any modular exponentiations.

\item[$\bullet$] is in a standard model.

\item[$\bullet$]  imposes up to $40\%$ lower communication costs than the state-of-the-art does. 


\end{itemize}

\vspace{-2mm}

 To attain its goals, our protocol does not use any trusted chipsets; instead, it relies on a novel combination of the following two approaches. First, it requires only the server  (not the device) to verify a client’s PIN. This allows separating the location where the PIN’s secret key is stored from the location where the authenticator itself is stored. This approach ensures that an adversary cannot retrieve the PIN, even if it penetrates either location.  Second, it  (a) requires that the server and device use key-evolving symmetric-key encryption (i.e., a combination of forward-secure pseudorandom bit generator and authenticated encryption) to encrypt sensitive messages they exchange,  and (b) requires that used keys be discarded right after their use. This approach ensures the secrecy of the communication between the parties and guarantees that the adversary cannot learn the PIN, even if it eavesdrops on the parties' communication and breaks into the device or server. We formally prove the security of this protocol. 


%Our protocol also allows a client to see a description of the transaction that the server will perform after successful authentication. For instance, in the context of online banking, a transaction's description includes the transaction amount, destination account number, and payee's details. This description is displayed on the device to allow the client to verify that it matches their intention. This ``dynamic binding'' between transaction and response is necessary to meet the requirements for payment authentication set out by the EU
%Payment Services Directive 2.



%
%Including the transaction
%within the computation of the authentication, response allows the server
%to detect if a man-in-the-middle attack changes a payment transaction,
%e.g. to redirect funds to an account under control by the criminals.
%
%This ``dynamic binding'' between transaction and response is necessary
%to meet the requirements for payment authentication set out by the EU
%Payment Services Directive 2. 





%These methods allow clients to generate one-time passwords (OTPs) when a client inserts its knowledge factor into the device. An OTP might be sent to and checked by a remote verifier to authenticate a client. 
% 
% 2FA schemes that rely on the combination of a PIN and a hardware token have gained popularity among banks and e-commerce.



%Very recently (January 2022), the ``Executive Office of the US President'' has released a memorandum requiring
%agencies to meet specific cybersecurity standards, including the use of multi-factor authentication, to reinforce the Government’s defences against increasingly sophisticated threat campaigns \cite{Zero-Trust-Cybersecurity}.




%!TEX root = main.tex

\vspace{-3mm}
\section{Related Work}

\vspace{-2mm}



%In order for clients to prove their identity to a (local/remote) computer resource, they provide a piece of evidence, called an ``authentication factor''. The authentication factors can be broadly categorised based on (a) what clients know (a.k.a. knowledge factors), such as PIN, password, (b)  what a client possesses (a.k.a. possession factors), such as an access card or physical hardware token, or (c) what clients are (a.k.a. inherent factors), such as a fingerprint, or iris. 
%
%
%The knowledge factors are still the most predominant factors used for authentication \cite{bonneau2010password,JacommeK21}, mainly because their implementations impose low costs on the verifiers and require clients to invest minimal efforts to authenticate themselves. However, the knowledge factors themselves are not strong enough to adequately prevent impersonation, for various reasons, e.g., clients may pick weak passwords that can be easily guessed, or reuse them to authenticate themselves to many servers which would increase their leakage if one of the servers is penetrated by adversaries \cite{SinigagliaCCZ20,JacommeK21}. 
%
%In general, authentication methods that depend on more than one factor are more difficult to compromise
%than single-factor methods. Although the authentication mechanisms that rely on more than two factors may guarantee stronger security, they would impose higher costs and negatively affect clients' experience; therefore, two-factor authentications have gained special attention; including those that rely on a combination of password/PIN and device possession that allows clients to generate one-time passwords (OTPs) when a client inserts its secret (e.g., password or PIN) into the device. An OTP can be sent to and checked by a remote verifier to authenticate a client. The combination of a password or PIN and a hardware token has gained popularity among various e-commerce and banks.  


In this section, first, we briefly discuss the common approaches for generating a One-Time Password (OTP) which yields from a combination of a PIN and a hardware token.  Then, we provide an overview of hardware token variants. We refer readers to Appendix \ref{Survey-of-Related-Work} for a more detailed discussion. 

\vspace{-4mm}
\subsection{Common Approaches for Generating OTP}
\vspace{-2mm}

In the authentication that relies on a combination of knowledge and possession factors, once the client enters the secret into the hardware token, the device combines this secret with the output of one of the following methods to generate a unique OTP:  (i) a random challenge, sent by the server to the device (through the client);  (ii) an internal counter maintained by the server and device, or (iii) the current accurate time, kept by the server and device. There exist 2FA solutions (including ours) that employ a combination of the above approaches. 

\vspace{-4mm}
\subsection{Variants of OTP Hardware Tokens}

\vspace{-2mm}
\subsubsection{Connected Tokens.}
This type of token requires a client to physically connect the token to their computer via which the client is authenticating. After that, the device transmits the authentication information to the computer. USB tokens and smart cards are two popular token technologies in this category.  Various companies including Google and  ``Fast IDentity Online'' (FIDO) Alliance have developed USB hardware tokens. However, researchers have discovered many vulnerabilities within this standard, e.g., in \cite{PanosMNPX17,ChangMSS17,LoutfiJ15,ndss/FengLP021}.  
%
Since the introduction of smart card technology, there have been numerous smart card-based 2FA protocols (e.g., in \cite{gupta2021machine,WangW18,radhakrishnan2022dependable,kim2009more}). But, existing smart card-based solutions either use public-key cryptography which imposes a high computation cost or tamper-proof secure chipsets embedded in the card which would increase the device's cost. 


%There are a number of different types, including USB tokens, smart cards and wireless tags.[7] Increasingly, FIDO2 capable tokens, supported by the FIDO Alliance and the World Wide Web Consortium (W3C), have become popular with mainstream browser support beginning in 2015.




\vspace{-4mm}
\subsubsection{Disconnected Tokens.}

This type of token does not have a physical connection to a client's computer making them more convenient than connected tokens. Two main categories of disconnected tokens are (1) Dedicated hardware-based Tokens (e.g., in \cite{secureID,Digipass-website,Gemalto}) and (2) Mobile phone-based Tokens (e.g., in \cite{SARA22,KoganMB17,KonothFFARB20}). The first category includes RSA SecureID \cite{secureID}, OneSpan Digipass 770 \cite{Digipass-website}, and Thales Gemalto SWYS QR Token Eco \cite{Gemalto}. In RSA SecureID, an adversary who has access to the device can generate the OTP by extracting the secret key stored on the device.  The advantage of  Digipass 770 and Thales Gemalto to RSA SecureID is that they let clients see the transaction details through the token which gives them more information about the transaction they approve, so phishing becomes harder. 
%
Our investigation suggests that Digipass 770 and Thales Gemalto also \emph{locally store and verify} clients' PINs. Jarecki \textit{et al.} \cite{JareckiJKSS21} proposed a protocol to ensure that even if the server or device is corrupted a client's PIN cannot be extracted. But, it imposes high costs due to the use of public-key cryptography and numerous rounds of communication. This protocol requires the client (in addition to remembering its PIN) to locally store a cryptographic secret key. 



%Thus, unlike the above solutions, offer all the following features simultaneously; it (a) efficient, (b) is provably secure, (c) is resilient against the adversary which may corrupt the server or have access to the client's token, (d) relies on a single server, and (e) allows the client to also check the transaction detail on its token. 




%

Solutions in the second category use a mobile phone as a hardware token to generate OTP. They often rely on the added features that mobile phones offer, such as possessing a Trusted Execution Environment (TEE) or being able to communicate directly with the server. The mobile phone-based scheme in  \cite{KoganMB17} uses a combination of time-based OTP and a hash chain. It ensures that even if the adversary corrupts the server, it cannot extract the client's secret. Nevertheless, it requires: (a) the client to store a long secret key (on the mobile phone), (b)  the laptop/PC that the client uses to be equipped with a camera, and (c)  the mobile phone to invoke a hash function over a million times that can cause the phone's battery to run out fast. The protocol proposed in \cite{KonothFFARB20} relies on a phone's TEE  and messages that the server can directly send to the phone. Later,  Imran \textit{et al.} \cite{SARA22} proposes a protocol that also relies on a phone's TEE, but it improves the protocol presented in \cite{KonothFFARB20}, in that it is compatible with more android devices and supports biometric authentication.  A primary limitation of mobile phone-based OTP tokens is that they cannot be used when there is no (phone) network coverage. Also, in certain cases sharing phone numbers with the server may not suit all clients, e.g., transactions' details along with the phone number might be sold for targeted advertisements. 
%





%
%. Disconnected tokens are the most common type of security token used (usually in combination with a password) in two-factor authentication for online identification



% We refer readers to \cite{SinigagliaCCZ20} for a survey of different methods of authentications and their adoptions by banks.  




%Therefore, instead of using only a (clientname and) password/PIN, clients need to prove the possession of an additional device and/or an inherent factor. 

%!TEX root = main.tex


\vspace{-2mm}
\section{Notations and Preliminaries}
\vspace{-1mm}
 \subsection{Notations and Assumptions}
  
 To disambiguate the different uses of keys and other items of data, variables are annotated with a superscript to indicate their origin. $\VC{\cdot}$ indicates data stored at the user, $\VS{\cdot}$ means data stored at the server, and $\VM{\cdot}$ indicates data item has been extracted from a message. We define a function $\mathtt{Discard}(.)$ that takes an array of inputs and securely deletes them, e.g., from storage, memory, and catch. 
 
 We assume the token is not penetrated by an adversary in the (very short) period when the inputs of $\mathtt{Discard}(.)$ are set and when  $\mathtt{Discard}(.)$ is executed. %We refer readers to \cite{ReardonBC13} for a survey of secure data deletion approaches. 
 %
  We also assume that a user will not use its hardware token after it has been stolen. % (and it will be issued with a replacement with fresh parameters). 
  Table \ref{notation-table} in Appendix \ref{notation-table-} presents a summary of notations used in this paper. 
 

 
 
% 
% 
% \begin{itemize}
% 
% \item $\prf_{k}(.)$:  Pseudorandom function taking a  key $k$. It is used to derive a verifier and session key
% 
% \item FS-PRG:  Forward-secure Pseudorandom Bit Generator. It is used to derive temporary keys.
% 
% \item $\VC{k}$ and $\VS{k}$: Authenticated Encryption (AE) key at the user and server sides respectively.  Key $k$ is randomly generated by the system operator and stored by the user as $\VC{k}$ and server as $\VS{k}$ at token creation. They are constant for the lifetime of the token.
% 
% \item $\VC{\state}$ and $\VS{\state}$: The state of FS-PRG at the user and server sides respectively. They are initialised to randomly generated state $\state_{\sss 0}$ at token creation. They are updated by using FS-PRG.
% 
% 
% \item $\VC{\keyt_{\sss 1}}$ and  $\VS{\keyt_{\sss 1}}$:  Temporary keys for the enrolment phase. They are output by FS-PRG and used for a single message exchange before being discarded.
% 
% \end{itemize}
 
 
 %We present a summary of variables in \prettyref{tab:variables}. 
 
 %% !TEX root =main.tex


\vspace{-3mm}

\begin{table}[!htb]
\begin{scriptsize}
\footnotesize{
\caption{ \small{Notation table}.}\label{commu-breakdown-party} 
\renewcommand{\arraystretch}{.9}
\scalebox{.935}{
% 1st table
\begin{tabular}{p{2cm}@{\hskip 1em} p{5cm}@{\hskip 1em}p{5cm}}%{|c|c|c|} 

\hline 

\cellcolor{gray!30} \scriptsize \textbf{Symbol}&\cellcolor{gray!30} \scriptsize\textbf{Purpose}&\cellcolor{gray!30} \scriptsize\textbf{Source and lifetime}  \\
    \hline
    
     \hline



\cellcolor{white!20}\scriptsize$\prf_{k}(.)$ &\cellcolor{white!20}\scriptsize  Pseudorandom function.&\cellcolor{white!20}\scriptsize Used to derive a verifier and session key.\\ 


\cellcolor{gray!20}\scriptsize FS-PRG &\cellcolor{gray!20}\scriptsize  Forward-secure Pseudorandom Bit Generator.&\cellcolor{gray!20}\scriptsize Used to derive temporary keys.\\ 


 \cellcolor{white!20}\scriptsize \VC{k}, \VS{k} &\cellcolor{white!20}\scriptsize Authenticated Encryption (AE) key at the client and server sides respectively.&\cellcolor{white!20}\scriptsize {Key $k$ randomly generated by the system operator and stored by the client as \VC{k} and server as \VS{k} at device creation. Constant for the lifetime of the device.}\\   
 %
 
 
  \cellcolor{gray!20}\scriptsize \VC{\state}, \VS{\state}&\cellcolor{gray!20}\scriptsize {The state of FS-PRG at the client and server sides respectively.}&  \cellcolor{gray!20}\scriptsize Initialised to randomly generated state $\state_{\sss 0}$ at device creation. Updated using FS-PRG. \\   
  %
  
  
  
  \cellcolor{white!20}\scriptsize  \VC{\keyt_{\sss 1}}, \VS{\keyt_{\sss 1}} &\cellcolor{white!20}\scriptsize Temporary key for the enrolment phase.& \cellcolor{white!20}\scriptsize Output by FS-PRG and used for a single message exchange before being discarded.\\   
  %
  \cellcolor{gray!20}\scriptsize \VC{\keyt_{\sss 2}}, \VS{\keyt_{\sss 2}}, \VC{\keyt_{\sss 3}}, \VS{\keyt_{\sss 3}} &\cellcolor{gray!20}\scriptsize Temporary keys of $\mathtt{PRF}$, used in the authentication phase.&\cellcolor{gray!20}\scriptsize Output by FS-PRG and used for a single message exchange before being discarded. \\   
  %

    \cellcolor{white!20}\scriptsize \VC{\counter}, \VS{\counter} &\cellcolor{white!20}\scriptsize Counter for synchronising FS-PRG state and detecting replayed messages.& \cellcolor{white!20} \scriptsize Initialised to zero at device creation. \VC{\counter} and \VS{\counter} are updated atomically along with \VC{\state} and \VS{\state} respectively. \\
%
   \cellcolor{gray!20}\scriptsize   \VS{\nonce}, \VM{\nonce} & \cellcolor{gray!20}\scriptsize Random challenge for detecting replayed messages.& \cellcolor{gray!20}\scriptsize Generated randomly by the server for each message.\\
 %
\cellcolor{white!20}\scriptsize  \VC{\salt}&\cellcolor{white!20}\scriptsize   Random PIN-obfuscation secret key. &\cellcolor{white!20}\scriptsize Initialised to randomly generated value at device creation. Not known by the server or system operator. \\ 
%
\cellcolor{gray!20}\scriptsize \VC{\pin} &\cellcolor{gray!20}\scriptsize  Client's PIN. &\cellcolor{gray!20}\scriptsize Entered by the client. It is never stored in the device and used to generate a verifier. \\      
%
\cellcolor{white!20}\scriptsize \VC{\verifier}, \VS{\verifier}, \VM{\verifier} &\cellcolor{white!20}\scriptsize  Verifier, generated from PIN-obfuscation key and the client's PIN. &\cellcolor{white!20}\scriptsize  Stored by the server after the enrolment phase. It is not stored by the client. \\  

%
 \cellcolor{gray!20}\scriptsize  \VS{\trans}, \VM{\trans} & \cellcolor{gray!20}\scriptsize Description of a transaction to be authenticated. & \cellcolor{gray!20}\scriptsize  Generated and sent by the server.\\
 %
\cellcolor{white!20}\scriptsize \VC{response} &\cellcolor{white!20}\scriptsize Authentication response. &\cellcolor{white!20}\scriptsize Computed by the client.\\   
%       
            
\cellcolor{gray!20}\scriptsize  \VS{expected} &\cellcolor{gray!20}\scriptsize Expected authentication response. &\cellcolor{gray!20}\scriptsize Computed by the server.\\
%%%%%%%


                 
 \hline
  

 \hline

   
             
\end{tabular}\label{tab:variables}
%
}
}
%\end{center}
\end{scriptsize}
\end{table}
\vspace{-5mm}


%%%%%%%%%%%%%%%%%%%%%%%%%%%%%%%%%%%%%%%%%%%























\subsection{Pseudorandom Function}\label{subsec:PRF}

Informally, a pseudorandom function (\prf) is a deterministic function that takes as input a key and some argument. It outputs a value indistinguishable from that of a truly random function with the same domain and range.  A formal definition of a $\prf$ is given by Katz and Lindell~\cite{KatzLindell2014} and is included in Appendix \prettyref{app:def-FS-PRG}.


 \vspace{-2mm}
\subsection{Forward-Secure Pseudorandom Bit Generator}
 %
A Forward-Secure Pseudorandom Bit Generator (FS-PRG) is a stateful object with two algorithms and two positive integers; namely, $\mathsf{FS\text{-}PRG} = \Big((\mathsf{FS\text{-}PRG}.\kgen, $ $ \mathsf{FS\text{-}PRG.next}),( b, n)\Big)$, as defined in \cite{BellareY03}.  The probabilistic key generation algorithm $\mathsf{FS\text{-}PRG}.\kgen$ takes a security parameter as input and outputs an initial state ${st}_{\sss 0}$ of length $s$ bits. $\mathsf{FS\text{-}PRG.next}$ is a key-updating algorithm which, given the current state ${st}_{\sss i-1}$, outputs a pair of a $b$-bit block ${out}_{\sss i}$ and the next state ${st}_{\sss i}$. We can produce a sequence  ${out}_{\sss 1},..., {out}_{\sss n}$  of  $b$-bit output blocks, by first generating a key  ${st}_{\sss 0}\stackrel{\sss \$}\leftarrow\mathsf{FS\text{-}PRG}.\kgen(1^{\sss \lambda})$ and then running $({out}_{\sss i}, st_{\sss i})\leftarrow  \mathsf{FS\text{-}PRG.next} (st_{\sss i-1})$ for all $i, 1\leq i\leq n$. As with a standard pseudorandom bit generator, the output blocks of this generator should be computationally indistinguishable from a random bit
string of the same length. The additional property required from a
FS-PRG is that even when the
adversary learns the state, output blocks generated before the point of
compromise remain computationally indistinguishable from random bits.
This requirement implies that it is computationally infeasible to
recover a previous state from the current state. Appendix \ref{app:def-FS-PRG} restates a formal definition and construction of  FS-PRG. 


Recall, $\mathsf{FS\text{-}PRG.next}$ updates the state of the forward-secure generation by one step; however, our protocol sometimes needs to invoke $\mathsf{FS\text{-}PRG.next}$ multiple times sequentially. Thus, for the sake of simplicity, we define a wrapper algorithm $ \update(\state_{\sss a}, d)$ which wraps  $\mathsf{FS\text{-}PRG.next}$. Algorithm $\update$  as input takes a current state and new parameter $d$ that determines how many times   $\mathsf{FS\text{-}PRG.next}$ must be invoked internally. It invokes  $\mathsf{FS\text{-}PRG.next}$ $d$ times and outputs the pair  $({out}_{\sss b}, \state_{\sss b})$  which are the output of $\mathsf{FS\text{-}PRG.next}$ when it is invoked for $d$-th time, where $ b> a$. 



%For example, a practical instantiation proposed by Bellare and Yee is
%based on AES. $\mathsf{GEN.next}$ computes the next state by AES-CTR mode
%encrypting $x+s$ zero-bits with a starting counter of $0$, under the key
%$\mathsf{St}_{i-1}$. The first $x$ bits of the ciphertext is $\mathsf{Out}_i$, and the next $s$
%bits are $\mathsf{St}_i$. For $x=s=128$ using AES-128, $\mathsf{GEN.next}$ requires two
%computations of the AES encryption under one key for each update to the
%state and meets the security definitions of forward-secure key updating
%for $n < 2^{64}$.

 \vspace{-1mm}
\subsection{Authenticated Encryption (AE)} 

Informally, authenticated encryption $\Pi=(\mathsf{Gen}, \mathsf{Enc}, \mathsf{Dec})$ is an encryption scheme that simultaneously ensures the secrecy and integrity of a message. It can be built via symmetric or asymmetric-key encryptions. In this work, we use authenticated symmetric-key encryption, due to its efficiency. $\mathsf{Gen}$ is a probabilistic key-generating algorithm that takes a security parameter and returns an encryption key $k$. $\mathsf{Enc}$ is a deterministic encryption algorithm that takes the secret key $k$ and a message $m$, it returns a ciphertext $M$ along with the corresponding tag $t$. $\mathsf{Dec}$ is a deterministic algorithm that takes the ciphertext $M$, the tag $t$, and the secret key $k$. It first checks the tag's validity, if it accepts the tag, then it decrypts the message and returns $(m,1)$. Otherwise, it returns $(.,0)$. 


The security of such encryption consists of the notion of secrecy and integrity. The secrecy notion requires that the encryption be secure against Chosen-Ciphertext Attacks, \ie, CCA-secure. The notion of integrity considers existential unforgeability under an adaptive chosen message attack. We refer readers to \cite{KatzLindell2014} for a formal definition of authenticated symmetric-key encryption. 



%\subsection{Authenticated Symmetric-key Encryption} We make use an Authenticated Encryption Scheme with Auxiliary data, such as AES-GCM.
%The encryption function $\enc_\key(\nonce,H,M)$ takes a non-repeating nonce \nonce, header $H$ and Message $M$, encrypts it under key \key{} and produces ciphertext. The recipient can recover $M$ by passing the decryption function \dec(\nonce,H,C) the same nonce and header, and ciphertext $C$, \ie $M = \dec_\key(\nonce,H,\enc_\key(\nonce,H,M))$.
%The header and message may have multiple fields, which are separated with $\|$.
%Since the header and nonce are both required for decryption, they will be included inside the message outside of the ciphertext.
%If any of \nonce, $H$ or $C$ have been tampered with, the authenticated decryption function will return an error.
%For clarity, in descriptions of the protocol we omit the case of a decryption error; should this occur the protocol exchange is immediately aborted.
%We combine decryption with message decoding through the $\extract_\key$ function which validates and decrypts a message using key \key{} and extracts fields from the plaintext. 



%!TEX root = main.tex

%\subsection{The Model} \label{sec::model}


\section{Threat Mode and System Design}\label{sec::model}

A two-factor authentication scheme involves two players,
%
%\vspace{-2mm}
%\begin{itemize}
%\item[$\bullet$] {Client ($C$)}: an honest party which tries to prove its identity to a server by using a combination of a PIN and a device. 
%
%
%\item[$\bullet$] {Server ($S$)}:  a semi-honest adversary which follows the protocol's instructions and tries to learn $C$'s PIN. It also tries to authenticate itself to $C$.  
%\end{itemize}
%
 {User ($U$)}: an honest party which tries to prove its identity to a server by using a combination of a PIN and a device and {Server ($S$)}:  a semi-honest adversary which follows the protocol's instructions and tries to learn $U$'s PIN. It also tries to authenticate itself to $U$.  


We let the server communicate with the device through the user's computer, i.e., the client. Specifically, similar to previous works (e.g., those in \cite{JareckiJKSS21,Digipass-website,Gemalto}), we assume the device has a camera that lets the device scan a 2-D barcode cotnaining messages the server sends to it via the client. Each of the above parties may have several instances running concurrently. In this work, we denote instances of user and server by  $U^{i}$ and  $S^{j}$ respectively. Each instance is called an oracle. \todo{Assume high bandwidth in one direction, low in another} 

Figure \ref{fig:setup.} outlines the message flow of our 2FA scheme during the authentication phase.  At a high level, the authentication phase works as follows. Any time the user wants to authenticate itself to the server, the user sends a message to the server via the client.  The server replies to the client with a challenge and transaction details (Step 1). The user scans the message that the client received with the device (Step 2). The device shows the transaction details on the screen (Step 3). The user inserts the PIN into the device (Step 4). The device generates a response (Step 5). The user manually types the response into the client (Step 6).  The client sends the response to the server for authentication (Step 7). 


\vspace{-3mm}
\begin{figure}
\begin{centering}
\includegraphics[width=12cm]{setup}
\end{centering}
\caption{\label{fig:setup.}Protocol participants and message flows}
\end{figure}
\vspace{-4mm}
 %
%\subsection{Formal Model} \label{sec::model}
%
To formally capture the capabilities of an adversary, $\mathcal{A}$, in a hardware token-based 2FA,  we mainly use the (adjusted) model proposed by Bellare \textit{et al.} \cite{BellarePR00}. In this model, the adversary’s capabilities are cast via queries that it sends to different oracles, i.e., instances of the honest parties; the user and server interact with each other for some fixed number of flows, until both instances have terminated. By that time, each instance should have accepted holding a particular session key (\seckey), session id (\SID), and partner id (\PID). At any point in time, an oracle may \emph{``accept''}. When an oracle accepts, it holds \seckey, \SID, and \PID. A user instance and a server instance can accept at most once. The above model was initially proposed for password-based key exchange schemes in which the adversary does not corrupt either player.  Later, Wang \textit{et al.} \cite{WangW18}  added more queries to the model of  Bellare \textit{et al.} to make it suitable for two-factor authentication schemes. The added queries allow an adversary to learn either of the user's factors (\ie, either PIN or secret parameters stored in the hardware token) or the server's secret parameters.  Below, we restate the related queries. 
 





\begin{itemize}
%
\item [$\bullet$] \execute($U^{i}, S^{j}$): this query captures \textbf{passive} attacks in which the adversary, $\mathcal{A}$, has access to the messages exchanged between $U^{i}$ and $S^{j}$ during the correct executions of a 2FA protocol, $\pi$. 



\item [$\bullet$] \reveal($I$): this query models the misuse of the session key $sk$ by instance $I$.  Adversary $\mathcal{A}$ can use this query if $I$ holds a session key; in this case, upon receiving this query, $sk$ is given to $\A$. 
%



\item [$\bullet$] \test($I$): this query models the semantic security of the session key. It is sent at most once by $\A$ if the attacked instance $I$  is ``fresh'' (\ie, in the current protocol execution $I$ has accepted and neither it nor the other instance with the same $\SID$ was asked for a Reveal query). This query is answered as follows. Upon receiving the query, a coin $b$ is flipped. If $b=1$, then session key $sk$ is given to $\A$; otherwise (if $b=0$), a random value is given to $\A$. 



\item [$\bullet$] \send($I, m$):  this query models \textbf{active} attacks where $\mathcal{A}$ sends a message, $m$, to instance $I$ which follows  $\pi$'s instruction, generates a response, and sends the response back to $\mathcal{A}$.  Query  \send($U^{\sss i}, \text{start}$) initialises $\pi$; when it is sent, $\mathcal{A}$ would receive the message that the user would send to the server. 

%Query  \send($C^{i}, \text{start}$) initialises $\pi$; when it is sent, $\mathcal{A}$ would recive the message that the client would send to the server. 
%
%\item [$\bullet$] \test($I$):
%%
%\item [$\bullet$] \reveal($I$): This query models the misuse of session keys and the idea that the loss of a session key should not harm other sessions. The query is only available to $\mathcal{A}$ if the targetted instance $I$ actually holds a session key, $sk$. When it is used, it returns $sk$ to $\mathcal{A}$. 



%
\item [$\bullet$]  \corrupt($I, a$): this query models the adversary's capability to corrupt the involved parties. 



\begin{itemize}

\item if $I=U$:  it can learn (only) one of the factors of $U$. Specifically, 
%



\begin{itemize}
\item if $a=1$, it outputs $U$'s PIN. 



\item if  $a=2$, it outputs all parameters stored in the hardware token. 
  \end{itemize}
  
  
  
\item if $I=S$, it outputs all parameters stored in $S$. 
\end{itemize}
%%
\end{itemize}

\subsubsection{Authenticated Key Exchange (AKE)  Security.} Security notions (i.e., session key's semantic security and authentication) are defined with regard to the executing of protocol $\pi$, in the presence of   $\mathcal{A}$. To this end, a game $Game^{\sss ake} (\mathcal{A},\pi)$ is initialized by drawing a PIN  from the PIN's universe,
providing coin tosses to $\A$ as well as to the oracles, and then running the adversary by letting it ask a polynomial
number of queries defined above. At the end of the game, $\A$  outputs its guess $b'$ for  bit $b$
involved in the Test-query. 



\noindent\textit{Semantic security.} It requires that the privacy of a session key be preserved in the presence of $\A$, which has access to the above queries. We say that $\A$  wins if it manages to correctly guess bit $b$ in the Test-query, \ie, manages to output $b'=b$. We denote its advantage as the probability that $\A$  can correctly guess the value of $b$; specifically, such an advantage is defined as $Adv_{\sss\pi}^{\sss ss}(\A)=2Pr[b=b']-1$, where the probability space is over all the random coins of the adversary and all the oracles.  Protocol $\pi$ is said to be semantically secure if $\A$'s advantage is negligible in the security parameter, i.e., $Adv_{\sss\pi}^{\sss ss}(\A)\leq \mu(\lambda)$.


%The authentication property is another  fundamental feature that a two-factor authentication scheme must offer; 



\noindent\textit{Authentication.} It requires that $\A$ must not be able: (a)  to impersonate $U$, even if it has access to the traffic between the two parties as well as having access to either $U$'s PIN, or its authentication device, or (b) to impersonate  $S$, even if it has access to the traffic between the two parties.   We say that   $\A$  violates mutual authentication if some oracle accepts a session key and terminates, but has no partner oracle, which shares the same key.  Protocol $\pi$ is said to achieve mutual authentication if for any adversary $\A$  interacting with the parties, there exists a negligible function $\mu(.)$ such that for any security parameter $\lambda$ the advantage of $\A$ (\ie, the probability of successfully impersonating a party)  is negligible in the security parameter, i.e., 
%
$Adv_{\sss\pi}^{\sss aut}(\mathcal{A})\leq \mu(\lambda)$.


In certain schemes (including ours), during the key agreement and authentication phase, the user needs to also verify a message, e.g., a bank transaction. To allow such verification to be carried out deterministically, which will be particularly useful in the scheme's proof, we define a predicate $y\leftarrow \phi(m, \pi)$, where $y\in \{0,1\}$. This predicate takes as input a message $m$ (\eg., bank's transactions) and a policy $\pi$ (e.g., a user's policy specifying a payment amount and destination account number). It checks if the message matches the policy. If they match, it outputs $1$; otherwise, it outputs $0$.  






%However, we do not use the original BPR2000
%model directly, but adopt the reified version proposed by
%Bresson et al. [50] with a few key modifications so that we
%can define the special security goals (e.g., security against
%smart card loss attack) for two-factor authentication. We
%refer the reader unfamiliar with the BPR2000 model to [35],
%[50] for more details.









\section{The Protocol}\label{sec::the-protocol}

Recall that we wish to build an authentication protocol for which the server can verify that the PIN has been entered correctly but that an adversary cannot discover the correct PIN given access to challenge/response pairs and all data stored on the device, or access to all data stored on the server. 
These properties must be assured even when the PIN is small enough to be brute-forced.
We achieve this goal through (a) performing the PIN verification only on the server, which imposes a rate limit on verification, (b) encrypting every sensitive message exchanged between the server and user using key-evolving symmetric-key encryption (\ie, a combination of forward-secure pseudorandom bit generator and authenticated encryption), and (c) protecting against server compromise by never directly sending the PIN to the server.

Our protocol consists of three main phases; namely, (i) a setup phase, performed once when the authentication device is manufactured, (ii) an enrolment phase for setting or changing a user's PIN, and (iii) an authentication phase in which the actual authentication is performed. As we already stated, each party has a unique (public) ID. In the protocol, we assume the parties include their IDs in their outgoing messages.  Similar to other two-factor authentication schemes, we assume the server maintains a local threshold, and if the number of incorrect responses from a user within a fixed time exceeds the threshold, then the user and its device will be locked out. Such a check is implicit in the protocol's description. 

\subsection{Setup Phase}
\label{sec:setup}

To bootstrap the protocol, in the setup phase, we require that the user
and server share an \emph{initial} randomly generated key $k$ for AE and key $\state_{\sss 0}$ for  FS-PRG.
The counter for the FS-PRG state is set to $0$ on both sides. 
These values could be securely loaded into the device at the time of
manufacture or can be sent (via a secure channel) to the user who can use the device camera to scan and store them in the device. In this phase, the device generates and locally stores a random secret key $\VC{\salt}$ for \prf.  \prettyref{fig:setup} presents the setup in detail. 


\vspace{-3mm}
\begin{figure}[!htbp]
\begin{center}
    \begin{tcolorbox}[enhanced,width=4.75in, height=65mm, left=1mm,top=-1mm,
    drop fuzzy shadow southwest,
    colframe=black,colback=white]
%{\small{
 \centering
 \procedure{}{%
 \hspace{8mm}\underline{\textbf{User's device}} \< \<\hspace{4mm} \underline{\textbf{Server}} \\
  % \noindent\rule{4.6in}{1pt}
 \pcln \< \<\hspace{4mm} k \ssample \mathsf{Gen}(\secparam) \\
 \pcln \< \<\hspace{4mm} \state_{\sss 0} \ssample \mathsf{FS\text{-}RPG}.\kgen(\secparam) \\
  \pcln \< \<\hspace{4mm} U_{\sss \mathrm{ID}} \gets \mathsf{IDGen}(\secparam) \pclb
  %
 \< \sendmessageleft*{(\text{$k$,  $\state_{\sss 0}$, $U_{\sss \mathrm{ID}}$)}} \< \\
 %
 \pcln \text{set  $\VC{k}$ to $k$} \< \<\hspace{4mm} \text{set $\VS{k}$ to $k$}\\
 %
 \pcln \text{set $\VC{\state}$ to  $\state_{\sss 0}$} \< \<\hspace{4mm} \text{set  $\VS{\state}$ to $\state_{\sss 0}$}\\
 %
  \pcln \text{store$(U_{\sss \mathrm{ID}}, \VC{k}, \VC{\state})$} \< \<\hspace{4mm} \text{store $(U_{\sss \mathrm{ID}}, \VS{k}, \VS{\state})$}\\
  %
 \pcln \VC{\counter} \gets 0 \< \<\hspace{4mm} \VS{\counter} \gets 0 \\
 %
 \pcln \VC{\salt} \ssample \{0,1\}^{\sss \psi} \< \<\hspace{4mm} \\
% \pcln \VC{\salt} \ssample \kgen(\secparam) \< \< \\
 }
%}}
\end{tcolorbox}
\end{center}
\vspace{-4.5mm}
    \caption{Setup phase.}
    \label{fig:setup}
\end{figure}

\vspace{-5mm}

\subsection{Enrolment Phase}
\label{sec:enrollment}

The goal of the enrolment phase is to set the user's PIN, without providing the  
 server with sufficient information to discover this PIN. 
% 
% The server allows this phase to take place only over a channel through which the client has already proven their identity. 
 %
  At the end of this phase, the server will have stored the verifier $\verifier$ corresponding to the user's selected PIN.
The steps involved in this phase are detailed in \prettyref{fig:enrollment}.  




%%%%%%%%%%%%%%%%%
\begin{figure}[!htb]
\setlength{\fboxsep}{1pt}
\begin{center}
    \    \begin{tcolorbox}[enhanced,width=4.75in, height=157mm, left=1mm,top=-.5mm,
    drop fuzzy shadow southwest,
    colframe=black,colback=white]
%{\small{
 \centering
 \procedure{}{%
  \hspace{8mm}\underline{\textbf{User's device}} \< \hspace{12ex} \<\hspace{4mm} \underline{\textbf{Server}} \\
 %%%%%%
 %
\pclb
 %
 \sendmessagerightx[5cm]{6}{(U_{\sss ID},  \text{enrolment})} \\% encrypt
 %%%%%% Server
 \pcln \< \<\hspace{2.6mm} \VS{\counter} \gets \VS{\counter} + 1  \\
 \pcln \< \<\hspace{2.6mm} \VS{\keyt_{\sss 1}}, \VS{\state} \gets \update(\VS{\state}, 1)\\% update
 %
% \pcln \< \< key_1= H(\VS{\keyt_1}||
 %
 \pcln \< \<\hspace{2.6mm} \VS{\nonce} \stackrel{\$}\leftarrow \bin^{\sss\secpar}  \\
 %
 \pcln \< \<\hspace{2.6mm} p= \VS{\nonce}|| \VS{\counter}\\ % server-- set plaintext
 %
 \pcln \< \<\hspace{2.6mm} {(M,t) \gets \enc_{\sss\VS{k}}(p) } \pclb
 %
 \sendmessageleftx[7cm]{6}{(M,t) } \\% server--encrypt
 %%%%% Client
 %
  \pcln (p,b) \gets \mathsf{Dec}_{\sss\VC{k}}(M,t) \< \< \\% client-- decrypt
  %
  \pcln \iif b\neq 1,    \text{\ then\ go\ to\ } \ref{enroll:fail} \\ % client-- verify tag
  %
   \pcln \text{Parse\ } p \text{\ which\ yields\ } \VM{\nonce}, \VM{\counter} \< \< \\ % client-- parse
  %
 \pcln\label{enroll:clinet-check-counter}  \iif \VM{\counter} \le \VC{\counter}, \text{\ then\ go\ to\ } \ref{enroll:fail} \\ %client check counter
 %
 %\pcln  \VC{\state} \gets  \tmp_{\VC{\state}}\\% client set state
 \pcln \text{Request \VC{\pin} from the user} \< \< \\
 %
 \pcln \VC{\verifier} \gets \prf_{\sss\VC{\salt}}(\VC{\pin}) \< \< \\
 %
 %%%%%%%%%%%%%%
 \pcln d=  \VM{\counter}-\VC{\counter} \< \< \\
 %
%  \pcln \forEach i \in[1,d]: \< \< \\
  %
   \pcln   \VC{\counter} \gets \VC{\counter} + d\< \< \\
    \pcln \VC{\keyt_{\sss 1}},  \VC{\state} \gets \update(\VC{\state}, d) \< \< \\ % update
 %%%%%%%%
 %
 %\pcln \VC{\keyt_{\sss 1}} \gets k \< \< \\ % set the key
 %
 \pcln p'=  \VM{\nonce} || \VC{\verifier}\< \< \\ %client--set plaintext
 %
 \pcln (M',t') \gets \enc_{\sss\VC{\keyt_{\sss 1}}}(p')\< \< \pclb
 %
 \sendmessagerightx[7cm]{6}{(M',t') } \\% encrypt
 %
  \pcln \label{enroll:fail} \text{Discard (\VC{\pin}, \VC{\verifier}, \VC{\keyt_{\sss 1}},  \VC{\nonce}})% \text{and\ abort} 
 %%%%%% Server
 \< \<\hspace{2.6mm} (p',b') \gets \mathsf{Dec}_{\sss\VS{\keyt_{\sss 1}}}(M',t') \< \< \\% server-- decrypt
  %
   \pcln\< \<\hspace{2.6mm} \iif b'\neq 1,    \text{\ then\ go\ to\ } \ref{enroll:fail-server-}\\ % client-- verify tag
  %
 \pcln\< \<\hspace{2.6mm} \text{Parse\ } p' \text{\ yielding\ }  \VM{\nonce},  \VM{\verifier} \< \< \\
  %
 \pcln \< \<\hspace{2.6mm} \iif \VM{\nonce} \ne \VS{\nonce}, \text{then \ go\ to\ } \ref{enroll:fail-server-} \\
 %
 \pcln \< \<\hspace{2.6mm} \text{Store \VM{\verifier} as \VS{\verifier}} \\
 %
% \pcln \label{enroll:fail} \text{Discard \VC{\pin}, \VC{\verifier}, \VC{\keyt_{\sss 1}},  \VC{\nonce}} \text{and\ abort} 
 %
  \pcln \label{enroll:fail-server-} \< \<\hspace{2.6mm}  \text{Discard (\VS{\keyt_{\sss1}}, \VS{\nonce}}) \\ %\text{and\ abort} \\
 }
%}}
\end{tcolorbox}
\end{center}
\vspace{-4.5mm}
    \caption{Enrolment phase.}
    \label{fig:enrollment}
\end{figure}


We briefly explain how this phase works.  The server first updates the FS-PRG's state, which results in a new state and random value \VS{\keyt_{\sss 1}}; it also increments its counter by one. Then, the server generates a random challenge \VS{\nonce}. The server sends the enrolment challenge message which is a combination of the current counter and the challenge encrypted via the AE under the shared key $k$.
%
 On receiving this message, the client passes it to the device. The device decrypts the message using $k$ that was shared with the server during the setup phase.  If decryption succeeds, it extracts the server's challenge and counters from the message. To recover $\VS{\keyt_{\sss 1}}$ from the message the device's counter must be less than or equal to the counter it received from the server, which the protocol ensures is the case with a high probability (see \prettyref{app:synchronisation}). Next, the device requests the PIN from the user and ensures it is what the user intends, \eg, by requesting it twice and checking they match.

The device then generates a verifier $\VC{\verifier}$, by deriving a pseudorandom value from the PIN using $\prf$ and the random key $\VC{\salt}$ it generated in the setup phase.  After that, the device locally synchronises the FS-PRG's state with the server by updating the state until it matches the counter received from the server; this yields \VC{\keyt_{\sss 1}}. This synchronisation is possible because the check at line \ref{enroll:clinet-check-counter} has already assured that the device's state is behind the server's state by at least one step. After the update, \VC{\keyt_{\sss 1}} will equal \VS{\keyt_{\sss 1}} because the initial FS-PRG's state is the same (from the setup phase) and the two generators have been updated the same number of times. The user's device then encrypts the verifier and challenge under \VC{\keyt_{\sss 1}} and sends this to the server. 
 %
On receiving and validating this message, the server decrypts the message using \VS{\keyt{\sss 1}}, then extracts the challenge and verifier.
If the challenge does not match the one corresponding to the current protocol exchange, the protocol halts.
If the challenge does match, the server stores the verifier, $\VS{\verifier}$,  associated with the user's account.

Finally, the device discards the challenge, \VC{\keyt_{\sss 1}}, PIN, and \VC{\verifier} so that the PIN can no longer be recovered from the device. Note that the device can re-generate \VC{\verifier} using \VC{\salt} when the user types in its PIN again. The server also discards the challenge and \VC{\keyt{\sss 1}} as they are no longer needed. Following the successful completion of this protocol, the server will store the verifier corresponding to the user's selected PIN and both server and device will have synchronised their FS-PRG's state.

\subsection{Authentication Phase}
\label{sec:authentication}

The goal of the authentication process is to give the server assurance that the device is currently present, the correct PIN has been entered, and the user has been shown the transaction that the server wishes to execute.

%%%%%%%%%%%%%%%%%%
\begin{figure}[!htbp]

\begin{center}
     \begin{tcolorbox}[enhanced,width=4.75in, height=196mm, left=1mm,top=-1mm,
    drop fuzzy shadow southwest,
    colframe=black,colback=white]
%{\small{
 \centering
 \procedure{}{%
  \hspace{8mm}\underline{\textbf{User's device}} \< \hspace{0ex} \<\hspace{2.6mm} \underline{\textbf{Server}} \\
\pclb
 %
\sendmessagerightx[5cm]{6}{(U_{\sss ID}, \text{authentication}) } \\% encrypt
 %
 \pcln \< \<\hspace{2.6mm} \VS{\counter} \gets \VS{\counter} + 1 \\ % server-- increment counter
 %
 \pcln \< \<\hspace{2.6mm}  \VS{\keyt_{\sss 2}}, \VS{\state} \gets \update(\VS{\state}, 1)\\% server-- update
 %
  \pcln \< \<\hspace{2.6mm}  \tmp_{\sss\VS{\counter}} \gets \VS{\counter}, \ \VS{\counter} \gets \VS{\counter} + 1 \< \< \\
 %
% \pcln \< \< \VS{\counter} \gets \VS{\counter} + 1 \\ % server-- increment counter again
 %
 \pcln \< \< \hspace{2.6mm} \VS{\keyt_{\sss 3}}, \VS{\state} \gets \update(\VS{\state}, 1)\\% server-- update again
 %
 \pcln \< \<\hspace{2.6mm} \VS{\nonce} \stackrel{\$}\leftarrow \bin^{\sss \secpar}, \  p= \VS{\nonce} || \VS{\trans} \ \\  % server--pick a challenge
%
 \pcln \< \<\hspace{2.6mm} (\ddot M, \ddot t)\leftarrow \enc_{\sss\VS{\keyt_{\sss 2}}}(p)\\
 %
 \pcln \< \<\hspace{2.6mm} (\hat M, \hat t)\leftarrow \enc_{\sss\VS{k}}( \tmp_{\VS{\counter}})\pclb
 %
 \sendmessageleftx[7cm]{6}{(\ddot M, \ddot t), (\hat M, \hat t)} \\ % server-- encrypt
 %
 %%%%%  client
 %
   \pcln (\tmp_{\sss\VM{\counter}},b') \gets \mathsf{Dec}_{\sss\VC{k}}(\hat M, \hat t) \< \< \\% client-- decrypt
  %
  \pcln \iif b'\neq 1,    \text{\ go\ to\ } \ref{auth:fail} \\ % client-- verify tag
   %
    \pcln \iif \tmp_{\sss\VM{\counter}} \le \VC{\counter}, \text{\ go\ to\ } \ref{auth:fail} \< \< \\
   %
    %%%%%%%%%%%%%%
 \pcln d=  \tmp_{\sss\VM{\counter}}-\VC{\counter},\  \VC{\counter} \gets \VC{\counter} + d+1\  \< \< \\
 %
  %\pcln \forEach i \in[1,d]: \< \< \\
  % \pcln    \VC{\counter} \gets \VC{\counter} + d+1\< \< \\
    \pcln\label{auth-protocol:first-update} \VC{\keyt_{\sss 2}},  \VC{\state} \gets \update(\VC{\state}, d) \< \< \\ % update
 %%%%%%%%
   %
    %\pcln \VC{\keyt_{\sss 2}} \gets k \< \< \\ % set the key
   %
 % \pcln  \VC{\keyt_{\sss 2}}, \VC{\state} \gets \update(\VC{\state}, \tmp_{\sss\VM{\counter}})\\ % client-- 1st update
 %
  \pcln (p,b) \gets \mathsf{Dec}_{\sss\VC{\keyt_{\sss 2}}}(\ddot M, \ddot t) \< \< \\% client-- decrypt
  %
  \pcln \iif b\neq 1,    \text{\ go\ to\ } \ref{auth:fail} \\ % client-- verify tag
  %
  \pcln \text{Parse\ } p \text{\ yielding\ } \VM{\nonce}, \VM{\trans}  \< \< \\ % client parse
 %
 \pcln \text{Display \VM{\trans} for user to check} \< \< \\
 %
 \pcln \text{\iif user rejects \VM{\trans}, \text{\ go\ to\ } \ref{auth:fail}} \< \< \\
 %
  %\pcln \VC{\counter} \gets \VM{\counter} \< \< \\%client-- set counter
 %
 \pcln \text{Request \VC{\pin} from user} \< \< \\
 %
 \pcln \VC{\verifier} \gets \prf_{\sss\VC{\salt}}(\VC{\pin}) \< \< \\
%
 \pcln  \VC{\keyt_{\sss 3}}, \VC{\state} \gets \update(\VC{\state}, 1) \< \< \\
  %
 \pcln p''=  \VM{\nonce} \|  \VM{\trans} \| \VC{\verifier}\< \< \\ %client--set plaintext
 %
 \pcln \label{auth:gen-res}  \VC{\mathit{response}} \gets \prf_{\sss\VC{\keyt_{\sss 3}}}(p''|| 1) \< \< \\ % client-- encrypt
 %
  \pcln sk^{C} \gets \prf_{\sss\VC{\keyt_{\sss 3}}}(p''|| 2) \< \<  \pclb% client-- encrypt
%  \pcln sk^{C} \gets \hash(\VC{\keyt_{\sss 3}}||p'') \< \< \pclb % client-- encrypt
 %
 \sendmessagerightx[7cm]{6}{ \VC{\mathit{response}}} \\
 %
  \pcln \label{auth:fail} \text{Discard (\VC{\pin}, \VC{\verifier}, \VC{\keyt_{\sss 2}}, \VC{\keyt_{\sss 3}}, \VM{\nonce}, \VM{\trans}}) \\ 
 %%%%% server
 \pcln \< \<\hspace{2.6mm} \VS{\mathit{expected}} \gets \prf_{\sss\VS{\keyt_{\sss3}}}(\VS{\nonce}  \| \VS{\trans} \| \VS{\verifier}||1) \\
 %
 \pcln \< \<\hspace{2.6mm} \iif \VC{\mathit{response}} \ne \VS{\mathit{expected}}, \text{go\ to\ } \ref{auth:fail-server} \\
 %
\pcln\< \<\hspace{2.6mm}  sk^{S} \gets\prf_{\sss\VS{\keyt_{\sss3}}}(\VS{\nonce}  \| \VS{\trans} \| \VS{\verifier}|| 2)\\
 %
 %\pcln \< \<sk^{S} \gets \hash(\VS{\keyt_{\sss 3}}||\VS{\nonce}  \| \VS{\trans} \| \VS{\verifier}) \\
 %
% \pcln \label{auth:fail} \text{Discard \VC{\pin}, \VC{\verifier}, \VC{\keyt_{\sss 2}}, \VC{\keyt_{\sss 3}}, \VM{\nonce}, \VM{\trans}} \\ 
  \pcln \label{auth:fail-server}\< \<\hspace{2.6mm}  \text{Discard (\VS{\keyt_{\sss 2}}, \VS{\nonce}, \VS{\trans}}) 
 }
 %
\end{tcolorbox}
\end{center}
\vspace{-3.4mm}
    \caption{Authentication phase.}
    \label{fig:auth}
\end{figure}
%%%%%%%%%%%%%%%%%%

This phase works as follows. The server first updates the FS-PRG's state and corresponding counter, which results in a new state \VS{\state}, a new random value  \VS{\keyt_{\sss 2}}, and a new temporary counter $\tmp_{\VS{\counter}}$. The server updates the state and the counter one more time which yields a new state \VS{\state}, a new random value \VS{\keyt_{\sss 3}}, and a new counter \VS{\counter}. The server generates a random challenge and two ciphertexts, $\ddot M$ and $\hat M$. The former ciphertext consists of the random challenge and the description of the transaction, encrypted under key \VS{\keyt_{\sss 2}}. The latter ciphertext contains the counter $\tmp_{\VS{\counter}}$, encrypted under key $\VC{k}$. The reason $\tmp_{\VS{\counter}}$ is encrypted under key $\VC{k}$ is to allow the device to decrypt the ciphertext easily 
in case of previous message loss; for instance, when the server sends $(\ddot M, \hat M)$ to the server, but they are lost in transit, multiple times, and a fresh pair finally arrives at the client after the server sends them upon the user's request. Encrypting $\tmp_{\VS{\counter}}$ under key $\VC{k}$ (instead of one of the evolving keys) lets the device deal with such a situation. 




Upon receiving the ciphertexts, the device validates and decrypts the messages. It extracts the challenge \VM{\nonce}, counter $\tmp_{\VS{\counter}}$, and transaction \VM{\trans}. It ensures that its own counter is behind the received counter. As will be discussed in \prettyref{app:synchronisation}, this check will succeed with high probability. The device synchronises its state and counter using the server's messages. Next, the device displays the transaction for the user to check. If the user does not accept the transaction (\eg, due to an attempted man-in-the-browser attack), then the protocol aborts immediately. Assuming the user is willing to proceed, then the device prompts for the PIN, and computes the verifier \VC{\verifier} using the key \VC{\salt}. If the user enters the correct PIN, the verifier will be the same as the one sent to the server during the enrolment phase.



For the device to generate the response message, it first updates its state one more time, which results in a pseudorandom value \VC{\keyt_{\sss 3}}. Then, it derives a pseudorandom value, \VC{\mathit{response}}, from a combination of the random challenge \VM{\nonce}, transaction \VM{\trans}, verifier \VC{\verifier}, and $x=1$  using $\prf$ and \VC{\keyt_{\sss 3}}. The device generates a session key, using the above combination and key with a difference that now $x=2$. The response message is truncated to be a convenient length, displayed on the screen of the device, typed by the user into the client, and sent to the server. The device discards the PIN, the verifier, all FS-PRG keys, the challenge, and the transaction's description, so as to protect the PIN from discovery. The server computes the expected response message based on its own values of the challenge, transaction, and verifier. Note that the verifier is retrieved from the value set during the enrolment phase. The server then compares the expected response with the response sent by the client. Only if they match, the authentication is considered to have succeeded. If the response does not match the one the server expects this could indicate that the message was tampered with, or that the user entered an incorrect PIN. Next, the server generates the session key the same way as the device does.  The server also discards the FS-PRG key, the challenge, and the transaction's description.

%
Below, we formally state the security of our protocol.  First, we present a theorem stating that the advantage of an adversary in breaking the semantic security of the above protocol is negligible.  
\begin{theorem}[Semantic Security]
%Let PIN be an element distributed uniformly at random over a finite dictionary of size $N$. 
Let $\adv$ be a probabilistic polynomial time (PPT) adversary with less than $q_{\sss s}$ interactions with the parties and $q_{\sss p}$ passive eavesdropping, i.e., number of local executions. Let $\lambda$ be a security parameter and $Adv_{\sss\pi}^{\sss ss}(\A)$ be  $\adv$'s advantage (in breaking the semantic security of an AKE scheme $\pi$) as defined in Section \ref{sec::model}. Then, such an advantage for the protocol $\psi$ has the following upper bound:  
%
\vspace{-4mm}
\begin{equation*} 
Adv_{\sss \psi}^{\sss ss}(\A) \leq 2(q_{\sss s}+q_{\sss p})\Big(Adv^{\sss\prf}(\adv)+Adv^{\sss Enc}(\adv)\Big)+\frac{8(2q_{\sss s}+q_{\sss p})}{2^{\sss\lambda}}
\end{equation*}
%
\end{theorem}

Next, we present a theorem stating that the advantage of an adversary in breaking the authentication of the above protocol is negligible.  

\begin{theorem} [Authentication]
Let PIN be an element distributed uniformly at random over a finite dictionary of size $N$. Also, 
let $\adv$ be a PPT adversary with less than $q_s$ interactions with the parties and $q_p$ passive eavesdroppings. Let $\lambda$ be a security parameter and $Adv_{\sss\pi}^{\sss aut}(\A)$ be  $\adv$'s advantage (in breaking the authentication of an AKE scheme $\pi$) as defined in Section \ref{sec::model}. Then, in the protocol $\psi$, $Adv_{\sss \psi}^{\sss aut}(\A)$ has the following upper bound:  
%
\vspace{-3mm}
  \begin{equation*}
 Adv_{\sss \psi}^{\sss aut}(\A)  \leq (q_{\sss s} + q_{\sss p})\Big(Adv^{\sss\prf}(\adv)+Adv^{\sss Enc}(\adv)\Big)+\frac{9q_{\sss s}+4q_{\sss p}}{2^{\sss\lambda}}+  \cfrac{q_{\sss s}}{N}
 \end{equation*}
%
\end{theorem}

\vspace{-2mm}


We refer to Appendix \ref{sec:security} for an informal security analysis and Appendix \ref{sec::Formal-Security-Analysis} for the formal security proof of the above two theorems.  


\section{System Usability}
\label{sec:asymmetric}
%\vspace{-1mm}

Usability is of critical importance for an effective authentication system as otherwise, users will refuse to use it or implement insecure workarounds \cite{de2013comparative}.
As we stated in Section \ref{sec::model}, in our protocol, the server interacts with the device via the client. To accommodate usability and let the device easily receive the server's message, we require the device to be able to receive a message of a few hundred bytes from the server. This functionality is already present on any device capable of transaction authentication because it must be able to receive a description of the transaction to show to the user on the trusted display. Typically this communication functionality is implemented by an inexpensive camera such as in the Gemalto SWYS QR~\cite{Gemalto} or OneSpan Digipass 770~\cite{Digipass-website}, which both scan a 2D barcode shown on the screen of the client. 
The response from the device to the server can be safely truncated because the protocol ensures offline brute-force attacks are not possible. So, the verifier can be manually typed without any special hardware required for this direction of communication. The verifier length should be selected to reduce the chance of success of an online brute force attack to an acceptable level, taking into consideration the rate-limiting implemented on the server. This security-usability trade-off is not specific to our protocol and exists in all hardware token-based multi-factor authentication schemes that do not assume a high-bandwidth communication channel from the authentication device to the server. 

Another consideration is handling mistyped or forgotten PINs. As we highlighted in Section \ref{sec::the-protocol}, when setting the PIN, the device can ask the user to confirm their PIN by entering the PIN twice and alerting the user if they do not match. However, because we assume that the device has no trusted hardware we cannot store the PIN in the device. Therefore during authentication, if the wrong PIN is entered, the user will only be alerted after the response code has been verified by the server. To enhance usability by detecting mistyped PINs earlier in the protocol, at some cost of security, the device could show the user an image computed as a function of the PIN entered to help the user detect a mistyped PIN, effectively serving as a checksum. To prevent someone observing the device from discovering the PIN from the image, the function could be designed to have a large number of collisions.

%!TEX root = main.tex

\section{Evaluation}\label{sec:eval}

In this section, we briefly analyse and compare our 2FA protocol with the smart-card-based protocol proposed in  \cite{WangW18} and the hardware token-based protocol in \cite{JareckiJKSS21} as the latter two protocols are relatively efficient, do not use secure chipsets, and they consider the same security threats as we do. We summarise the analysis result in Table \ref{comparisonTable}. We refer readers to \prettyref{app:long-eval} for a more detailed evaluation. 

%Before we compare the two protocols' costs, we highlight a vital difference between the two. 




%!TEX root = main.tex


%\vspace{-2mm}
\begin{table} 

\begin{center}
\caption{ \small Comparison of efficient two-factor authentication protocols.}  \label{comparisonTable} 
%\renewcommand{\arraystretch}{1}
\begin{tabular}{|c|c|c|c|c|c|c|c|c|c|} 
\hline

{\cellcolor{gray!40}\scriptsize {Features}} &\cellcolor{gray!40}{\scriptsize {Operation}}&\cellcolor{gray!40}{ \scriptsize {Our Protocol}}&\cellcolor{gray!40}{\scriptsize{\cite{WangW18}}}&\cellcolor{gray!40} {\scriptsize{\cite{JareckiJKSS21}}}  \\
\hline

\multirow{2}{*}{\scriptsize Computation cost}


&\scriptsize{Sym-key}&\cellcolor{gray!20}\scriptsize$18$&\cellcolor{gray!20}\scriptsize$19$&\cellcolor{gray!20}\scriptsize $7$\\


\cline{2-5}

&\scriptsize {Modular expo.}&\cellcolor{gray!20}\scriptsize {$0$}&\cellcolor{gray!20}\scriptsize$5$&\cellcolor{gray!20}\scriptsize$12$\\

\hline

\scriptsize  Communication cost&$-$&\cellcolor{gray!20}\scriptsize$2804$-bit&\cellcolor{gray!20}\scriptsize$3136$-bit&\cellcolor{gray!20}\scriptsize$3900$-bit\\

\hline 
\scriptsize Not requiring multiple pass/PIN&$-$&\cellcolor{gray!20}\scriptsize{\textcolor{blue}\checkmark}&\cellcolor{gray!20}\scriptsize\textcolor{red}{$\times$}&\cellcolor{gray!20}\scriptsize\textcolor{red}{$\times$}\\ 
\hline
\scriptsize Not requiring modular expo.  &$-$&\cellcolor{gray!20}\scriptsize{\textcolor{blue}\checkmark}&\cellcolor{gray!20}\scriptsize\textcolor{red}{$\times$}&\cellcolor{gray!20}\scriptsize\textcolor{red}{$\times$}\\ 

\hline


\scriptsize Security assumption  &$-$&\cellcolor{gray!20}\scriptsize{Standard}&\cellcolor{gray!20}\scriptsize{Random oracle}&\cellcolor{gray!20}\scriptsize{Random oracle}\\ 

\hline
\end{tabular}
%}
%\renewcommand{\arraystretch}{1}
%\end{footnotesize}
\end{center}
%}
\end{table}







\subsection{Computation Cost}

In our protocol, each party (user or server) invokes the authenticated encryption scheme $4$ times and the pseudorandom function $5$ times. In the protocol proposed by Wang \etal~\cite{WangW18}, the user invokes a hash function $11$ times and performs $3$ modular exponentiations while the server invokes the hash function $8$ times and performs $2$ modular exponentiations. Moreover, in the protocol presented by Jarecki \etal~\cite{JareckiJKSS21}, the user invokes a hash function $3$ times and runs symmetric key encryption once. It also performs $10$ modular exponentiations. While the server invokes a pseudorandom function once and performs at least $2$ modular exponentiations and $2$ symmetric-key encryptions. Thus,  our protocol and the ones in \cite{WangW18,JareckiJKSS21} involve a constant number of symmetric key primitive invocations; however, our protocol does not involve any modular exponentiations, whereas those in  \cite{WangW18,JareckiJKSS21} involve a constant number of modular exponentiations which leads to a higher cost. 

\subsection{Communication Cost}

In our protocol, the communication cost of the user is $1268$ bits and the server is $1536$ bits. However, in the protocol proposed in \cite{WangW18},  the communication cost of the user is $1952$ bits and the server is  $1184$ bits; while in the protocol developed in \cite{JareckiJKSS21} the user's and server's communication costs are at least $2856$ and $1044$ bits respectively. Hence, our protocol imposes $10\%$ and $40\%$ lower communication costs than the protocols in \cite{WangW18} and \cite{JareckiJKSS21} do respectively.


\subsection{Other Features}

 In our protocol, a user needs to know and insert into the device only a single secret, \ie, a  PIN. In contrast, in the protocol in \cite{WangW18}, a user has to know and insert an additional secret, \ie, a random ID. As shown by Scott~\cite{Scott12a}, the scheme in \cite{WangW18} will not remain secure, even if only the user's ID is revealed. The protocol in \cite{JareckiJKSS21} requires the user (in addition to remembering its PIN) to locally store a cryptographic secret key of sufficient length, \eg, $128$ bits; this secret key must not be kept on the device but must be inserted into it during the authentication phase. 
 % 
 Furthermore, our protocol is secure in the standard model whereas the protocols in \cite{WangW18,JareckiJKSS21} are in the non-standard random oracle model. 



%!TEX root = main.tex

%%%%%%%%%%%
\vspace{-2.1mm}
\section{Conclusion and Future Work}
\vspace{-.5mm}

we have introduced a 2FA protocol designed to withstand a formidable adversary who may (a) eavesdrop on the communication between a user and the server, and (b) gain physical access to the user's hardware token, its PIN, or compromise the server. Our protocol offers a distinctive combination of key features that are not found in state-of-the-art schemes. 
%
Specifically,  our protocol (i) mandates the user to remember only a single PIN, (ii) relies solely on symmetric-key primitives, (iii) operates within a standard model, (iv) does not assume trust in the user's computer, and (v) incurs minimal communication costs. Notably, this protocol is the first of its kind to offer these features without necessitating tamper-resistant hardware. 
% 
%
Future research could investigate the \emph{usability} of a hardware token that embeds our 2FA protocol. Investigating the use of hardware tokens for delegated computation is an intriguing avenue to explore.








%\section{Acknowledgements}


\bibliographystyle{splncs03}
\bibliography{ref}
\appendix
%!TEX root = main.tex


\section{Related Work}\label{Related-Work}





%In order for clients to prove their identity to a (local/remote) computer resource, they provide a piece of evidence, called an ``authentication factor''. The authentication factors can be broadly categorised based on (a) what clients know (a.k.a. knowledge factors), such as PIN, password, (b)  what a client possesses (a.k.a. possession factors), such as an access card or physical hardware token, or (c) what clients are (a.k.a. inherent factors), such as a fingerprint, or iris. 
%
%
%The knowledge factors are still the most predominant factors used for authentication \cite{bonneau2010password,JacommeK21}, mainly because their implementations impose low costs on the verifiers and require clients to invest minimal efforts to authenticate themselves. However, the knowledge factors themselves are not strong enough to adequately prevent impersonation, for various reasons, e.g., clients may pick weak passwords that can be easily guessed, or reuse them to authenticate themselves to many servers which would increase their leakage if one of the servers is penetrated by adversaries \cite{SinigagliaCCZ20,JacommeK21}. 
%
%In general, authentication methods that depend on more than one factor are more difficult to compromise
%than single-factor methods. Although the authentication mechanisms that rely on more than two factors may guarantee stronger security, they would impose higher costs and negatively affect clients' experience; therefore, two-factor authentications have gained special attention; including those that rely on a combination of password/PIN and device possession that allows clients to generate one-time passwords (OTPs) when a client types its secret (e.g., password or PIN) into the device. An OTP can be sent to and checked by a remote verifier to authenticate a client. The combination of a password or PIN and a hardware token has gained popularity among various e-commerce and banks.  


In this section, we outline the related work. Appendix \ref{Survey-of-Related-Work} presents a survey of related work. 


\subsection{Common Approaches for Generating OTP}

In authentication systems using both knowledge and possession factors, the hardware token combines the user's entered secret with  (i) random challenge, (ii) an internal counter, or (iii) current accurate time to generate a unique OTP, sometimes after validating the secret. There are 2FA solutions, including our proposal in this paper, that combine these approaches.


%
%In authentication mechanisms that rely on a combination of knowledge and possession factors, once the user enters the secret into the hardware token, the token (in some cases after validating the secret) combines this secret with the output of one of the following methods to generate a unique OTP: 
%
%\begin{enumerate}[leftmargin=5mm]
%
%\item \textit{a random challenge}: This approach mandates the server to send a random challenge to the token through the client, requiring protocols to maintain the confidentiality of these challenges in the presence of potential eavesdropping adversaries.
%
%
%
%
%
%% this approach requires the server to send a random challenge to the token (through the client). Those protocols that use this approach need to ensure the random challenges themselves remain confidential in the presence of an eavesdropping adversary.
%
%\item  \textit{an internal counter}:  Solutions using this approach must consider token-side counter desynchronisation.. 
%
%
%\item \textit{the current accurate time}: This approach relies on synchronised clocks between the authentication server and token, but they may become unsynchronised over time. 
%
%\end{enumerate}



%There exist 2FA solutions (including the one we propose in this paper) that employ a combination of the above approaches. 
\subsection{Variants of OTP Hardware Tokens}

\subsubsection{Connected Token.}


%One approach to hardware tokens is to plug them into the computer's USB port, but for such devices to work there must be corresponding software installed on the computer, and this might not be possible on shared devices or computers implementing a corporate IT policy.
%Also, if any special software is needed it would have to be implemented for every supported operating system and processor architecture, and periodically updated, so imposing higher development costs.
%In the longer term, support for hardware authentication devices may become a common operating system provision, but current proposed standards to do so, such as FIDO 2.0 do not support transaction authentication and are not designed to mitigate compromise of the user's computer.
%For this reason, connected authentication hardware does not meet the requirements we have set out.
%Similarly using a smartphone as the hardware authentication device does not fulfil the usability requirement because not everyone has a smartphone or is willing to install special software on it, and smartphones have a large attack surface that can be exploited to extract the authentication secrets.







This token needs a physical connection to a computer (e.g., laptop or card reader) for user authentication. Once connected, it sends authentication information to the computer, either automatically or with a button press. USB tokens and smart cards are popular in this category. Google, Dropbox, and the FIDO Alliance have developed USB hardware token specifications. YubiKey is a well-known one implementing FIDO specifications.





However, researchers have discovered various vulnerabilities within FIDO specifications, e.g., see  \cite{PanosMNPX17,ChangMSS17,LoutfiJ15,ndss/FengLP021}.  Also, these devices require specific software on the computer, which may not work for shared devices or adhere to a corporate IT policy. If specialised software is needed, it must be developed for each supported OS and processor type, requiring ongoing updates and higher development and upkeep expenses.


%If any specialised software is required, it must be developed and implemented for each supported operating system and processor architecture. This entails periodic updates, leading to increased development and maintenance costs.


Smart card technology is another widely employed authentication method. Since its introduction in \cite{chang1991remote}, there have been numerous protocols for smart card-based 2FA (\eg, in \cite{gupta2021machine,WangW18,radhakrishnan2022dependable}). % along with a few works that identify vulnerabilities of existing solutions, \eg, in \cite{TianLHL20,WangGCW16,ChaturvediDMM16}. 
%
However, the existing smart card-based solutions (\eg, in \cite{gupta2021machine,WangW18,radhakrishnan2022dependable}) are often based on public-key cryptography which imposes a high computation and energy cost. Also some solutions (e.g., in \cite{kim2009more}) rely on tamper-proof secure chipsets embedded in the card which would ultimately increase its cost. 


%There are a number of different types, including USB tokens, smart cards and wireless tags.[7] Increasingly, FIDO2 capable tokens, supported by the FIDO Alliance and the World Wide Web Consortium (W3C), have become popular with mainstream browser support beginning in 2015.





\subsubsection{Disconnected Tokens.}

This type of token does not have a physical connection to a user's computer making them more convenient than connected tokens. 
%
%A disconnected token is often equipped with a built-in screen and a keypad allowing a user to type in the knowledge factor and view the OTP on the screen.  
%
Two main categories of disconnected tokens are (i) dedicated hardware-based token and (ii) mobile phone-based token. 
%
RSA SecureID \cite{secureID}, OneSpan Digipass 770 \cite{Digipass-website}, and Thales Gemalto SWYS QR Token  \cite{Gemalto} fall in the former category. 



In RSA SecureID, the OTP is generated using the current time and a secret key (allocated to the user and) stored in the token \cite{biryukov2003cryptanalysis}. Thus, not only the token has to have a synchronised clock with the server, also the token's OTP can be generated by an adversary having physical access to the token.%, as it can extract the token's key.  

The main advantage of  Digipass 770 and Thales Gemalto SWYS QR Token to RSA SecureID is that they allow users to see and verify the transaction details through the token. Therefore, the user is given more understandable information about the transaction it is approving,
so phishing (by man-in-the-browser attacks or social engineering attacks) becomes harder. Our investigation suggests that Digipass 770 and Thales Gemalto SWYS QR Token also \emph{locally stores and verify} users' PINs. 




%
% later, when the client enters a PIN, the token locally verifies the PIN and if it accepts the PIN, then it generates an OTP. 
%Specifically, once a user receives the token, it also receives an activation code from the verifier, \eg, the user's bank.  Then, the user (i) registers the activation code in the token and (ii) registers the activation code to the verifier, so the verifier knows that this specific user has a token with the provided activation code. Then, the user registers its PIN in the token which stores it locally. Every time a user uses the verifier's online system  (e.g., online banking) and makes a transaction, the system generates and displays an encrypted visual image. The user uses the token (camera) to scan the image and then enters its PIN into the token. 
%
%Next, the token checks the PIN; if the PIN matches the previously registered PIN, then it decrypts the image and displays the transaction's content on the token's screen which allows the user to check whether the transaction is the one it has made. If the user accepts the transaction and presses a certain button, then the token generates and displays an OTP that the user can type into the verifier's online system~\cite{Digipass-website}.  Thales Gemalto SWYS QR Token Eco also uses a mechanism similar to the one we described above. 

%Thus, in the above solutions, an attacker who has physical access to the token can extract the PIN and impersonate the client. 

%Jules et al. \cite{juels2016configurable} discussed that the adversary who can intercept the user and server's communication and also has physical access to the user's token or the server's storage can extract the user's PIN and impersonate the user. To address the issue they also suggested a solution that can address the above issue by using (i) a forward-secure pseudorandom number generation, (ii) multiple servers, etc. Yet, the proposed scheme lacks formal proof and neglects situations where users must verify transaction details on the token.

Matsuo \textit{et al.} \cite{MatsuoMY11} proposed an efficient symmetric-key cryptography scheme, but it relies on a secure chipset to store a secret key, which differs from most schemes, including ours. Also, their scheme does not address offline dictionary attacks and assumes the server is never breached. 
%
Jarecki \textit{et al.} \cite{JareckiJKSS21} introduced a protocol for single-server scenarios, safeguarding against PIN extraction and user impersonation even if the server or token is compromised. It relies on symmetric and asymmetric encryption and  hash functions.


%proposed a (single server) protocol to ensure that even if the server or the token is corrupted a user's PIN cannot be extracted and the adversary cannot impersonate an honest user. It is mainly based on a hash function, both symmetric and asymmetric-key encryptions, and (Diffie–Hellman) key exchange. 

%This scheme suffers from several shortcomings; namely, (1) it imposes a high computation and communication cost due to its complexity,  the use of public-key cryptography, and numerous rounds of communication, even between the user's computer and token, (2) it requires the token to perform asymmetric-key operations and invoke symmetric-key primitives many times, (3) it mandates users to input their passwords/PINs into their own (personal) computers (referred to as client $C$ in the paper) instead of entering them into the designated hardware token. 



This scheme has multiple drawbacks: (1) it incurs high computation and communication costs due to its complexity, and numerous communication rounds, (2) it demands the token to perform multiple asymmetric-key operations and numerous symmetric-key operations, (3) it obliges users to enter passwords/PINs on their personal computers instead of using the designated hardware token. %This poses a concern since the PINs become more vulnerable to exposure by attackers. This is due to the fact that users' computers are nearly always connected to the Internet, serve multiple functions, and are more susceptible to unauthorised access, (4) to maintain the security of the protocol, it is necessary for users' personal computers to be fully trusted, especially in situations where the token or the server is compromised. However, this introduces an additional assumption that may not always be desirable.


%for the protocol's security to hold, it is required that users personal computers to be fully trusted, in the case where the token or the server is corrupted; this is an additional assumption that may not be always desirable. %In contrast, our protocol does not suffer from the above shortcomings. 
%
%This protocol requires the user (in addition to remembering its PIN) to remember/store a cryptographic secret key locally (but not on the token), as a result of invoking a subroutine called asymmetric  ``password-authenticated key exchange'' (PAKE). 


%Moreover, an alternative authentication protocol, presented in  \cite{zhang2020strong}, operates independently of a trusted chipset. However, it has been specifically tailored for ``federated identity systems'' and thus is not well-suited for two/multi-factor authentication scenarios.

%Furthermore, there is another authentication protocol, that does not rely on a trusted chipset, presented in \cite{zhang2020strong}. Nevertheless, it has been designed for ``federated identity systems'' and is not suitable for two/multi-factor authentication settings. 



%Thus, unlike the above solutions, offer all the following features simultaneously; it (a) efficient, (b) is provably secure, (c) is resilient against the adversary which may corrupt the server or have access to the client's token, (d) relies on a single server, and (e) allows the client to also check the transaction detail on its token. 




The solutions introduced in \cite{SARA22,KoganMB17,KonothFFARB20}
fall in the later category, i.e., mobile phone-based token. In these solutions, a mobile phone serves as a hardware token to generate a OTP. These solutions frequently leverage the additional capabilities inherent in mobile phones, including features like a ``trusted execution environment'', direct server communication, and a rechargeable battery. 

Mobile phone-based OTP tokens have limitations: they require network coverage, may not be suitable for all users due to privacy concerns when sharing phone numbers, and not everyone owns a smartphone or is willing to install special software on it. Moreover, smartphones have a large attack surface that can be exploited to extract authentication secrets. 
%
For major banks, it is often not commercially viable to refuse service to customers unable or unwilling to use smartphone-based authentication, and in countries that impose a universal service obligation on certain banks (e.g., the United Kingdom) it would not be legal to do so.
Therefore, it is established practice for banks to offer dedicated hardware-based authentication tokens, at least to customers who request these.

%
%\end{enumerate}




%
%. Disconnected tokens are the most common type of security token used (usually in combination with a password) in two-factor authentication for online identification



% We refer readers to \cite{SinigagliaCCZ20} for a survey of different methods of authentications and their adoptions by banks.  




%Therefore, instead of using only a (clientname and) password/PIN, clients need to prove the possession of an additional device and/or an inherent factor. 
%
%!TEX root = main.tex


\section{Definition and Construction of Forward-Secure Pseudorandom Bit Generator and Keyed Pseudorandom Function}\label{app:def-FS-PRG}

In this section, we restate the formal definition of the forward-secure pseudorandom bit generator (taken from \cite{BellareY03}), briefly explain how it can be constructed, and also define a keyed pseudorandom function \prf. A standard pseudorandom generator is said to be secure if its output is computationally indistinguishable from a random string of the same length. 

However, the forward security of a stateful generator requires more security guarantees. Specifically, in this setting,  adversary $\mathcal{A}$ may at some point penetrate the machine in which the state is stored and obtain the current state. In this case, the adversary is able to compute the future output of the generator. But, it is required that the bit strings generated in the past still be secure, i.e., the strings are computationally indistinguishable from random bit strings. This implies that it is computationally infeasible for the adversary to recover the previous state from the current one. 





In this setting, the adversary is allowed to choose when it wants to penetrate the machine, as a function of the output blocks it has seen so far. Thus, first, the adversary runs in a ``find'' stage where it is fed output blocks, one at a time, until it says it wants to break in, and at that time the current state is returned.  Next, in the ``guess'' stage, it must decide if the output blocks that were given to it were the outputs of the generator, or were independent random bits. This is captured formally by two experiments; namely, real and random. In the real experiment, the forward secure generator is used to generate output blocks. 

Nevertheless, in the ideal experiment, the output blocks are truly random strings (of the same length as that of the blocks in the real experiment). Note that below ``$\mathcal{A}(\text{find}, out, h)$'' denotes $\mathcal{A}$ in the find stage, and is given an output block $out$ and current history $h$ and returns a pair $(I, h)$ where $h$ is an updated history and $I \in\{\text{find}, \text{guess}\}$. Below, we restate the two experiments. 






%This stage continues until $I = \text{guess}$ or all $n$ output blocks have been generated. In the latter case, the adversary is given the final state in the guess phase. 


%%%%%%% REAL %%%%%%
%{\small{
%\begin{center}
%\begin{mybox}[colback=white,  width=65mm, height=48mm, left=-1mm, drop fuzzy shadow southwest]{$\mathsf{Exp}_{\sss \text{real, aux}}^{\sss \text{fs-prg}}(\mathcal{A})$}
%$$
%  \begin{array}{l}
% {st}_{\sss 0}\stackrel{\sss \$}\leftarrow\kgen(1^{\sss \lambda}) \\
%%
%i\gets 0; h\gets \text{aux} \\
%\text {Repeat}\\
%i\gets i+1\\
%\hspace{4mm} ({out}_{\sss i}, st_{\sss i})\leftarrow\mathsf{GEN.next}(st_{\sss i-1})\\
%\hspace{4mm}  (I,h)\gets\mathcal{A}(\text{find}, out_{\sss i}, h)\\
%\text{Until}\hspace{2mm}  (I=guess)\hspace{2mm}  \text{or} \hspace{2mm}  (i=n)\\
%g\gets\mathcal{A}(\text{guess}, st_{\sss i}, h)\\
%\text{Return} g\\
%   \end{array} 
%$$
%\end{mybox}
%\end{center}
%%}}
%
%%%%%%%%%%. IDEAL %%%%%%
%\begin{center}
%\begin{mybox}[colback=white,  width=65mm, height=54mm, left=-1mm, drop fuzzy shadow southwest]{$\mathsf{Exp}_{\sss \text{ideal, aux}}^{\sss \text{fs-prg}}(\mathcal{A})$}
%$$
%  \begin{array}{l}
% {st}_{\sss 0}\stackrel{\sss \$}\leftarrow\kgen(1^{\sss \lambda}) \\
%%
%i\gets 0; h\gets \text{aux} \\
%\text {Repeat}\\
%i\gets i+1\\
%\hspace{4mm} ({out}_{\sss i}, st_{\sss i})\leftarrow\mathsf{GEN.next}(st_{\sss i-1})\\
%\hspace{4mm}{out}_{\sss i}\stackrel{\sss \$}\leftarrow\{0, 1\}\\
%\hspace{4mm}  (I,h)\gets\mathcal{A}(\text{find}, out_{\sss i}, h)\\
%\text{Until}\hspace{2mm}  (I=guess)\hspace{2mm}  \text{or}\hspace{2mm}  (i=n)\\
%g\gets\mathcal{A}(\text{guess}, st_{\sss i}, h)\\
%\text{Return} g\\
%   \end{array} 
%$$
%\end{mybox}
%\end{center}


\

\begin{minipage}{57mm}
\begin{tcolorbox}[left=0mm]
$
  %\left[ 
  \begin{array}{l}
%
\underline{\mathsf{Exp}_{\sss \text{real}}^{\sss \text{fs-prg}}(\mathcal{A},\text{aux})}\\
%
\\
%
{st}_{\sss 0}\stackrel{\sss \$}\leftarrow \mathsf{FS\text{-}PRG}.\kgen(1^{\sss \lambda}) \\
%
i\gets 0\\ h\gets \text{aux} \\
\text {Repeat}\\
i\gets i+1\\
\hspace{4mm} ({out}_{\sss i}, st_{\sss i})\leftarrow    \mathsf{FS\text{-}PRG.next}    (st_{\sss i-1})\\
\hspace{4mm}  (I, h)\gets\mathcal{A}(\text{find}, out_{\sss i}, h)\\
\text{Until}\hspace{2mm}  (I=guess)\hspace{2mm}  \text{or} \hspace{2mm}  (i=n)\\
g\gets\mathcal{A}(\text{guess}, st_{\sss i}, h)\\
\text{Return} g\\
  \end{array}%\right] 
  \qquad$
  \end{tcolorbox}
   \end{minipage}
   \begin{minipage}{57mm}
\begin{tcolorbox}[left=0mm]
$
  %%%%%%%%%%%%%%%%%%%%
  %\left[
  \begin{array}{l}
  %
 \underline{\mathsf{Exp}_{\sss \text{ideal}}^{\sss \text{fs-prg}}(\mathcal{A}, \text{aux})}\\
  %
  \\
  %
   {st}_{\sss 0}\stackrel{\sss \$}\leftarrow \mathsf{FS\text{-}PRG}.\kgen (1^{\sss \lambda}) \\
%
i\gets 0; h\gets \text{aux} \\
\text {Repeat}\\
i\gets i+1\\
\hspace{4mm} ({out}_{\sss i}, st_{\sss i})\leftarrow  \mathsf{FS\text{-}PRG.next}   (st_{\sss i-1})\\
\hspace{4mm}{out}_{\sss i}\stackrel{\sss \$}\leftarrow\{0, 1\}\\
\hspace{4mm}  (I,h)\gets\mathcal{A}(\text{find}, out_{\sss i}, h)\\
\text{Until}\hspace{2mm}  (I=guess)\hspace{2mm}  \text{or}\hspace{2mm}  (i=n)\\
g\gets\mathcal{A}(\text{guess}, st_{\sss i}, h)\\
\text{Return} g\\
  \end{array}%\right]
$
\end{tcolorbox}
   \end{minipage}
   
\



Given the experiments, the adversary's advantages are defined in the following two equations. 


\begin{equation}\label{equ::adv-1st-term}
\mathtt{Adv}^{\sss \text{fs-prg}}(\mathcal{A})=Pr[\mathsf{Exp}_{\sss \text{real}}^{\sss \text{fs-prg}}(\mathcal{A},\text{aux})=1]-Pr[\mathsf{Exp}_{\sss \text{ideal}}^{\sss \text{fs-prg}}(\mathcal{A},\text{aux})=1]
\end{equation}


\begin{equation}\label{equ::adv-2nd-term}
\mathtt{Adv}^{\sss \text{fs-prg}}(t)= Max\{\mathtt{Adv}^{\sss \text{fs-prg}}(\mathcal{A})\}
\end{equation}


Equation \ref{equ::adv-1st-term}  refers to the (fs-prg) advantage of $\mathcal{A}$ in attacking the forward-secure pseudorandom bit generator, FS-PRG. Moreover, Equation \ref{equ::adv-2nd-term} refers to the maximum advantage of $\mathcal{A}$ in attacking FS-PRG, where the adversary has a time-complexity at most $t$. It is required that the adversary's advantage is negligible for practical values of $t$. 




Bellare \textit{et al.} \cite{BellareY03} proposed various instantiations of FS-PRG, including the one based on AES. In the latter case, one can set a block size $b$ and a state size $s$ to $128$ bits. We refer readers to \cite{BellareY03} for further discussion. 

%In the latter case, $\mathsf{GEN.next}$ computes the next state by AES-CTR mode
%encrypting $x+s$ zero-bits with a starting counter of $0$, under the key
%$\mathsf{St}_{i-1}$. The first $x$ bits of the ciphertext is $\mathsf{Out}_i$, and the next $s$
%bits are $\mathsf{St}_i$. For $x=s=128$ using AES-128, $\mathsf{GEN.next}$ requires two
%computations of the AES encryption under one key for each update to the
%state and meets the security definitions of forward-secure key updating
%for $n < 2^{\sss 64}$.

\

\begin{definition} Let $\prf:\{0,1\}^{\sss\psi}\times \{0,1\}^{\sss \eta}\rightarrow \{0,1\}^{\sss  \lambda}$ be an efficient  keyed function. It is said $\prf$ is a pseudorandom function if for all probabilistic polynomial-time distinguishers $B$, there is a negligible function, $\mu(.)$, such that:
%

\begin{equation*}
\bigg | \Pr[B^{\sss \prf_{\hat{k}}(.)}(1^{\sss \psi})=1]- \Pr[B^{\sss \omega(.)}(1^{\sss \psi})=1] \bigg |\leq\mu(\psi)
\end{equation*}
%
where  the key, $\hat{k}\stackrel{\sss\$}\leftarrow\{0,1\}^{\sss\psi}$, is chosen uniformly at random and $\omega$ is chosen uniformly at random from the set of functions mapping $\eta$-bit strings to $\iota$-bit strings. We define $Adv^{\sss\prf}(\adv)$ as the advantage of the adversary which interacts with pseudorandom and random functions. 

%We let public parameters $\zeta:(\psi,\eta, \iota)$ be the description of $\mathtt{PRF}$.
\end{definition}

Since a pseudorandom function is deterministic and outputs the same value if queried twice on the same inputs, when proving a protocol that uses a $\prf$, it is assumed that the distinguisher never queries oracles $\prf$ and $\omega$ twice on the same inputs \cite{KatzLindell2014}. 

\clearpage


%
%!TEX root = main.tex



\section{Synchronisation}
\label{app:synchronisation}

A user's device needs to be synchronised with the server
in order for the server to check the correctness of the response generated by the device.
This is particularly the case in our proposed protocol because
if one side advances too far, it is by design impossible for it to
move backwards. Specifically, we must provide assurance that the server state remains at the same state as the device's state, or that the server is ahead of the device, i.e., $\VS{\counter} \geq \VC{\counter}$. Then, as challenge messages always contain the current value of the server's counter, the device is always able to catch up with the server. We achieve this via three approaches.  Firstly, by requiring the FS-PRG's state to advance with the counter, such that the counter is consistent with the state. Secondly, by requiring that the device never advances its state directly, but only advances to the point that the server currently is at. Thirdly, by requiring the device only to advance its state in response to an authenticated challenge from the server.%; hence, an adversary is not able to advance the state ahead of what the server intends.


The protocol takes into account the case where messages are dropped.
Response messages are not involved in advancing the forward-secure state; therefore, if these messages are dropped, then it would not have any effect on synchronisation. However, challenge messages are important, if any of them is dropped, then the device would not advance the state and would be behind the server. Nevertheless, this would not cause any issue, because the server's next challenge message will include the new value of the counter and the device will advance the state until it matches the server's state. Note that 
the FS-PRG advance process is fast; thus,  multiple invocations of this will not create a noticeable delay.
%
 Note that in the case where the enrolment's response message is dropped, the PIN will remain unchanged; as a result,  the user may be surprised that the new PIN does not work. But, the old PIN will keep working and enrolment can be repeated to update the PIN.


%!TEX root = main.tex

%\vspace{-1mm}
\section{Informal Security Analysis}
\label{sec:security}

In this section, we \textit{informally} analyze the security of the proposed protocol. We analyze its security through a set of key scenarios defined in terms of adversary capabilities and protection goals.
The scenarios are designed to assume a strong adversary so that the results are generalizable to other situations but are constrained to make sense, e.g., we assume that at least one factor is secure.


%In addition to the threats listed, we assume the adversary has access to previous authentication messages and the challenge message to any ongoing authentication.
%\vspace{-1mm}
\subsection{Threats and Protection Objectives}

In this section, we outline the threats our protocol must resist. 

%Then, we present a few \emph{high-level} protection objectives that our protocol must achieve. The threats are as follows. 

%\paragraph{T.DEV: token access} adversaries may steal the authentication token. The adversary will then know \VC{k}, \VC{\salt} and the current values of \VC{\counter} and \VC{\state}, because we assume the token does not take advantage of tamper-proof hardware. Nevertheless, such information is not sufficient to help the adversary learn \VC{\verifier}, \VM{\trans}, \VC{\pin}, \VC{\keyt_{\sss 1}}, \VC{\keyt_{\sss 2}}, \VC{\keyt_{\sss 3}} or previous values of \VC{\state} ; these are all discarded at the end of a protocol exchange. We assume that the client will not use the token after it has been stolen, and will be issued with a replacement.

\begin{itemize}[leftmargin=5.2mm]
\item[$\bullet$]\underline{\textit{T.DEV: token access.}} An adversary may steal the authentication token,  gaining access to \VC{k}, \VC{\salt} and the current values of \VC{\counter} and \VC{\state} as we assume that the token does not rely on a trusted chipset. 


\item[$\bullet$]\underline{\textit{T.MITM: Man-in-the-middle.}} An adversary may have access to the traffic exchanged between the client and server.

\item[$\bullet$]\underline{\textit{T.PIN: Knowledge of PIN.}} An adversary may know the PIN entered by a user, for example from observing them type it in.

\item[$\bullet$]\underline{\textit{T.SRV: Server compromise.}}  The server, being the party that relies on authentication, should not be considered entirely malicious, i.e., an active adversary. However, it is reasonable to assume that the server's database could be compromised, potentially revealing \VS{k}, \VS{\verifier}, and the current values of \VC{\counter} and \VC{\state}.


\end{itemize}

%\subsection{Protection objective}

%Next, we present the high-level security objective that our protocol must achieve. 

Next, we outline our protocol's key security objective.

\begin{itemize}[leftmargin=5.2mm]
\item[$\bullet$]\underline{\textit{O.AUTH: Authentication.}} If the server considers the authentication to have succeeded,  then the correct token was used and the correct PIN was entered.

\item[$\bullet$]\underline{\textit{O.TRAN: Transaction authentication.}} If the server considers the authentication to have succeeded, then the correct token was used, the correct PIN was entered, and the token showed the correct transaction.

\item[$\bullet$]\underline{\textit{O.PIN: PIN protection.}} The adversary should not be able to discover the user's PIN. 
\end{itemize}


%\vspace{-3mm}
\subsection{Scenarios}
In this section, we briefly explain why the protocol meets its objective in different threat scenarios. 

\begin{enumerate}[leftmargin=5.5mm]

\item\underline{\textit{O.AUTH against 
T.PIN and T.MITM.}}  In this case, the adversary does not have
access to the authentication token but does know the user's PIN and communication between the client and server.  For the adversary to perform a successful authentication, it must compute $\prf_{\sss\VC{\keyt_{\sss 3}}}(\VM{\nonce} \|  \VM{\trans} \| \VC{\verifier}|| 1)$. Nevertheless, it does not know \VC{\keyt_{\sss 3}} or the state from which \VC{\keyt_{\sss 3}} has been generated. Since  \VC{\keyt_{\sss 3}} is an output of $\prf$ and is sufficiently large, it is computationally indistinguishable from a truly random value. Thus,  the probability of finding \VC{\keyt_{\sss 3}} is negligible in the security parameter.  The adversary may also try to guess the truncated response. However, its probability of success is at most $\frac{1}{\lambda}$, where $\lambda$ is the bit-length of the truncated response.\footnote{For instance, for $8$-digit response its probability of success is $\frac{1}{2^{\sss 28}}$.}  
%
Thus, the only party that will generate a valid response is the token itself (when the PIN is provided) at line \ref{auth:gen-res} of \prettyref{fig:auth}.  We have already assumed that the adversary does not have access to the token. Therefore, it cannot generate a valid response.%\footnote{This argument also captures replay attacks where the adversary may reuse the gathered information.}

\item\underline{\textit{O.AUTH against T.DEV and T.MITM.}}\label{O.AUTH:T.DEV-plus-T.MITM}
In this scenario, the adversary has compromised the user's token (but not its PIN), has records of previous messages, and wishes to impersonate the user. Note, in this case, the adversary does not learn \VC{\verifier}, \VM{\trans}, \VC{\pin}, \VC{\keyt_{\sss 1}}, \VC{\keyt_{\sss 2}}, \VC{\keyt_{\sss 3}} or previous values of \VC{\state}. As these are all discarded when the protocol terminates, i.e., when $\mathtt{Discard}(.)$ is executed. 
%
%As with the previous scenario, the adversary may either re-use a legitimate response, forge one, or infer the response from other sources.
Since the random challenge in the expected response is unique, and the $\prf$ provides an unpredictable output, previous responses will not be valid. Hence, a \textbf{replay attack would not work}. The adversary can use the token to discover the parameters of the response message, except the PIN. In this case, it has to perform an online dictionary attack by guessing a PIN, using the extracted parameters to generate a response, and sending the response to the server. However, the server will lock out the token if the number of incorrect guesses exceeds the predefined threshold. Other places where the PIN is used are in (i) the enrolment response, where the verifier derived from the PIN is encrypted under an evolving fresh secret key, and (ii) the authentication response, where the response is a pseudorandom value derived from the PIN's verifier using an evolving fresh secret key. In both cases, the evolving keys cannot be obtained from the current state, due to the security of FS-PRG. 



%\paragraph{Enrollment response} For the case of the encrypted enrollment response, consider an adversary \adv{} which is given $p$ plaintexts $P_1 \ldots P_p$ corresponding to the $p$ possible PINs and one ciphertext $C$, corresponding to the encrypted enrollment message.
%Let us assume this adversary is able to output the index of the plaintext that is the decryption of $C$, with better than random probability.
%As the key used in the enrollment protocol is only used once, the adversary does require access to an encryption or decryption oracle.
%
%Now consider an CPA-secure encryption scheme \enc. 
%The adversary \bdv{} generates two plaintexts $P_1$ and $P_2$ and requests $C=\enc(P_j)$ from the sender where $j$ is selected randomly by the sender.
%\bdv{} sends $P_1$, $P_2$ and $C$ to \adv, which will provide the value of $j$ selected by the sender, with better than random probability.
%Therefore if an adversary exists that can guess the PIN from an enrollment message, then this implies that the encryption scheme used is not CPA secure.
%
%\paragraph{Authentication response} The other place where the verifier is used is in the authentication response, as part of the input of the \prf{}.
%The only information the adversary receives about the verifier is the single output of the PRF.
%The verifier is not used elsewhere, nor is the key used for the PRF \keyt2.
%Importantly, the adversary is not able to generate the PRF for any potential verifier.
%The adversary therefore learns nothing more than a single point in the domain of \prf that is indistinguishable from random and therefore leaks no information about the verifier and consequently the PIN.

%\paragraph{3: Man-in-the-browser}
%T.PIN, T.MITM: O.AUTH.
%The adversary is performing a man-in-the-middle attack and wishes to perform authentication.
%The client is performing authentication, but does not want to perform the transaction that the adversary desires.
%From the definition of existential forgery, we know that the adversary cannot generate an authentication response that the genuine authentication token generated.
%The token will always show the transaction to the client before generating a response containing that transaction, therefore the adversary cannot generate an acceptable response that was not shown to the client.

\item\underline{\textit{O.PIN against T.DEV and  T.MITM.}}
The adversary has compromised the token and wishes to obtain the user's PIN. As with Scenario \ref{O.AUTH:T.DEV-plus-T.MITM}, the PIN cannot be obtained from the token, the responses in the authentication or enrolment phases.

\item\underline{\textit{O.PIN against T.SRV and T.MITM.}}
\label{sec:servercompromise}
The adversary has compromised the server and wishes to obtain the user's PIN.
In this case, the adversary has learned the verifier but does not know the value of the secret key, used to generate the verifier. If the server retains values of the verifier for previous PINs (in the case where the server does not delete them), then the adversary would also learn further verifiers for the same token. The PIN is only used for computing the verifier, so the only way to obtain the PIN would be to find the key of the $\prf$ which is not feasible except for a negligible probability in the security parameter. The only information this discloses is that if two values for the verifier are equal, then that implies that two PINs for the same token were equal. Even this minimal information leakage can be removed if the server rejects the PINs that were used before. 

\item\underline{\textit{O.TRAN against T.PIN and T.MITM or T.DEV and T.MITM.}} As we discussed above, an adversary cannot successfully authenticate, even if it sees the traffic between the client and server and has access to either the PIN or the token. Furthermore, due to the security of the authenticated encryption, the token can detect, if the transaction's description, that the server sends to it, has been tampered with. 
 
\end{enumerate}

%\vspace{-2mm}
\subsection{Excluded Scenarios}
%\vspace{-1mm}

We exclude some scenarios (for instance, because they do not make sense or are very hard to efficiently deal with without introducing additional trusted third parties). 

\begin{itemize}[leftmargin=5.3mm]
\item[$\bullet$]\textit{Compromised PIN and token.}
If the adversary has compromised both factors of a 2FA protocol, then the server cannot distinguish between the adversary and the legitimate user.

\item[$\bullet$]\textit{Authentication on compromised server.}
If the adversary has compromised the server, then it can either directly perform actions of the server or change keys to ones known by the adversary. Hence, it does not make sense to aim for O.AUTH in this situation.

\item[$\bullet$]\textit{Compromised server and token.}
If the adversary has compromised the server and the token, then the PIN can be trivially brute-forced with knowledge of \VS{\verifier} and \VC{\salt}. %We consider this scenario to be unrealistic.% because the server should be well protected and physically compromising a large number of tokens should be challenging.
\end{itemize}

We highlight that the existing 2FA schemes also exclude the above scenarios from their threat models. 

%!TEX root = main.tex

\section{Formal Security Analysis}\label{sec::Formal-Security-Analysis}

In this section, we present the security proof of the protocol, presented in Section \ref{sec::the-protocol}.  First, we prove the semantic security of the scheme and then prove its authentication. 

\subsection{Semantic Security}\label{sec::semSec-proof}

In this section, we assert that under standard assumptions protocol $\psi$, presented in Section \ref{sec::the-protocol}, securely distributes
session keys. To do so, we incrementally define a sequence of games starting at the real
game $G_{\sss 0}$ and ending up at $G_{\sss  7}$. We first define various events in every game and then explain each game. 

\begin{itemize}
%
\item  $S_{\sss  i}$: it takes place if $b=b'$, where $b$ is the bit involved in the test query and $b'$ is the output of $\A$ which wants to guess $b$. 
%
\item $Auth_{\sss  i}$: it occurs if $\A$ generates and sends to the server an authenticator message that is accepted by the server.
%
\item $Enc_{\sss  i}$: occurs if $\A$ submits data it has encrypted by itself using the correct key that an honest party would use to encrypt. 
%
\end{itemize}

\begin{itemize}
%
\item[$\bullet$] \textit{\textbf{Game}} $G_{\sss  0}$: This is the real attack game. Several oracles are  available to the adversary; namely, the pseudorandom function ($\prf$),  the encryption/decryption oracles ($\mathtt{Enc}$ and $\mathtt{Dec}$) and all instances $U^{\sss  i}$ and $S^{\sss  j}$.
%
According to the definition we presented in Section \ref{sec::model}, the advantage of the adversary in this protocol is: 
%
\begin{equation}\label{eq::adv-ake-}
Adv_{\sss  \psi}^{\sss  ss}(\A)=2Pr[S_{\sss  0}]-1
\end{equation}

Similar to the security proof in \cite{BressonCP03}, we assume that if any of the games halts and $\A$ does not output $b'$, then $b'$  is chosen at random. Also, if $\A$ has not finished playing the game after sending $q_{\sss  s}$  \send$(.)$ queries or if it plays the game more than a predefined time $t$, the game is stopped and a random value is assigned to $b'$. 
%



\item[$\bullet$] \textit{\textbf{Game}}  $G_{\sss  1}$: This game is similar to  $G_{\sss  0}$, except that the output of the $\prf$ is replaced by an output of  a uniformly random function $f$, i.e., when the simulator in Figure \ref{fig::PRF-SIM} is used. Since the output of $f$ (in the simulator) and $\prf$ are indistinguishable, except with a negligible probability, we will have: 
%
\begin{equation}\label{eq::game_1}
|Pr[S_{\sss  1}]-Pr[S_{\sss  0}]|\leq (q_{\sss s}+q_{\sss p})Adv^{\sss\prf}(\adv)
\end{equation}

that captures both send and execute queries. We highlight that as we use a standard $\prf$, the probability of finding a collision is $0$. 



\item[$\bullet$] \textit{\textbf{Game}}  $G_{\sss  2}$: This game is the same as $G_{\sss  1}$, with the difference that we simulate the authenticated encryption scheme (i.e., $Enc$ and $Dec$ algorithms). We replace the output of $Enc$ with a uniformly random value picked from the encryption scheme's range. The adversary has access to the encryption and decryption oracles. Since we treat the encryption scheme as a black box, the two games are distinguishable except with a negligible probability; this we will have: 
 %
\begin{equation}\label{eq::game_2}
|Pr[S_{\sss  2}]-Pr[S_{\sss  1}]| \leq (q_{\sss s}+q_{\sss p}) Adv^{\sss Enc}(\adv)
\end{equation}

The above also captures both the send and execute queries. Since we have used a standard encryption scheme, the probability of finding a collision (e.g., two ciphertexts result in the same plaintext or two plaintexts result in the same ciphertext) is $0$, as the scheme is bijective. 



\item[$\bullet$] \textit{\textbf{Game}}  $G_{\sss  3}$: This game is the same as $G_{\sss  2}$, with the difference that we simulate the verification of a transaction, i.e., via predicate $\phi$ defined in Section \ref{sec::model}. Moreover, we simulate all parties' instances via defining simulators for  \send, \execute, \reveal, and \test\ queries. We present the simulators for user's and server's \send\ queries in Figures \ref{fig::Send-sim-to-client} and \ref{fig::Send-sim-to-server} respectively. Also, we present the simulators for the rest of the queries in Figure \ref{fig::sim-for-exe-rev-test}. By definition,  $\phi$ is a deterministic function, given the transaction $t^{\sss  U}$ and policy $\pi$, it always returns the same output as the user's device does when verifying $t^{\sss  U}$ in the previous game.  Therefore, both $\phi$ and the user's device would output identical values, given pair $(t^{\sss  U}, \pi)$, meaning that their outputs are indistinguishable in both games.  Given the above argument, we conclude that:  
%
\begin{equation}\label{eq::game_3}
Pr[S_{\sss  3}]-Pr[S_{\sss  2}]=0
\end{equation}
%
\item[$\bullet$] \textit{\textbf{Game}}  $G_{\sss  4}$: This game is the same as $G_{\sss  3}$, with the difference that when the adversary manages to use the correct encryption key and encrypts (or decrypts) a message itself, then the simulation aborts. Therefore, we have: 
%
\begin{equation*}
|Pr[S_{\sss  4}]-Pr[S_{\sss  3}]|\leq Pr[Enc_{\sss  4}]
\end{equation*}


We know that the key has been picked uniformly at random and is of length $\lambda$ bits (recall that the outputs of $\prf$ have been replaced with truly random values in $G_{\sss  1}$). Therefore:
%
 $$Pr[Enc_{\sss  4}]=\frac{4(q_{\sss  s}+q_{\sss  p})}{2^{\sss \lambda}}$$ and 
 %
 \begin{equation}\label{eq::game_4}
 |Pr[S_{\sss  4}]-Pr[S_{\sss  3}]|\leq \frac{4(q_{\sss  s}+q_{\sss  p})}{2^{\sss \lambda}}
 \end{equation}
%

\item[$\bullet$] \textit{\textbf{Game}}  $G_{\sss  5}$: In this game, we modify the simulator such that it would abort if
the adversary correctly guesses the authenticator. Therefore, we modify the way the server responds to  query {\send($S^{\sss  j},   \bar{\VC{\mathit{response}}}$)} as follows: 

\begin{enumerate}
%
\item computes $\VS{\mathit{expected}} \gets \prf_{\sss\bar{\VS{\keyt_{\sss3}}}}(\ddot{\VS{\nonce}}  \| \VS{\trans} \| \VS{\verifier}|| 1)$.
%
\item checks if $\bar{\VC{\mathit{response}}}=\VS{\mathit{expected}}$. It proceeds to the next step if the equation holds. 
%
\item\label{Game::check-sim} checks if $\Big((U_{\sss  ID},  \text{enrolment}), (U_{\sss  ID},  \text{authentication}), (\bar M, \bar t),$ $ (\bar M',$ $ \bar t'),$ $ (\ddot M', $ $\ddot t'), $ $(\hat M', \hat t'), \bar{\VC{\mathit{response}}}\Big)\in \vv L$.

\item\label{Game::prf} checks if $ \bar{\VC{\mathit{response}}}\in L_{\sss \adv}$. 
%
\item if both checks in steps \ref{Game::check-sim} and \ref{Game::prf} fail, then it rejects authenticator $\bar{\VC{\mathit{response}}}$ and terminates without accepting the key. Otherwise, it accepts the key. 
%
\end{enumerate}
This game ensures that if the message (\ie, the authenticator) does not come from the simulator or the adversary (which decrypted $\ddot M',\ddot t', \hat M'$, and  $\hat t'$, then correctly computed a valid authenticator by querying $\update$ and $\prf_{\sss  k'}$) then it aborts. So, games $G_{\sss  4}$ and $G_{\sss  5}$ are indistinguishable unless the server rejects a valid authenticator. However, this means the adversary has correctly guessed the output of $\prf$. Thus, 
%
\begin{equation}\label{eq::game_5}
|Pr[S_{\sss  5}]-Pr[S_{\sss  4}]|\leq \frac{q_{\sss  s}}{2^{\sss \lambda}}
\end{equation}



\item[$\bullet$] \textit{\textbf{Game}}  $G_{\sss  6}$: 
 In this game, we modify the simulator in a way that it would abort if $\adv$ decrypts $ (\ddot M', \ddot t', \hat M', \hat t')$ and uses the result to generate and send a valid authenticator to the server. To do so, we modify the way the server responds to  query {\send($S^{\sss  j},   \bar{\VC{\mathit{response}}}$)}, as follows: 

\begin{enumerate}
%
\item computes $\VS{\mathit{expected}} \gets \prf_{\sss\bar{\VS{\keyt_{\sss3}}}}(\ddot{\VS{\nonce}}  \| \VS{\trans} \| \VS{\verifier}||1)$.
%
\item checks if $\bar{\VC{\mathit{response}}}=\VS{\mathit{expected}}$. It proceeds to the next step if the equation holds. 
%
\item\label{Game::check-sim-} checks if $\Big((U_{\sss  ID},  \text{enrolment}), (U_{\sss  ID},  \text{authentication}), (\bar M, \bar t),$ $ (\bar M',$ $ \bar t'),$ $ (\ddot M', $ $\ddot t'), $ $(\hat M', \hat t'), \bar{\VC{\mathit{response}}}\Big)\in \vv L$. If this check fails, then it rejects authenticator  $\bar{\VC{\mathit{response}}}$ and terminates, without accepting any key.

\item\label{Game::prf-} checks if $(\ddot{\VS{\nonce}}  \| \VS{\trans} \| *,\bar{\VC{\mathit{response}}})\in L_{\sss \adv}$.  
%
\item aborts, if the above check (in step \ref{Game::prf-}) passes.
%
\end{enumerate}

The above modification ensures that all valid authenticators are sent by the simulator. Let $\hat {Auth}_{\sss  6}$ be the event that the check in step \ref{Game::prf-} passes. Games $G_{\sss  5}$ and $G_{\sss  6}$ are indistinguishable unless $\hat {Auth}_{\sss  6}$ occurs. Hence,
%
$$|Pr[S_{\sss  6}]-Pr[S_{\sss  5}]|\leq Pr[\hat {Auth}_{\sss  6}]$$


We know that $\hat {Auth}_{\sss  6}$ occurs with probability $ \frac{q_{\sss  s}}{2^{\sss \lambda}}$ when the query $q=\ddot{\VS{\nonce}}  \| \VS{\trans} \| *$ to $\prf$ results in $\bar{\VC{\mathit{response}}}$. Thus, 
%
\begin{equation}\label{eq::game_6}
|Pr[S_{\sss  6}]-Pr[S_{\sss  5}]|\leq \frac{q_{\sss  s}}{2^{\sss \lambda}}
\end{equation}

\item[$\bullet$] \textit{\textbf{Game}}  $G_{\sss  7}$: In this game, we modify the simulator such that it would abort if the adversary comes up with the authenticator and session key without decrypting  $ (\ddot M', \ddot t', \hat M', \hat t')$.  Therefore, we modify  the way the user's device processes query \send($U^{\sss  i}, (\ddot M', \ddot t'), (\hat M', \hat t')$) as follows.
 %
 \begin{enumerate}
 \item compute $\bar{\VC{\mathit{response}}} \gets \prf_{\sss  \ddot t'}( \ddot M'|| 1)$.
 \item compute $\bar {sk}^{\sss U}\gets \prf_{\sss  \ddot t'}( \ddot M'|| 2)$.
 \end{enumerate}

We also amend the way the server compiles query \send($S^{\sss  j},   \bar{\VC{\mathit{response}}}$) as follows. 
%
\begin{enumerate}[label=\alph*]
%
\item\label{Game::prf--}  checks if  $(\ddot M'||1, \bar{\VC{\mathit{response}}})\in L_{\sss \adv}$ or  $(\ddot M'||2, \bar {sk}^{\sss U})\in L_{\sss \adv}$. 
%
\item aborts, if either of the above checks (in step \ref{Game::prf--}) passes.
%
\end{enumerate}



 Let $\hat {Auth}_{\sss  7}$ be the event that the check in step \ref{Game::prf--} passes. Games $G_{\sss  6}$ and $G_{\sss 7}$ are indistinguishable unless $\hat {Auth}_{\sss  7}$ occurs. Therefore,
%
$$|Pr[S_{\sss  7}]-Pr[S_{\sss  6}]|\leq Pr[\hat {Auth}_{\sss  7}]$$


Event $\hat {Auth}_{\sss  7}$ occurs with probability $ \frac{2q_{\sss  s}}{2^{\sss \lambda}}$ when the query $q=\ddot M'||1$ to $\prf$ results in $\bar{\VC{\mathit{response}}}$ or $q=\ddot M'||2$ to $\prf$ results in $\bar {sk}^{\sss U}$. Thus, 
%
\begin{equation}\label{eq::game_7}
|Pr[S_{\sss  7}]-Pr[S_{\sss  6}]|\leq \frac{2q_{\sss  s}}{2^{\sss \lambda}}
\end{equation}

Moreover, the session key and authenticator are random values, as they are the outputs of $\prf$ whose secrete key is not known. Therefore, $Pr[S_{\sss  7}]=\frac{1}{2}$. 



By summing up all the above relations \ref{eq::game_1}-\ref{eq::game_7}, we would have 
%
\begin{equation}\label{equ::sum}
|Pr[S_{\sss  7}]-Pr[S_{\sss  0}]|\leq (q_{\sss s}+q_{\sss p})\Big(Adv^{\sss\prf}(\adv)+Adv^{\sss Enc}(\adv)\Big)+\frac{4(q_{\sss  s}+q_{\sss  p})}{2^{\sss \lambda}}+\frac{4q_{\sss  s}}{2^{\sss \lambda}}
\end{equation}

By combining Equations \ref{eq::adv-ake-} and \ref{equ::sum}, we would have: 
%
\begin{equation*} 
Adv_{\sss  \psi}^{\sss  ss}(\A) \leq 2(q_{\sss s}+q_{\sss p})\Big(Adv^{\sss\prf}(\adv)+Adv^{\sss Enc}(\adv)\Big)+\frac{8(2q_{\sss  s}+q_{\sss  p})}{2^{\sss \lambda}}
\end{equation*}

This completes the proof. 


%\
%
%\item[$\bullet$] \textit{\textbf{Game}}  $G_2$: In this game, we simulate the $\prf$ and all the instances as the real players would do for the send, execute, reveal, and test queries. 

 
%Since the pseudorandom function is deterministic and outputs the same value if queried twice on the same inputs, without loss of generality we assume a distinguisher never queries the pseudorandom oracle twice on the same inputs.

%%%%
\end{itemize}

%!TEX root = main.tex


%%%%%%%%%%%%%%%%%%%%%%%%%%%%%%%%%%%%%%
\begin{figure}[!hpt]
%\setlength{\fboxsep}{1pt}
\begin{center}
    \begin{tcolorbox}[enhanced,width=3.3in, left=.1cm,
    drop fuzzy shadow southwest,
    colframe=black,colback=white]
{\small{
\doubleunderline{Pseudorandom function}\\
%The simulator maintains a history of already answered queries and acts as follows. 

The simulator upon receiving query $(\prf,q)$ acts as follows. 

\begin{itemize}
%\item  if record $(q,r)$ is in list $L_{\prf}$, then output $r$. 
 \item picks a function $f$, i.e., $f\stackrel{\sss \$}\leftarrow Func$, where $Func$ is the set of all functions mapping $|q|$-bit strings to $|q|$-bit strings. 
 \item adds record $(q,f(r))$ to list $L_{\sss\adv}$ and then outputs $f(r)$.  
 \end{itemize}
 
 
}}
\end{tcolorbox}
\end{center}
%\vspace{-4mm}
\caption{Pseudorandom function's simulator.} 
\label{fig::PRF-SIM}
\end{figure}





%%%%%%%%%%%%%%%%%%%%%%%%%%%%%%%%%%%%%%%%
\begin{figure}[H]
\setlength{\fboxsep}{1pt}
\begin{center}
    \begin{tcolorbox}[enhanced,width=3.3in,left=0.1cm, 
    drop fuzzy shadow southwest,
    colframe=black,colback=white]
{\small{
%\underline{Pseudorandom function}\\
%%The simulator maintains a history of already answered queries and acts as follows. 
%The simulator upon receiving query $(\prf,q)$ acts as follows. 
%
%\begin{itemize}
%%\item  if record $(q,r)$ is in list $L_{\prf}$, then output $r$. 
% \item picks a function $f$, i.e., $f\stackrel{\sss \$}\leftarrow Func$, where $Func$ is the set of all functions mapping $|q|$-bit strings to $|q|$-bit strings. 
% \item  output $f(r)$. %Next, add record $(q,r)$ to $L_{\prf}$. 
% \end{itemize}
% 
% 
% \noindent\rule{5.1in}{1pt}
 
  \doubleunderline{\send($U^{\sss i}, .$)}\\
 
 This query is dealt with as below: 
 
 \begin{itemize}[leftmargin=.4cm]
 %
 \item if the user's instance is not in the ``expecting'' state and it receives query  \send($U^{\sss i}, \text{start}, \text{phase}$), where $\text{phase}\in \{\text {enrolment, authentication}\}$ then it: 
 %
  \begin{enumerate}
  %
  \item generates pair $(U_{\sss ID},  \text{phase})$. 
  %
  \item responds to the query with $(U_{\sss ID},  \text{phase})$. 
  %
  \item sets the user's instance state to expecting.
  %
  \end{enumerate}
 % 
 \item if the user's instance state is in expecting, then: 
 %
 \begin{itemize}[leftmargin=.4cm]
 %
 \item upon receiving \underline{\send($U^{\sss i}, (\bar M, \bar t)$)}, it:
 % 
  \begin{enumerate}
  %
  \item authenticates and decrypts the ciphertext $\bar M$ as $(p, b) \gets \mathsf{Dec}_{\sss\VC{k}}(\bar M, \bar t)$. If the authentication fails (\ie, $b\neq 1$), it halts.  
  %
  \item extracts $(\VM{\nonce}, \VM{\counter})$ from plaintext $p$ and checks if $\VM{\counter} > \VC{\counter} $. If the check fails, it halts. 
  %
  \item generates $\VC{\verifier}$ using $\VC{\salt}$ and $\VC{\pin}$ as follows $\VC{\verifier} \gets \prf_{\VC{\salt}}(\VC{\pin})$. Then, it updates its state as follows: $\forall i, 1\leq i\leq \VM{\counter}- \VC{\counter}: \text{\ (a)\ } \VC{\counter} \gets \VC{\counter} + 1\text{\ and (b) \ } (k,  \VC{\state}) \gets \update(\VC{\state}, \VC{\counter})$. 
  %
  \item encrypts $p'=\VM{\nonce}||\VC{\verifier}$ using key $k$  as follows: $(\bar M', \bar t') \gets \enc_{\sss k}(p')$, which results in a ciphertext $\bar M'$ and tag $\bar t'$. 
  %
  \item responds to the query with $(\bar M', \bar t')$. It sets the user's instance state to ``not expecting''.  
  %
  \end{enumerate}
 %
 \item  upon receiving \underline{\send($U^{\sss i}, (\ddot M', \ddot t'), (\hat M', \hat t')$)}, it: 
 %
 \begin{enumerate}
 %
 \item   authenticates and decrypts the ciphertext $\hat M'$ as follows: $(p', b') \gets \mathsf{Dec}_{\sss\VC{k}}(\hat M', \hat t')$. If the authentication fails (i.e., $b'\neq 1$), it halts.  
 %
  \item extracts $(\tmp_{\sss\VM{\counter}}, \VM{\counter})$ from $p'$ and checks if $\tmp_{\sss\VM{\counter}} > \VC{\counter} $. If the check fails, it halts. 
 %
 \item updates its state as follows: $\forall i, 1\leq i\leq \tmp_{\sss\VM{\counter}}- \VC{\counter}: \text{\ (a)\ } \VC{\counter} \gets \VC{\counter} + 1\text{\ and (b) \ } (k,  \VC{\state}) \gets \update(\VC{\state}, \VC{\counter})$. 
 %
 \item  authenticates and decrypts $\ddot M'$ as follows: $(p, b) \gets \mathsf{Dec}_{\sss k}(\ddot M', \ddot t')$. If the authentication fails (\ie, $b\neq 1$), it halts.
 %
 \item    extracts $(\VM{\nonce}, \VM{\trans})$ from plaintext $p$.  
 %
 \item runs the predicate, $y \leftarrow\phi(\VM{\trans}, \gamma)$. If $y=0$, it halts. 
 %
\item generates $\VC{\verifier}$ using $\VC{\salt}$ and $\VC{\pin}$ as follows, $\VC{\verifier} \gets \prf_{\sss\VC{\salt}}(\VC{\pin})$.
 %
 \item updates its state one more time as follows, $(k',  \VC{\state}) \gets \update(\VC{\state}, \VM{\counter})$.
 %
 \item computes  the authenticator: $\bar{\VC{\mathit{response}}} \gets \prf_{\sss k'}( \VM{\nonce} || \VM{\trans} || \VC{\verifier}|| 1)$ and session key:  $\bar {sk}^{\sss U} \gets \prf_{\sss k'}( \VM{\nonce} || \VM{\trans} || \VC{\verifier}|| 2)$.
 %
 \item responds the send query with  $\bar{\VC{\mathit{response}}}$. It makes the user's instance accept the key and then terminates the instance. 
 %
 \end{enumerate}
 %
 \end{itemize}
 %
 \end{itemize}
 %
}}
To keep track of all the exchanged messages, it stores the above incoming and going messages in vector $\vv L$. So, we have $\Big((U_{\sss ID},  \text{enrolment}),$ $ (U_{\sss ID},  \text{authentication}), $ $(\bar M, \bar t), $ $(\bar M', \bar t'),$ $(\ddot M', \ddot t'), $ $(\hat M', \hat t'), $ $\bar{\VC{\mathit{response}}}\Big)\in \vv L$.
\end{tcolorbox}
\end{center}
\vspace{-3mm}
\caption{Simulators for \send\ query to  a user's instance.} 
\label{fig::Send-sim-to-client}
\end{figure}
%%%%%%%%%%%%%%%%%%%%%%%%%%%%%%%%%



%%%%%%%%%%%%%%%%%%%%%%%%%%%%%%%%%%%%%%%%
\begin{figure}[H]
\setlength{\fboxsep}{1.2pt}
\begin{center}
    \begin{tcolorbox}[enhanced,width=3.3in, left=0.1cm,
    drop fuzzy shadow southwest,
    colframe=black,colback=white]
{\small{

 \doubleunderline{\send($S^{\sss j}, .$)}\\
 
 This query is dealt with as below: 
 
 \begin{itemize}[leftmargin=.4cm]
 %
 \item upon receiving  \underline{\send($S^{\sss j}, (U_{\sss ID},  \text{enrolment})$)}, it:
 %
 \begin{enumerate}
 %
 \item  increments its counter as $\VS{\counter} \gets \VS{\counter} + 1$, updates its state as $\bar{\VS{\keyt}_{\sss 1}}, \VS{\state} \gets \update(\VS{\state}, \VS{\counter})$, picks a random value $\bar{\VS{\nonce}} \stackrel{\sss\$}\leftarrow \bin^{\sss\secpar}$, and generates ciphertext and tag   $(\bar M,\bar t) \gets \enc_{\sss\VS{k}}(\bar{\VS{\nonce}}|| \VS{\counter})$. 
 
 \item responds to the query with $(\bar M,\bar t)$. The state of the server instance is set to ``expecting''.
 %
 %\item upon receiving query  \send($S^{j}, $), 
 %
 \end{enumerate}
 %
 \item upon receiving  \underline{\send($S^{\sss j}, (\bar M',  \bar t'$))}, it:
 %
 \begin{enumerate}
 %
 \item authenticates and decrypts the ciphertext $\bar M'$ as $(p', b') \gets \mathsf{Dec}_{\sss\bar{\VS{\keyt}_{\sss 1}}}(\bar M', \bar t')$. If the authentication fails (i.e., $b'\neq 1$), it halts. 
 %
\item    extracts $(\VM{\nonce}, \VM{\verifier})$ from plaintext $p'$. 

It sets $\VS{\verifier}\gets\VM{\verifier}$ and also  checks if $\VM{\nonce}=\bar{\VS{\nonce}}$. If the equation does not hold, it halts. The state of the server instance is set to expecting.
 %
 \end{enumerate}
 %
 \item upon receiving  \underline{\send($S^{\sss j}, (U_{\sss ID},  \text{authentication})$)}, it:
 %
 \begin{enumerate}
 %
  \item  increments its counter  $\VS{\counter} \gets \VS{\counter} + 1$, updates its state  $\bar{\VS{\keyt}_{\sss 2}}, \VS{\state} \gets \update(\VS{\state}, \VS{\counter})$,  temporarily stores this counter $\tmp_{\sss\VS{\counter}} \gets \VS{\counter}$, increments the counter again $\VS{\counter} \gets \VS{\counter} + 1$, updates its state again $\bar{\VS{\keyt}_{\sss 3}}, \VS{\state} \gets \update(\VS{\state}, \VS{\counter})$, and picks a random value $\ddot{\VS{\nonce}} \stackrel{\sss\$}\leftarrow \bin^{\sss\secpar}$.
  %
  \item generates two pairs of ciphertext and tag as follows, $(\ddot M', \ddot t')\leftarrow \enc_{\sss\bar{\VS{\keyt_{\sss 2}}}}(\ddot{\VS{\nonce}} || \VS{\trans})$ and  $(\hat M', \hat t')\leftarrow \enc_{\sss\VS{k}}(\tmp_{\sss\VS{\counter}} || \VS{\counter})$. 
  %
  \item responds to the query with $(\ddot M', \ddot t')$, $(\hat M', \hat t')$. The state of the server instance is set to expecting.
 %
 \end{enumerate}
 %
 \item upon receiving  \underline{\send($S^{\sss j},   \bar{\VC{\mathit{response}}}$)}, it:
 %
 \begin{enumerate}
 %
 \item computes $\VS{\mathit{expected}} \gets \prf_{\sss\bar{\VS{\keyt_{\sss3}}}}(\ddot{\VS{\nonce}}  \| \VS{\trans} \| \VS{\verifier}|| 1)$. It checks whether  $ \bar{\VC{\mathit{response}}}=\VS{\mathit{expected}}$. If the equality does not hold, the server instance terminates without accepting any session key. 
 %
  \item generates the session key $\bar {sk}^{\sss S} \gets \prf_{\sss\bar{\VS{\keyt_{\sss3}}}}(\ddot{\VS{\nonce}}  \| \VS{\trans} \| \VS{\verifier}|| 2)$. 
 %
It accepts the key and terminates.
%
  \end{enumerate}
  %
 \end{itemize} 
 %
}}
\end{tcolorbox}
\end{center}
\caption{Simulators for \send\ query to  a server's instance.} 
\label{fig::Send-sim-to-server}
\end{figure}
%%%%%%%%%%%%%%%%%%%%%%%%%%%%%%%%

%%%%%%%%%%%%%%%%%%%%%%%%%%%%%%%%%%%%%%%%
\begin{figure}[H]
\setlength{\fboxsep}{1pt}
 \vspace{-3mm}
\begin{center}
    \begin{tcolorbox}[enhanced,width=3.3in, height=105mm, left=0.1cm, 
    drop fuzzy shadow southwest,
    colframe=black,colback=white]
     \vspace{-3mm}
{\small{

 \doubleunderline{\execute($U^{\sss i}, S^{\sss j}$)}\\
 
 This query is dealt with as below: 
 %
 \begin{enumerate} 
 %
 \item $(U_{\sss ID},  \text{enrolment})\gets$\send($U^{\sss i}, \text{start}, \text{enrolment}$).%, where $\text{phase}\in \{\text {enrollment, authentication}\}$. 
 %
\item  $(\bar M,\bar t)\gets$\send($S^{\sss j}, (U_{\sss ID},  \text{enrolment})$).
  %
\item $(\bar M', \bar t')\gets$\send($U^{\sss i}, (\bar M, \bar t)$).
  %
\item $(U_{\sss ID},  \text{authentication})\gets$\send($U^{\sss i}, \text{start}, \text{authentication}$).
  %
\item $(\ddot M', \ddot t', \hat M', \hat t')\gets$\send($S^{\sss j}, (U_{\sss ID},  \text{authentication})$).
  %
\item $\bar{\VC{\mathit{response}}}\gets$\send($U^{\sss i}, (\ddot M', \ddot t'), (\hat M', \hat t')$).
  %
  \item outputs the following transcript: $[(U_{\sss ID},  \text{enrolment}), (\bar M, \bar t), (\bar M', \bar t'),(U_{\sss ID},  $ $\text{authentication}), $ $(\ddot M', \ddot t'), (\hat M', \hat t'), \bar{\VC{\mathit{response}}}]$.
  %
 \end{enumerate} 
 %
  \noindent\rule{3.2in}{1pt}
  %
  \doubleunderline{\reveal($I$)}\\
  
This query is processed as follows.
  
\ $\bullet$ returns session key $\bar {sk}^{\sss I}$ (computed by $I\in\{U,S \}$), if $I$ has already accepted the key.
  
   %
  \noindent\rule{3.2in}{1pt}
  %
    \doubleunderline{\test($I$)}\\
    
  This query is processed as below.  
  
 \begin{enumerate} 
 %
 \item $sk\gets$\reveal($I$).
 %
 \item $b\stackrel{\sss\$}\gets\{0,1\}$.
 %
 \item sets $v$ as follows:  \begin{equation*}
   v= 
\begin{cases}
    sk,              &\text{if } b= 1\\
   r \stackrel{\sss\$}\gets\{0,1\}^{\sss \iota}, & \text{otherwise }\\
\end{cases}
\end{equation*}
 %
 \item returns $v$.
 \end{enumerate}
}}
\end{tcolorbox}
\end{center}
 %\vspace{-3mm}
\caption{Simulators for \execute, \reveal, and \test\ queries. } 
\label{fig::sim-for-exe-rev-test}
% \vspace{-3mm}
\end{figure}



\subsection{Authentication}

In this section, we prove the protocol's authentication. We begin with the case where the adversary $\adv$ has access to the traffic between the two parties and wants to impersonate the user, $U$; we denote such a case with $\bar {aut}$. The Authentication proof relies on the semantic security proof (and games) we presented in Section \ref{sec::semSec-proof}.  Now, we outline the proof.  By definition, it holds that:
%
 \begin{equation}\label{eq::adv-ake}
Adv_{\sss \psi}^{\sss \bar {aut}}(\A)=Pr[Auth_{\sss  0}]
\end{equation}


Also, we can extend Equation \ref{eq::game_1} to:
%
\begin{equation*}
|Pr[Auth_{\sss  1}]-Pr[Auth_{\sss  0}]| \leq (q_{\sss s}+q_{\sss p})Adv^{\sss\prf}(\adv),
\end{equation*}
%
 because the only difference between the two games (i.e., $G_{\sss  0}$ and $G_{\sss  1}$) is that the output of the $\prf$ is replaced with an output of a uniformly random function $f$. 
 
 
 Furthermore, we can extend Equations \ref{eq::game_1}-\ref{eq::game_7} as follows: 
 
 
 \begin{equation*}
 \begin{split}
  &  |Pr[Auth_{\sss  1}]-Pr[Auth_{\sss  0}]| \leq (q_{\sss s}+q_{\sss p}) Adv^{\sss \prf}(\adv) \\ 
 &  |Pr[Auth_{\sss  2}]-Pr[Auth_{\sss 1}]| \leq (q_{\sss s}+q_{\sss p}) Adv^{\sss Enc}(\adv) \\ 
 & Pr[Auth_{\sss  3}]-Pr[Auth_{\sss  2}]=0\\
 & |Pr[Auth_{\sss  4}]-Pr[Auth_{\sss  3}]|\leq \frac{4(q_{\sss  s}+q_{\sss  p})}{2^{\sss \lambda}}\\
 & |Pr[Auth_{\sss  5}]-Pr[Auth_{\sss  4}]|\leq \frac{q_{\sss  s}}{2^{\sss \lambda}}\\
 & |Pr[Auth_{\sss  6}]-Pr[Auth_{\sss  5}]|\leq \frac{q_{\sss  s}}{2^{\sss \lambda}}\\
 & |Pr[Auth_{\sss  7}]-Pr[Auth_{\sss  6}]|\leq \frac{2q_{\sss  s}}{2^{\sss \lambda}}
 %
\end{split}
\end{equation*}

 Moreover, since the authenticator is a random value in $G_{\sss 7}$, it holds that $Pr[Auth_{\sss 7}]=\frac{q_{\sss  s}}{2^{\sss \lambda}}$. We conclude the proof, by summing up the above relations and combining with Equation \ref{eq::adv-ake}: 
 %
  \begin{equation}
 Adv_{\sss \psi}^{\sss \bar {aut}}(\A) = Pr[Auth_{\sss  0}] \leq (q_{\sss s} + q_{\sss p})\Big(Adv^{\sss\prf}(\adv)+Adv^{\sss Enc}(\adv)\Big)+\frac{9q_{\sss  s}+4q_{\sss  p}}{2^{\sss \lambda}}%+ \frac{4q_p}{2^{\lambda}}
 \end{equation}
 
 
Next, we proceed to the case where the adversary is given further access to the PIN, i.e., $\adv$ can also send query  \corrupt($C, 1$). We argue that given such an extra capability does not affect the adversary's advantage and the above analysis (as the protocol and its analysis have relied on the security of the CCA-secure symmetric encryption and $\prf$). Now move on to the case where $\adv$ (a) is given all the parameters stored in the hardware token, and (b) has access to all the traffic between the two parties, i.e., $\adv$ can also send query  $\bar{\text{Cpt}_{\sss  2}} =$ \corrupt$(C, 2)$. We argue that in this case, the upper bound of $\adv$'s advantage will be changed as follows:  $Adv_{\sss\psi, \bar{\text{Cpt}_{\sss  2}}}^{\sss  aut}(\A)\leq  \cfrac{q_{\sss  s}}{N}$. The reason for such a big change is that in this case, $\adv$ has all secret parameters, except the PIN and verifier $\VC{\verifier}$. \footnote{The case where $\adv$ has the additional capability to send query \corrupt($U, 2$) was never discussed and analysed in \cite{BressonCP03}. However, we noticed that $\adv$ in that scheme would have the same upper bound advantage as $\adv$ in our scheme does.} Thus, when we take the forward security into account, the advantage of the adversary (due to the union bound)  is as follows: 
%
  \begin{equation*}
 Adv_{\sss \psi}^{\sss  aut}(\A)  \leq (q_{\sss s} + q_{\sss p})\Big(Adv^{\sss\prf}(\adv)+Adv^{\sss Enc}(\adv)\Big)+\frac{9q_{\sss s}+4q_{\sss  p}}{2^{\sss \lambda}}+  \cfrac{q_{\sss  s}}{N}
 \end{equation*}


\subsection{PIN's Privacy Against A Corrupt Server} 

In the case where the adversary (i) has access to the parties' traffic and (ii) can make query \corrupt($S, 1$), to extract all parameters of the server, then the probability that the adversary can find the valid PIN depends on the probability of finding the correct PIN and finding $U$'s correct key of $\prf$; therefore, the probability is at most $\cfrac{q_{\sss  p}}{2^{\sss \lambda}N}$. 





%!TEX root = main.tex


\section{Evaluation}\label{app:long-eval}

In this section, we analyse and compare the 2FA protocol we presented in Section \ref{sec::the-protocol} with the smart-card-based protocol proposed in  \cite{WangW18}, the hardware token-based protocol in \cite{JareckiJKSS21}, and the symmetric-key-based scheme in  \cite{MatsuoMY11}.  We consider the protocols in  \cite{WangW18,JareckiJKSS21} because they are relatively efficient, do not rely on secure hardware, and consider the same security threats as we do, \ie., resistance against card/token loss, against an offline attack, and against a corrupt server. We also include the scheme in  \cite{MatsuoMY11}  in our analysis because it is highly efficient, only uses symmetric-key primitives, and is in the standard model. We summarise the result of the analysis in Table \ref{comparisonTable}. 

%Before we compare the two protocols' costs, we highlight a vital difference between the two. 


%!TEX root = main.tex


%\vspace{-2mm}
\begin{table} 

\begin{center}
\caption{ \small Comparison of efficient two-factor authentication protocols.}  \label{comparisonTable} 
%\renewcommand{\arraystretch}{1}
\begin{tabular}{|c|c|c|c|c|c|c|c|c|c|} 
\hline

{\cellcolor{gray!40}\scriptsize {Features}} &\cellcolor{gray!40}{\scriptsize {Operation}}&\cellcolor{gray!40}{ \scriptsize {Our Protocol}}&\cellcolor{gray!40}{\scriptsize{\cite{WangW18}}}&\cellcolor{gray!40} {\scriptsize{\cite{JareckiJKSS21}}}  \\
\hline

\multirow{2}{*}{\scriptsize Computation cost}


&\scriptsize{Sym-key}&\cellcolor{gray!20}\scriptsize$18$&\cellcolor{gray!20}\scriptsize$19$&\cellcolor{gray!20}\scriptsize $7$\\


\cline{2-5}

&\scriptsize {Modular expo.}&\cellcolor{gray!20}\scriptsize {$0$}&\cellcolor{gray!20}\scriptsize$5$&\cellcolor{gray!20}\scriptsize$12$\\

\hline

\scriptsize  Communication cost&$-$&\cellcolor{gray!20}\scriptsize$2804$-bit&\cellcolor{gray!20}\scriptsize$3136$-bit&\cellcolor{gray!20}\scriptsize$3900$-bit\\

\hline 
\scriptsize Not requiring multiple pass/PIN&$-$&\cellcolor{gray!20}\scriptsize{\textcolor{blue}\checkmark}&\cellcolor{gray!20}\scriptsize\textcolor{red}{$\times$}&\cellcolor{gray!20}\scriptsize\textcolor{red}{$\times$}\\ 
\hline
\scriptsize Not requiring modular expo.  &$-$&\cellcolor{gray!20}\scriptsize{\textcolor{blue}\checkmark}&\cellcolor{gray!20}\scriptsize\textcolor{red}{$\times$}&\cellcolor{gray!20}\scriptsize\textcolor{red}{$\times$}\\ 

\hline


\scriptsize Security assumption  &$-$&\cellcolor{gray!20}\scriptsize{Standard}&\cellcolor{gray!20}\scriptsize{Random oracle}&\cellcolor{gray!20}\scriptsize{Random oracle}\\ 

\hline
\end{tabular}
%}
%\renewcommand{\arraystretch}{1}
%\end{footnotesize}
\end{center}
%}
\end{table}










%In this section, we analyse and compare the complexities of Feather with those of delegated and traditional PSIs that support multi-client in the semi-honest model.

\subsection{Computation Cost}

We start by analysing our protocol's computation cost. First, we focus on the protocol's enrolment phase. The user's computation cost, in this phase, is as follows. It invokes the authenticated encryption scheme $2$ times. It also invokes once the pseudorandom function, $\prf$.
%
% Thus, the client's computation complexity in this phase is $O(1)$. Next, we analyse the server's computation cost in this phase. 
% 
  Moreover, the server invokes the authenticated encryption scheme twice, and calls $\prf$ only once, in this phase. 
%  
%  Therefore, the server's computation complexity is $O(1)$.  
%
Now, we move on to the authentication phase. The user invokes the authenticated encryption scheme $2$ times and invokes $\prf$ $4$ times. 
%
%So, the client's computation complexity in this phase is $O(1)$. 
%
 In this phase, the server invokes the authenticated encryption scheme and $\prf$ $2$ and $4$ times respectively. %Note that in the normal case when the client receives the server's messages, we would have $d=1$.  

Next, we analyse the computation cost of the protocol in \cite{WangW18}. We consider all operations performed on the smart card or card reader as user-side operations. The enrolment phase involves $3$ and $2$ invocations of a hash function at the user and server sides respectively. This protocol has an additional phase called login which costs the user  $5$ invocations of the hash function and $2$ modular exponentiations for each authentication.  The verification requires the server $6$ invocations of the hash function and $2$ modular exponentiations. This phase requires the user to perform $1$ modular exponentiation and invoke the hash function $3$ times. 

%The above protocol unlike the protocol we proposed does not include any counter, so its complexity is $O(1)$; however, it requires the parties to perform a constant number of modular exponentiations for each verification which ultimately imposes higher overheads cost than the protocol we proposed. 

Now, we analyse the computation cost of the protocol presented in \cite{JareckiJKSS21}. In our analysis, due to the high complexity of this protocol, we estimate the protocol's \emph{minimum} costs. The actual cost of this protocol is likely to be higher than our estimation. The protocol's phases have been divided into enrolment and login, \ie, verification. The enrolment phase requires a user to perform a single modular exponentiation and invoke a hash function $2$ times. It also involves, as a subroutine, the initialisation of the asymmetric  ``password-authenticated key exchange'' (PAKE) proposed in \cite{GentryMR06}, which involves at least $2$ modular exponentiations, $1$ invocation of hash function and symmetric-key encryption. In the login phase, the user performs at least $7$ modular exponentiations. In the login phase, the server invokes a pseudorandom function once and performs at least $2$ modular exponentiations and $2$ symmetric-key encryptions (due to the execution of PAKE). 

Now, we focus on the protocol proposed in \cite{MatsuoMY11}. As we previously mentioned the scheme is highly efficient. The scheme assumes the user and server have already agreed on the user's password. 
%
In total, the user (and the trusted chipset) invokes the verification algorithm of a Message Authentication Code (MAC) scheme once and the pseudorandom function three times. The server cost is similar to the user's cost, with the difference that the server also invokes the tag generator algorithm of the MAC scheme once. The scheme does not involve any modular exponentiations. 

Thus,  our protocol and the ones in \cite{WangW18,JareckiJKSS21,MatsuoMY11} involve a constant number of symmetric-key primitive invocations; however, our protocol and the protocol in \cite{MatsuoMY11} do not involve any modular exponentiations, whereas the protocol in  \cite{WangW18,JareckiJKSS21} involves a constant number of modular exponentiations which leads to a higher cost. 

When executed on the extremely low-power microcontrollers common in dedicated hardware authentication tokens, asymmetric cryptography has an energy consumption of around 1,000 times the corresponding symmetric equivalent~\cite{energyconsumption}.
Such tokens are expected to operate for several years without a change of batteries and often do not have user-replaceable or even rechargeable batteries to reduce manufacturing costs and increase tamper evidence.
The power saving that results from using solely symmetric primitives is therefore important for feasibility.
For the commercial product that this protocol was implemented for, there was no available on the market that was both cost-effective and able to execute asymmetric cryptography with a speed that would result in a good user experience.

\subsection{Communication Cost}


We next analyse our protocol's communication cost. In the enrolment phase,  the user only sends two pairs of messages: $(U_{\text ID}, $ $\text{enrolment})$ and $(M', t')$, where the total size of messages in the first pair is about $250$ bits (assuming the ID is of length $128$ bits),  while the total size of messages in the second pair is about  $512$ bits as they are the outputs of symmetric-key primitives, \ie, symmetric-key encryption and message authentication code schemes whose output size is $256$ bits. The server sends out only a single pair $(M', t')$ whose total size is about $512$ bits. 
% 
The parties' communication cost in the authentication phase is as follows. The user only sends three messages: $(U_{\text ID}, $ $\text{authentication}, \VC{\mathit{response}})$, where the combined size of the first two messages is about $250$ bits while the third message's size is about $256$ bits. The server sends only two pairs of messages $(\ddot M, \ddot t)$ and $(\hat M, \hat t)$ with a total size of $1024$ bits. Therefore, the total communication cost that our protocol imposes is about $2804$ bits. 

Next, we evaluate the cost of the protocol in \cite{WangW18}. The user's total communication cost in the enrolment and login phases is $1792$ bits. Note that we set the user's ID's size to $128$ bits and we set the hash function output size to $160$ bits, as done in \cite{WangW18}. In the verification phase, the user sends to the server a single value of size $160$ bits. In the verification, the server sends to the user two values that in total costs the server $1184$ bits. So, this protocol's total communication concrete cost is about $3136$ bits.%, and its complexity is $O(1)$.  


Now, we analyse the communication cost of the protocol in \cite{JareckiJKSS21}. As before, in our cost evaluation, we estimate the protocol's minimum cost.  In the enrolment phase, a user sends a random key, of a pseudorandom function, to the server and the device, where the size of the key is about $128$ bits. It also, due to the initialisation of PAKE, sends a $128$-bit value to the server. In the login phase, the user sends out three parameters of size $128$ bits and a single parameter of size $20$ bits.  It also invokes PAKE with the server that requires the user to send out at least one signature of size $1024$ bits. The device sends to the user a ciphertext of asymmetric-key encryption which is of size $1024$ bits along with a $20$-bit message. Thus, the user-side total communication cost is at least $2856$ bits. The server in the login phase sends out a message $zid$ of size $20$ bits and invokes PAKE that requires the server to send out at least a ciphertext of symmetric-key encryption which is of size $1024$ bits.  So, the server-side communication cost is at least $1044$ bits. So, the total communication cost of this protocol is at least $3900$ bits. 

Next, we focus on the communication cost of the protocol in \cite{MatsuoMY11}. In total, the user for each authentication sends   $(r_{\sss A}, ID_{\sss A}, Auth_{\sss A})$ to the server. The server also sends $(r_{\sss A}, r_{\sss B}, ID_{\sss B}, Auth_{\sss B})$ to the user, where each message is of size $128$-bit. Thus, the protocol's total communication cost is $896$ bits. 


Hence, our protocol imposes a $10\%$ and $40\%$ lower communication cost than the protocols in \cite{WangW18} and \cite{JareckiJKSS21} that are secure against a corrupted token or server. The protocol in \cite{MatsuoMY11} has the lowest communication cost but it is not secure against a corrupted server. 

When communication is implemented using a 2-D barcode, this reduction in message size as compared to the state of the art allows for an increased level of error correction in the generated barcodes.
Consequently, the decoding process becomes more resilient to inaccurate alignment and issues like reflections on the client screen, therefore improving user experience through a reduction in time needed to capture a valid barcode.

%\subsection{The Number of Secrets a client Must Know}

\subsection{Other Features}
 In our protocol, a user needs to know and type into the token only a single secret, \ie, a PIN. Furthermore, we do not put any trust assumptions on users' personal computers (or clients). Therefore, they can be corrupted at any time, e.g., simultaneously when another party such as the token or server is corrupted. 
 
 In contrast, in the protocol in \cite{WangW18}, a user has to know and type an additional secret, \ie, a random ID. As shown by Scott~\cite{Scott12a}, this scheme will not remain secure, even if only the user's ID is revealed. Also, in this scheme, users type in their PINs into another device, i.e., a card reader. For the security of the scheme holds, it is required that the card reader be fully trusted, in the case where the server or smart card is breached. 
 
  
   Similarly, the protocol in \cite{JareckiJKSS21} requires users to insert their PINs into their personal computers instead of inserting them into the dedicated hardware token; this approach is problematic as the PINs will be at a higher risk of being exposed to attackers because users' computers are (i) often connected to the Internet, and (ii) used for various purposes, consequently, they are more likely to be broken into. Furthermore, the protocol relies on an additional trusted party as well, i.e., trusted users' computers. Specifically, for the protocol's security to hold, it is required that users' personal computers are fully trusted, in the case where the token or the server is corrupted, which is not desirable. In contrast, our protocol requires users to insert their PIN into the hardware token that is used only for authentication and is always disconnected from the Internet. 
 
 In the protocol in \cite{MatsuoMY11} a user needs to know only a single PIN. It requires the server to be fully trusted and it uses a trusted chipset. So, it relies on the strongest security assumption. 
 % 
Our protocol and the one in \cite{MatsuoMY11} are secure in the standard model whereas the protocols in \cite{WangW18,JareckiJKSS21} are in the non-standard random oracle model. 


%%!TEX root = main.tex

%%%%%%%%%%%


\section{Straw-man Solutions}\label{sec::Straw-man-Solutions}
In this section, we provide an overview of a couple of solutions that seem to work and discuss their shortcomings. 

\subsection{Straw-man Proposal I}\label{sec:straw-man-1}

A simple authentication protocol would be for the server to generate a secret key $k$, then enrol a client's device by sharing this key over a secure channel, e.g., loaded onto the device at the time of manufacture.
Then, when the secure channel is not available (e.g., during Internet banking) the server sends a
randomly generated challenge to the client which replies with a Message Authentication Code (MAC)  computed of this challenge, under $k$.
This protocol provides the server assurance
that the response originated from the correct client and
bounds the time at which the response was generated to be between the time that the challenge was sent and when the response was received. 
%
%Note that the nonce must be unpredictable. It would not be sufficient to
%just have a non-repeating value such as a counter. 
%
The resulting protocol is shown in \prettyref{fig:strawman1}.  An alternative design would be to omit the challenge message containing
the nonce. For example, we could compute the MAC of a counter. However, this protocol is vulnerable to a pre-play attack, where a
response is collected and replayed at a later time. Alternatively, the
MAC could be computed of a timestamp. This approach
gives the server assurance of when the response was generated.  But, it
requires the device to have a real-time clock. Doing so
would increase power requirements and limit the device's lifespan because
the battery could not be replaced by the client without desynchronising
the clock. Alternatively, a backup battery could be included, but this
would significantly increase the device cost.

All of these protocols/approaches have a major weakness; namely, if the device is stolen, then the adversary can generate valid
authentication responses. %To resist this attack, we can augment the protocol to verify the identity of the client.






\begin{figure}[!thb]
\begin{tcolorbox}[enhanced,width=4.75in, height=100mm, left=1mm,
    drop fuzzy shadow southwest,
    colframe=black,colback=white]
    {\small{
 %\centering

 \procedure{}{%
 \textbf{Client} \<  \textbf{Server} \pclb
  \pcintertext[dotted]{Setup (one-time)} \\ 
  %\< \< \key \sample \kgen(\secparam) \pclb
    \<  k \stackrel{\sss \$}\leftarrow \{0,1\}^{\sss \lambda} \pclb
  \< \sendmessageleft*{\text{\key{} shared over a secure channel}} \< \\
  \text{store}\ k \<  \text{store}\ k \pclb
  \pcintertext[dotted]{Authentication (for each transaction)} \\
  \<  \nonce{} \stackrel{\sss \$}\leftarrow \bin^{\sss\secpar} \pclb
  \< \sendmessageleft*{\nonce} \< \\
  \text{\textit{response}} \gets \mac_{\sss k}(\nonce) \<  \pclb
  \< \sendmessageright*{response} \< \\
  \<  \text{\textit{expected}} \gets \mac_{\sss k}(\nonce) \\
  \<\hspace{-9mm}    \pcif \textit{response} \ne \textit{expected} \pcthen \pcfail
   }
    \caption{Straw-man protocol I.}
    
    \label{fig:strawman1}
    }}
    \end{tcolorbox}
\end{figure}




\subsection{Straw-man Proposal II}\label{sec:straw-man-2}

We extend the previous challenge-response protocol to compute the response
over both the challenge and a client's PIN, to prevent an adversary who
has stolen the device from completing the authentication phase. This creates a
2FA scheme, depending on something the client has (i.e., the
authentication device) and something the client knows (i.e., the PIN). The
server can compute the
expected response and validate the response produced by the device. If this validation succeeds, then the server has the
same assurances of Straw-man Proposal 1 and additionally knows that the correct
PIN was entered into the device. To this end, we could store the PIN on the server. 

Nevertheless, it is undesirable for the server to know the PIN, as the client may
use the same PIN for other unrelated purposes. Having the server store
the hash of the PIN would not help because the low entropy of a
convenient PIN (around 13 bit for a 4-digits) is trivially vulnerable to
a brute-force pre-image attack. We can avoid this problem by replacing
the PIN in the protocol with a verifier, which is the output of a PRF computed over the PIN
under a secret key held only by the device. This
verifier must be sent to the server when the device is enrolled. Given
the correct PIN, the device could compute the verifier. In this case, a corrupt server
would not be able to recover the PIN from the verifier, without
knowledge of the secret key.
%
 We can incorporate a description of the transaction into the challenge
message and computation of the authentication response. This transaction
is generated by the server to indicate to the client what action will be
performed if the authentication succeeds. The resulting protocol is shown in 
\prettyref{fig:strawman2}.



An alternative protocol design would be to store the PIN on the device,
and for the device to only permit the authentication key $k$ to be used if
the PIN is entered correctly. For this design to be
resistant to an adversary who has stolen the device, it must not be feasible to extract the PIN and  must not be feasible to bypass the PIN verification. This functionality requires security features not commonly
available on low-cost microcontrollers. Security assured co-processors
are available, but would substantially increase the cost of the device. %, the required circuit-board area, and could still be vulnerable to attack. 
%
The protocol above meets many desirable criteria for an authentication
protocol. Specifically, a verified authentication response gives the server assurance that (a)
 the device is present, due to the random nonce, (b)
 the correct device was used, due to the use of the key, (c) the device has
not been stolen, due to the PIN, and (d) the client saw the transaction that
the server is about to perform, due to the inclusion of the transaction's description within the MAC computation. The protocol does not require
the device to have a real-time clock, so a single client-replaceable battery may be used. The PIN is also not stored by the device and so no
special tamper-resistant hardware is necessary. There is no need to
protect the authentication key against physical tampering because anyone
with access to the device could simply use the device to perform
authentication.

There is still a remaining serious risk. Let us suppose that the adversary has
recorded a valid authentication response and the corresponding
challenge. The adversary who has access to the device can extract the authentication key. Now, the
adversary has all the information needed to locally brute-force the short PIN, and then generate a valid response to any future authentication challenge
from the server. %In the reminder of the paper we will explore how to mitigate this vulnerability without the need of additional hardware.
%%%%%%%%%


\begin{figure}[!htb]
\begin{tcolorbox}[enhanced,width=4.75in, height=151mm, left=1mm,top=.1mm,
    drop fuzzy shadow southwest,
    colframe=black,colback=white]
 \procedure{}{%
 \textbf{Client} \< \textbf{Server} \pclb
  \pcintertext[dotted]{Setup (one-time)} \\
  \<   k \stackrel{\sss \$}\leftarrow \{0,1\}^{\sss \lambda} \pclb
  \< \sendmessageleft*{k\text{\ shared over a secure channel}} \< \\
  \text{store}\ k \<  \text{store}\ k \pclb
  \pcintertext[dotted]{Enrolment (one-time)} \\ 
  \salt \stackrel{\sss \$}\leftarrow \{0,1\}^{\sss \lambda} \< \< \\
  \text{request \pin{} from client} \< \< \\
  \text{\textit{v}} \gets \prf_\salt(\pin) \< \< \pclb
  \< \sendmessageright*{\text{\textit{v} shared over a secure channel}} \< \\
  \<  \text{store \textit{v}} \pclb
  \pcintertext[dotted]{Authentication (for each transaction)} \\
  \<  \nonce{} \stackrel{\sss \$}\leftarrow \bin^{\sss\secpar} \pclb
  \< \sendmessageleft*{\nonce, t} \< \\
  \text{request \pin{} from client} \< \< \\
  v \gets \prf_\salt(\pin) \< \< \\
  \text{\textit{response}} \gets \mac_{\sss k}(\nonce, t, v) \< \< \pclb
  \< \sendmessageright*{response} \< \\
  \<\hspace{-14mm}  \text{\textit{expected}} \gets \prf_{\sss k}(\nonce, t, v) \\
  \< \hspace{-16mm} \pcif \textit{response} \ne \textit{expected} \pcthen \pcfail
   }
    \caption{Straw-man protocol II.}
    \label{fig:strawman2}
    
    \end{tcolorbox}
\end{figure}







\end{document}
