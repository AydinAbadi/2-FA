%!TEX root = main.tex

\vspace{-3mm}
\section{Related Work}

\vspace{-2mm}



%In order for clients to prove their identity to a (local/remote) computer resource, they provide a piece of evidence, called an ``authentication factor''. The authentication factors can be broadly categorised based on (a) what clients know (a.k.a. knowledge factors), such as PIN, password, (b)  what a client possesses (a.k.a. possession factors), such as an access card or physical hardware token, or (c) what clients are (a.k.a. inherent factors), such as a fingerprint, or iris. 
%
%
%The knowledge factors are still the most predominant factors used for authentication \cite{bonneau2010password,JacommeK21}, mainly because their implementations impose low costs on the verifiers and require clients to invest minimal efforts to authenticate themselves. However, the knowledge factors themselves are not strong enough to adequately prevent impersonation, for various reasons, e.g., clients may pick weak passwords that can be easily guessed, or reuse them to authenticate themselves to many servers which would increase their leakage if one of the servers is penetrated by adversaries \cite{SinigagliaCCZ20,JacommeK21}. 
%
%In general, authentication methods that depend on more than one factor are more difficult to compromise
%than single-factor methods. Although the authentication mechanisms that rely on more than two factors may guarantee stronger security, they would impose higher costs and negatively affect clients' experience; therefore, two-factor authentications have gained special attention; including those that rely on a combination of password/PIN and device possession that allows clients to generate one-time passwords (OTPs) when a client inserts its secret (e.g., password or PIN) into the device. An OTP can be sent to and checked by a remote verifier to authenticate a client. The combination of a password or PIN and a hardware token has gained popularity among various e-commerce and banks.  


In this section, first, we briefly discuss the common approaches for generating a One-Time Password (OTP) which yields from a combination of a PIN and a hardware token.  Then, we provide an overview of hardware token variants. We refer readers to Appendix \ref{Survey-of-Related-Work} for a more detailed discussion. 

\vspace{-4mm}
\subsection{Common Approaches for Generating OTP}
\vspace{-2mm}

In the authentication that relies on a combination of knowledge and possession factors, once the client enters the secret into the hardware token, the device combines this secret with the output of one of the following methods to generate a unique OTP:  (i) a random challenge, sent by the server to the device (through the client);  (ii) an internal counter maintained by the server and device, or (iii) the current accurate time, kept by the server and device. There exist 2FA solutions (including ours) that employ a combination of the above approaches. 

\vspace{-4mm}
\subsection{Variants of OTP Hardware Tokens}

\vspace{-2mm}
\subsubsection{Connected Tokens.}
This type of token requires a client to physically connect the token to their computer via which the client is authenticating. After that, the device transmits the authentication information to the computer. USB tokens and smart cards are two popular token technologies in this category.  Various companies including Google and  ``Fast IDentity Online'' (FIDO) Alliance have developed USB hardware tokens. However, researchers have discovered many vulnerabilities within this standard, e.g., in \cite{PanosMNPX17,ChangMSS17,LoutfiJ15,ndss/FengLP021}.  
%
Since the introduction of smart card technology, there have been numerous smart card-based 2FA protocols (e.g., in \cite{gupta2021machine,WangW18,radhakrishnan2022dependable,kim2009more}). But, existing smart card-based solutions either use public-key cryptography which imposes a high computation cost or tamper-proof secure chipsets embedded in the card which would increase the device's cost. 


%There are a number of different types, including USB tokens, smart cards and wireless tags.[7] Increasingly, FIDO2 capable tokens, supported by the FIDO Alliance and the World Wide Web Consortium (W3C), have become popular with mainstream browser support beginning in 2015.




\vspace{-4mm}
\subsubsection{Disconnected Tokens.}

This type of token does not have a physical connection to a client's computer making them more convenient than connected tokens. Two main categories of disconnected tokens are (1) Dedicated hardware-based Tokens (e.g., in \cite{secureID,Digipass-website,Gemalto}) and (2) Mobile phone-based Tokens (e.g., in \cite{SARA22,KoganMB17,KonothFFARB20}). The first category includes RSA SecureID \cite{secureID}, OneSpan Digipass 770 \cite{Digipass-website}, and Thales Gemalto SWYS QR Token Eco \cite{Gemalto}. In RSA SecureID, an adversary who has access to the device can generate the OTP by extracting the secret key stored on the device.  The advantage of  Digipass 770 and Thales Gemalto to RSA SecureID is that they let clients see the transaction details through the token which gives them more information about the transaction they approve, so phishing becomes harder. 
%
Our investigation suggests that Digipass 770 and Thales Gemalto also \emph{locally store and verify} clients' PINs. Jarecki \textit{et al.} \cite{JareckiJKSS21} proposed a protocol to ensure that even if the server or device is corrupted a client's PIN cannot be extracted. But, it imposes high costs due to the use of public-key cryptography and numerous rounds of communication. This protocol requires the client (in addition to remembering its PIN) to locally store a cryptographic secret key. 



%Thus, unlike the above solutions, offer all the following features simultaneously; it (a) efficient, (b) is provably secure, (c) is resilient against the adversary which may corrupt the server or have access to the client's token, (d) relies on a single server, and (e) allows the client to also check the transaction detail on its token. 




%

Solutions in the second category use a mobile phone as a hardware token to generate OTP. They often rely on the added features that mobile phones offer, such as possessing a Trusted Execution Environment (TEE) or being able to communicate directly with the server. The mobile phone-based scheme in  \cite{KoganMB17} uses a combination of time-based OTP and a hash chain. It ensures that even if the adversary corrupts the server, it cannot extract the client's secret. Nevertheless, it requires: (a) the client to store a long secret key (on the mobile phone), (b)  the laptop/PC that the client uses to be equipped with a camera, and (c)  the mobile phone to invoke a hash function over a million times that can cause the phone's battery to run out fast. The protocol proposed in \cite{KonothFFARB20} relies on a phone's TEE  and messages that the server can directly send to the phone. Later,  Imran \textit{et al.} \cite{SARA22} proposes a protocol that also relies on a phone's TEE, but it improves the protocol presented in \cite{KonothFFARB20}, in that it is compatible with more android devices and supports biometric authentication.  A primary limitation of mobile phone-based OTP tokens is that they cannot be used when there is no (phone) network coverage. Also, in certain cases sharing phone numbers with the server may not suit all clients, e.g., transactions' details along with the phone number might be sold for targeted advertisements. 
%





%
%. Disconnected tokens are the most common type of security token used (usually in combination with a password) in two-factor authentication for online identification



% We refer readers to \cite{SinigagliaCCZ20} for a survey of different methods of authentications and their adoptions by banks.  




%Therefore, instead of using only a (clientname and) password/PIN, clients need to prove the possession of an additional device and/or an inherent factor. 